%% To do: trim these definitions

\newcommand\nc\newcommand
\nc\rnc\renewcommand

\usepackage{epsfig}
\usepackage{latexsym}
%% \usepackage{bbold}  %% for $\mathbbm{1}$

%% %% Look for the \smalltriangleup definition. I expected it in amssymb
%% \nc\smalltriangleup{\scriptscriptstyle \triangle}
%% \nc\smalltriangledown{\scriptscriptstyle \triangledown}

\nc\out[1]{}

\nc\mynoteOut[2]{\mynote{#1}\out{#2}}

% While working, use these defs
%% \nc\mynote[1]{{\em [#1]}}
%% \nc\mynotefoot[1]{\footnote{\mynote{#1}}}
% But for the submission, use these
\nc\mynote\out
\nc\mynotefoot\out

\nc\todo{\mynote{To do.}}

\nc\figlabel[1]{\label{fig:#1}}
\nc\figref[1]{Figure~\ref{fig:#1}}

\nc\seclabel[1]{\label{sec:#1}}
\nc\secref[1]{Section~\ref{sec:#1}}
\nc\secreftwo[2]{Sections~\ref{sec:#1} and~\ref{sec:#2}}

\nc\appref[1]{Appendix~\ref{sec:#1}}

%% The name \secdef is already taken
\nc\sectiondef[1]{\section{#1}\seclabel{#1}}
\nc\subsectiondef[1]{\subsection{#1}\seclabel{#1}}
\nc\subsubsectiondef[1]{\subsubsection{#1}\seclabel{#1}}

\nc\needcite{\mynote{ref}}

% \nc\myurl\texttt

% http://cs.wlu.edu/~necaise/refs/latex2e/env-floats.3.html#lnfigure

% Arguments: env, label, caption, body
\nc\figdefG[4]{\begin{#1}[tbp]
#4
\caption{#3}
\figlabel{#2}
\end{#1}}

% Arguments: label, caption, body
\nc\figdef{\figdefG{figure}}
\nc\figdefwide{\figdefG{figure*}}

% Arguments: label, caption, body
\nc\figrefdef[3]{\figref{#1}\figdef{#1}{#2}{#3}}

\nc\figrefdefwide[3]{\figref{#1}\figdefwide{#1}{#2}{#3}}


% Image format: PNG or JPEG?  JPEG lets us shrink the files, at some cost
% in fidelity.  Png is much slower to process even when the files are
% smaller.  I guess there's some conversion process going on.
% JPEG compressed at 35x, the figures are smaller and faster to
% process than png.  The eps files are huge (about 80x). 
% Since PNG is lossless, keep the master figures in that format and convert.


\nc\picext{png}
%\nc\picext{jpg}
%\nc\picext{eps}
%\nc\picext{tif}

\nc\picfile[1]{pictures/\picext/#1.\picext}

\nc\pict[1]{\includegraphics[width=3.2in]{\picfile{#1}}}

\nc\picframe[1]{\frame{\pict{#1}}}
\nc\picframewide[1]{\frame{\includegraphics[width=6in]{\picfile{#1}}}}

\nc\picdef[2]{\figdef{#1}{#2}{\centering \picframe{#1}}}
\nc\picdefwide[2]{\figdefwide{#1}{#2}{\picframewide{#1}}}

\nc\picrefdef[2]{\picdef{#1}{#2}\figref{#1}}
\nc\picrefdefwide[2]{\figref{#1}\picdefwide{#1}{#2}}

\nc\figneeded[1]{\figdef{needed}{#1}}

\nc\symTwo[1]{\mathbin{#1\!\!\!#1}}
\nc\symThree[1]{\mathbin{#1\!\!\!#1\!\!\!#1}}

\nc{\lb}{[\![}
\nc{\rb}{]\!]}
\nc{\db}[1]{\lb#1\rb}

\usepackage{tikz}
%%%<
%% \usepackage[active,tightpage]{preview}
%% \PreviewEnvironment{tikzpicture}
%%%>
\usepackage{amsmath}

%% \usepackage{tikz-cd}

%% % I used tikz-cd version 0.3c, of December 30, 2012.
%% % July 2012 version lacks a feature I use (shift left).

%% \usetikzlibrary{matrix, calc, arrows}
%% \tikzset{ ampersand replacement=\& }

\nc\wpicture[2]{\includegraphics[width=#1]{#2}}

\nc\wfig[2]{
\begin{center}
\wpicture{#1}{#2}
\end{center}
}
\nc\fig[1]{\wfig{4in}{#1}}

\nc\usebg[1]{\usebackgroundtemplate{\wpicture{1.2\textwidth}{#1}}}

\nc\framet[2]{\frame{\frametitle{#1}#2}}

\nc\hidden[1]{}

\nc\Back{\!\!\!\!\!\!\!}
