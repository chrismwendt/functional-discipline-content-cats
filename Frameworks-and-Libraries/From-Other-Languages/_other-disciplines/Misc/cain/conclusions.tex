\chapter{Conclusions}
\label{chap:conclusions}

\begin{epigraphs}
\qitem{
  ``What dreadful nonsense we \emph{are} talking!''
}{---\textcite[255]{carroll-2004}}
\qitem{
  ``You may call it `nonsense' if you like, but \emph{I've} heard
  nonsense, compared with which that would be as sensible as a
  dictionary!''
}{---\textcite[173]{carroll-2004}}
\end{epigraphs}

Our main objective with this project was to study some of the
applications of category theory to functional programming,
particularly in Haskell and Agda, and, more specifically, to describe
and explain the concepts of category theory needed for conceptualizing
and better understanding functors, polymorphism, monads, and algebraic
data types, which we did in Chapters \ref{chap:functors},
\ref{chap:naturals}, \ref{chap:monads}, and \ref{chap:algebras},
respectively. In Chapter \ref{chap:categories}, we identified
categories as the starting point for relating category theory to
functional programming. In the case of algebraic data types and, more
usefully, folds, we identified algebras and initial algebras over
endofunctors as the required concepts for satisfying our main goal; in
the case of functors, the notions of functor and endofunctor; in the
case of monads, the concepts of monad and Kleisli triple; and, in the
case of polymorphism or, more precisely, parametric polymorphism,
natural transformations.

Obviously, we did not cover all of category theory. For instance, we
did not deal with concepts such as adjoints, epimorphisms, limits,
monomorphisms, and universal constructions, which were listed in the
project proposal, but did not answer our purpose. Having said that, we
did cover the trinity of concepts category, functor, and natural
transformation, which is the foundation of all category theory
\parencite[vii]{maclane-1998}, and which creates an opportunity for a
deeper understanding of the subject.

Additionally, some of the applications of category theory to
functional programming are not as straightforward as suggested here.
For example, polymorphic functions actually correspond to lax natural
transformations \parencite[350]{wadler-1989}, and algebraic data types
in Haskell correspond to initial algebras and terminal coalgebras over
endofunctors \parencite[§ 2]{vene-2000}, but such concepts go beyond
the scope of this project. However, our use of category theory seems
to be appropriate and useful, especially from the standpoint of
functional programming.

Although subjective, we believe that this project provides some
interesting examples of how to take advantage of category theory in
functional programming and programming in general, as well as a way to
become a better programmer.

Needless to say, the ideas of category theory might be difficult to
understand at first. As a matter of fact,
\textcite[25]{bird-demoor-1997} claim that ``one does not so much
learn category theory as absorb it over a period of time.'' We claim
that it is definitely worth it.

\section{Future Work}
\label{sec:future}

All unanswered questions and concepts beyond the scope of this project
could be considered as suggestions for future work. For instance, the
questions of Haskell's and Agda's categories, the existence of initial
algebras over endofunctors, and others. We describe some ideas which
we find interesting and appropriate.

\subsection{Adjoints}
\label{sec:future-adjoints}

Category theory is based on the concepts of categories, functors, and
natural transformations. Even though these ideas are important, a
fundamental notion of category theory is adjoints
\parencite[11]{marquis-2013}, which ``arise everywhere''
\parencite[vii]{maclane-1998}. Taking into account our approach, can
we study the applications of adjoints with the purpose of better
understanding functional programming? Based on \parencites[§
  13]{barr-wells-2012}[79--81]{elkins-2009}[§~2.4]{pierce-1991}{rydeheard-1986-adjunctions}[§
  6]{rydeheard-burstall-1988}, the answer seems to be yes. In
addition, \textcite[§ 9]{awodey-2010} and
\textcite[§~IV]{maclane-1998} seem to offer a good starting point.

\subsection{Applicative functors}
\label{sec:future-monoidals}

Based on \parencite{mcbride-paterson-2008}, we identified and studied
monoidal categories and functors in order to be able to understand
applicative functors from a category-theoretical point of view. We
could use our results in a future project, which seems to be a very
relevant next step, particularly in the context of the ``current, and
very likely to succeed,'' Haskell 2014 \texthaskell{Applicative =>
  Monad}
proposal\footnote{\url{http://www.haskell.org/haskellwiki/Functor-Applicative-Monad_Proposal}.},
which adds an \texthaskell{Applicative} constraint to the
\texthaskell{Monad} type class and promotes \texthaskell{join} to
\texthaskell{Monad}, which we briefly discussed in Remark
\ref{re:monad-bind}.

\subsection{Categories}
\label{sec:future-categories}

In Section \ref{sec:category-haskell}, we described \hask, the
category of Haskell types and functions, in order to be able to relate
category theory to functional programming. We could use the
\texthaskell{Category} type class instead
\parencites[49--51]{yorgey-2009}[74--75]{elkins-2009}:
\begin{codehaskell}
class Category cat where
  id  :: cat a a
  (.) :: cat b c -> cat a b -> cat a c

instance Category (->) where
  id  = Prelude.id
  (.) = (Prelude..)
\end{codehaskell}
Likewise, we could study Kleisli categories
\parencite[59--60]{moggi-1991} in terms of this type class, which
might be a more intuitive way to justify the Kleisli triple laws. In
addition, the \texthaskell{Category} class would lead us to
\texthaskell{Arrow}, which is a generalization of functions
\parencite[51--57]{yorgey-2009}.

\subsection{Folds}

In Chapter \ref{chap:algebras}, we examined catamorphisms and their
relation to the \texthaskell{foldr} function for lists. Can we apply
the results of that chapter to \texthaskell{foldl} and folds in
general? In particular, can we use algebras and initial algebras over
endofunctors for conceptualizing the \texthaskell{Foldable}
\parencite[44--47]{yorgey-2009} and \texthaskell{Traversable}
\parencites[§ 3]{mcbride-paterson-2008}[47--49]{yorgey-2009} type
classes?

\subsection{Monoids}
\label{sec:future-monoids}

As an alternative to the ideas of Section \ref{sec:future-adjoints}, a
``fundamental notion of category theory is that of a monoid''
\parencite[vii]{maclane-1998}, as described in Example
\ref{ex:category-monoid}. In Haskell, monoids are defined by the
\texthaskell{Monoid} type class:
\begin{codehaskell}
class Monoid a where
  mempty  :: a
  mappend :: a -> a -> a
\end{codehaskell}
In connection with Remark \ref{re:monad-monoid} and Section
\ref{sec:future-monoidals}, studying monoids and their relation to
monads seems like a pertinent complement to this project. As basis for
this study, we have identified \parencites[§
  VII]{maclane-1998}[39--44]{yorgey-2009}.

\clearemptydoublepage
