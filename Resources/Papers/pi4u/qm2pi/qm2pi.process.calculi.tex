\section{Concurrent process calculi and spatial logics }\label{sec:concurrent_process_calculi_and_spatial_logics_} % (fold)
In the last thirty years the process calculi have matured into a
remarkably powerful analytic tool for reasoning about concurrent and
distributed systems. Process-calculus-based algebraical specification of
processes began with Milner's Calculus for Communicating Systems (CCS)
\cite{MilnerCCS80} and Hoare's Communicating Sequential Processes
(CSP) \cite{CSP} \cite{CSP1} \cite{CSP2} \cite{CSP3}, and continue
through the development of the so-called mobile process calculi,
e.g. Milner, Parrow and Walker's $\pi$-calculus \cite{ParrowWalker},
Cardelli and Caires's spatial logic \cite{CairesC04} \cite{CairesC03}
\cite{Caires04}, or Meredith and Radestock's reflective calculi
\cite{MeredithR05} \cite{meredith2005rho}. The process-calculus-based
algebraical specification of processes has expanded its scope of
applicability to include the specification, analysis, simulation and
execution of processes in domains such as:

\begin{itemize}
\item telecommunications, networking, security and application level protocols
\cite{AbadiB02} 
\cite{AbadiB03} 
\cite{BrownLM05} 
\cite{LaneveZ05}; 
\item programming language semantics and design
\cite{BrownLM05}
\cite{djoin}
\cite{JoCaml}
\cite{WojciechowskiS99};
\item webservices
\cite{BrownLM05}
\cite{LaneveZ05}
\cite{MeredithB03};
\item and biological systems
\cite{Cardelli04}
\cite{DanosL03}
\cite{RegevS03}
\cite{PriamiRSS01}.
\end{itemize}

Among the many reasons for the continued success of this approach are
two central points. First, the process algebras provide a
compositional approach to the specification, analysis and execution of
concurrent and distributed systems. Owing to Milner's original
insights into computation as interaction \cite{Milner93}, the process
calculi are so organized that the behavior ---the semantics--- of a
system may be composed from the behavior of its components
\cite{Fokkink}. This means that specifications can be constructed in
terms of components ---without a global view of the system--- and
assembled into increasingly complete descriptions.

The second central point is that process algebras have a potent proof
principle, yielding a wide range of effective and novel proof
techniques \cite{MilnerS92} \cite{SangiorgiWalker} \cite{Sangiorgi95}
\cite{hop}. In particular, \emph{bisimulation} encapsulates an effective
notion of process equivalence that has been used in applications as
far-ranging as algorithmic games semantics
\cite{Abramsky2005Algorithmic-Gam} and the construction of
model-checkers \cite{Caires04}. The essential notion can be stated in
an intuitively recursive formulation: a \emph{bisimulation} between two
processes $P$ and $Q$ is an equivalence relation $E$ relating $P$
and $Q$ such that: whatever action of $P$ can be observed, taking it
to a new state $P'$, can be observed of $Q$, taking it to a new state
$Q'$, such that $P'$ is related to $Q'$ by $E$ and vice versa. $P$ and
$Q$ are \emph{bisimilar} if there is some bisimulation relating
them. Part of what makes this notion so robust and widely applicable
is that it is parameterized in the actions observable of processes
$P$ and $Q$, thus providing a framework for a broad range of
equivalences and up-to techniques \cite{milner92techniques} all governed by the same core
principle \cite{SangiorgiWalker} \cite{Sangiorgi95} \cite{hop}.
% section concurrent_process_calculi_and_spatial_logics_ (end)