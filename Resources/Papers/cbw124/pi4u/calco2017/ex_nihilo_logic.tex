% ------------------------------------------------------------------------
% ENTCS LaTeX Paper ******************************************************
% ------------------------------------------------------------------------
% Submitted:      Dec 15 2004
% Final Version:  
% Accepted:       
% ------------------------------------------------------------------------
% This is a journal top-matter template file for use with AMS-LaTeX.
%%%%%%%%%%%%%%%%%%%%%%%%%%%%%%%%%%%%%%%%%%%%%%%%%%%%%%%%%%%%%%%%%%%%%%%%%%

%\documentclass{tran-l}
%\documentclass[twocolumn]{amsart}
%\documentclass[]{amsart}
%\documentclass[]{llncs}

\documentclass[]{entcs}
\usepackage[]{prentcsmacro}

%\usepackage[active]{srcltx} % SRC Specials for DVI Searching
\usepackage{url}

% From Allen's stable.
\usepackage{bigpage}
\usepackage{bcprules}
\usepackage{code}
\usepackage{amsfonts}
\usepackage{latexsym}
\usepackage{amssymb}
%\usepackage{caption}
\usepackage{multicol}

% Double brackets
\newcommand{\ldb}{[\![}
\newcommand{\rdb}{]\!]}
\newcommand{\lliftb}{\langle\!|}
\newcommand{\rliftb}{|\!\rangle}
% \newcommand{\lpquote}{\langle}
% \newcommand{\rpquote}{\rangle}
% \newcommand{\lpquote}{\lceil}
% \newcommand{\rpquote}{\rceil}
\newcommand{\lpquote}{\ulcorner}
\newcommand{\rpquote}{\urcorner}
\newcommand{\newkw}{\nu}

% SYNTAX
\newcommand{\id}[1]{\texttt{#1}}
\newcommand{\none}{\emptyset}
\newcommand{\eps}{\epsilon}
\newcommand{\set}[1]{\{#1\}}
\newcommand{\rep}[2]{\id{\{$#1$,$#2$\}}}
\newcommand{\elt}[2]{\id{$#1$[$#2$]}}
\newcommand{\infinity}{$\infty$}

\newcommand{\pzero}{\mathbin{0}}
\newcommand{\seq}{\mathbin{\id{,}}}
\newcommand{\all}{\mathbin{\id{\&}}}
\newcommand{\choice}{\mathbin{\id{|}}}
\newcommand{\altern}{\mathbin{\id{+}}}
\newcommand{\juxtap}{\mathbin{\id{|}}}
\newcommand{\concat}{\mathbin{.}}
\newcommand{\punify}{\mathbin{\id{:=:}}}
\newcommand{\fuse}{\mathbin{\id{=}}}
\newcommand{\scong}{\mathbin{\equiv}}
\newcommand{\nameeq}{\mathbin{\equiv_N}}
\newcommand{\alphaeq}{\mathbin{\equiv_{\alpha}}}
\newcommand{\names}[1]{\mathbin{\mathcal{N}(#1)}}
\newcommand{\freenames}[1]{\mathbin{\mathcal{FN}(#1)}}
\newcommand{\boundnames}[1]{\mathbin{\mathcal{BN}(#1)}}
%\newcommand{\lift}[2]{\texttt{lift} \; #1 \concat #2}
\newcommand{\binpar}[2]{#1 \juxtap #2}
\newcommand{\outputp}[2]{#1 \id{[} #2 \id{]}}
\newcommand{\prefix}[3]{#1 \id{(} #2 \id{)} \concat #3}
\newcommand{\lift}[2]{#1 \lliftb #2 \rliftb}
\newcommand{\quotep}[1]{\lpquote #1 \rpquote}
\newcommand{\dropn}[1]{\rpquote #1 \lpquote}

\newcommand{\newp}[2]{\id{(}\newkw \; #1 \id{)} #2}
\newcommand{\bangp}[1]{\id{!} #1}

\newcommand{\substp}[2]{\id{\{} \quotep{#1} / \quotep{#2} \id{\}}}
\newcommand{\substn}[2]{\id{\{} #1 / #2 \id{\}}}

\newcommand{\psubstp}[2]{\widehat{\substp{#1}{#2}}}
\newcommand{\psubstn}[2]{\widehat{\substn{#1}{#2}}}

\newcommand{\applyp}[2]{#1 \langle #2 \rangle}
\newcommand{\absp}[2]{\id{(} #1 \id{)} #2}

\newcommand{\transitions}[3]{\mathbin{#1 \stackrel{#2}{\longrightarrow} #3}}
\newcommand{\meaningof}[1]{\ldb #1 \rdb}

\newcommand{\Proc}{\mathbin{Proc}}
\newcommand{\QProc}{\quotep{\mathbin{Proc}}}

\newcommand{\entailm}{\mathbin{\vdash_{\mathfrak m}}} %matching
\newcommand{\entailp}{\mathbin{\vdash_{\mathfrak p}}} %behavioral
\newcommand{\entailv}{\mathbin{\vdash_{\mathfrak v}}} %validation
\newcommand{\congd}{\mathbin{\equiv_{\mathfrak d}}}
\newcommand{\congs}{\mathbin{\equiv_{\mathfrak s}}}
\newcommand{\congp}{\mathbin{\equiv_{\mathfrak p}}}
%\newcommand{\logequiv}{\mathbin{\leftrightarrow}}

\newcommand{\barb}[2]{\mathbin{#1 \downarrow_{#2}}}
\newcommand{\dbarb}[2]{\mathbin{#1 \Downarrow_{#2}}}

% From pi-duce paper
\newcommand{\red}{\rightarrow}
\newcommand{\wred}{\Rightarrow}
\newcommand{\redhat}{\hat{\longrightarrow}}
\newcommand{\lred}[1]{\stackrel{#1}{\longrightarrow}} %transitions
\newcommand{\wlred}[1]{\stackrel{#1}{\Longrightarrow}}

\newcommand{\opm}[2]{\overline{#1} [ #2 ]} % monadic
\newcommand{\ipm}[2]{{#1} ( #2 )} 
\newcommand{\ipmv}[2]{{#1} ( #2 )} % monadic
\newcommand{\parop}{\;|\;}		% parallel operator
\newcommand{\patmatch}[3]{#2 \in #3 \Rightarrow #1}
\newcommand{\sdot}{\, . \,}		% Space around '.'
\newcommand{\bang}{!\,}
%\newcommand{\fuse}[1]{\langle #1 \rangle}		
\newcommand{\fusion}[2]{#1 = #2} % fusion prefix/action
\newcommand{\rec}[2]{\mbox{\textsf{rec}} \, #1. \, #2}
\newcommand{\match}[2]{\mbox{\textsf{match}} \; #1 \; \mbox{\textsf{with}} \; #2}
\newcommand{\sep}{:}
\newcommand{\val}[2]{\mbox{\textsf{val}} \; #1 \; \mbox{\textsf{as}} \; #2}

\newcommand{\rel}[1]{\;{\mathcal #1}\;} %relation
\newcommand{\bisim}{\stackrel{.}{\sim}_b} %bisimilar
\newcommand{\wb}{\approx_b} %weak bisimilar
\newcommand{\bbisim}{\stackrel{\centerdot}{\sim}} %barbed bisimilar
\newcommand{\wbbisim}{\stackrel{\centerdot}{\approx}} %weak barbed bisimilar
\newcommand{\bxless}{\lesssim}	%expansion less (amssymb required)
\newcommand{\bxgtr}{\gtrsim}	%expansion greater (amssymb required)
\newcommand{\beq}{\sim}		%barbed congruent
\newcommand{\fwbeq}{\stackrel{\circ}{\approx}}	%weak barbed congruent
\newcommand{\wbeq}{\approx}	%weak barbed congruent
\newcommand{\sheq}{\simeq}	%symbolic hypereq
\newcommand{\wbc}{\approx_{cb}}

% rho logic

\newcommand{\ptrue}{\mathbin{true}}
\newcommand{\psatisfies}[2]{#1 \models #2}
\newcommand{\pdropf}[1]{\rpquote #1 \lpquote}
\newcommand{\plift}[2]{#1 \lliftb #2 \rliftb}
\newcommand{\pprefix}[3]{\langle #1 ? #2 \rangle #3}
\newcommand{\riff}{\Leftrightarrow}

\newcommand{\Formula}{\mathbin{Form}}

% End piduce contribution

\newcommand{\typedby}{\mathbin{\:\colon}}
\newcommand{\mixedgroup}[1]{\id{mixed($#1$)}}
\newcommand{\cast}[2]{\id{CAST AS} \; #1 \; (#2)}
\newcommand{\bslsh}{\mathbin{\id{\\}}}
\newcommand{\bslshslsh}{\mathbin{\id{\\\\}}}
\newcommand{\fslsh}{\mathbin{\id{/}}}
\newcommand{\fslshslsh}{\mathbin{\id{//}}}
\newcommand{\bb}[1]{\mbox{#1}}
\newcommand{\bc}{\mathbin{\mathbf{::=}}}
\newcommand{\bm}{\mathbin{\mathbf\mid}}
\newcommand{\be}{\mathbin{=}}
\newcommand{\bd}{\mathbin{\buildrel {\rm \scriptscriptstyle def} \over \be}}
\newcommand{\category}[1]{\mbox{\bf #1}}

%GRAMMAR
\newlength{\ltext}
\newlength{\lmath}
\newlength{\cmath}
\newlength{\rmath}
\newlength{\rtext}

\settowidth{\ltext}{complex type name}
\settowidth{\lmath}{$xxx$}
\settowidth{\cmath}{$::=$}
\settowidth{\rmath}{\id{attributeGroup}}
\settowidth{\rtext}{repetition of $g$ between $m$ and $n$ times}

\newenvironment{grammar}{
  \[
  \begin{array}{l@{\quad}rcl@{\quad}l}
  \hspace{\ltext} & \hspace{\lmath} & \hspace{\cmath} & \hspace{\rmath} & \hspace{\rtext} \\
}{
  \end{array}\]
}

% Over-full v-boxes on even pages are due to the \v{c} in author's name
%\vfuzz2pt % Don't report over-full v-boxes if over-edge is small

% THEOREM Environments ---------------------------------------------------
% MATH -------------------------------------------------------------------
 \newcommand{\veps}{\varepsilon}
 \newcommand{\To}{\longrightarrow}
 \newcommand{\h}{\mathcal{H}}
 \newcommand{\s}{\mathcal{S}}
 \newcommand{\A}{\mathcal{A}}
 \newcommand{\J}{\mathcal{J}}
 \newcommand{\M}{\mathcal{M}}
 \newcommand{\W}{\mathcal{W}}
 \newcommand{\X}{\mathcal{X}}
 \newcommand{\BOP}{\mathbf{B}}
 \newcommand{\BH}{\mathbf{B}(\mathcal{H})}
 \newcommand{\KH}{\mathcal{K}(\mathcal{H})}
 \newcommand{\Real}{\mathbb{R}}
 \newcommand{\Complex}{\mathbb{C}}
 \newcommand{\Field}{\mathbb{F}}
 \newcommand{\RPlus}{\Real^{+}}
 \newcommand{\Polar}{\mathcal{P}_{\s}}
 \newcommand{\Poly}{\mathcal{P}(E)}
 \newcommand{\EssD}{\mathcal{D}}
 \newcommand{\Lom}{\mathcal{L}}
 \newcommand{\States}{\mathcal{T}}
 \newcommand{\abs}[1]{\left\vert#1\right\vert}
% \newcommand{\set}[1]{\left\{#1\right\}}
%\newcommand{\seq}[1]{\left<#1\right>}
 \newcommand{\norm}[1]{\left\Vert#1\right\Vert}
 \newcommand{\essnorm}[1]{\norm{#1}_{\ess}}

%%% NAMES
\newcommand{\Names}{{\mathcal N}}
\newcommand{\Channels}{{\sf X}}
\newcommand{\Variables}{{\mathcal V}}
\newcommand{\Enames}{{\mathcal E}}
\newcommand{\Nonterminals}{{\mathcal S}}
\newcommand{\Pnames}{{\mathcal P}}
\newcommand{\Dnames}{{\mathcal D}}
\newcommand{\Types}{{\mathcal T}}

\newcommand{\fcalc}{fusion calculus}
\newcommand{\xcalc}{${\mathfrak x}$-calculus}
\newcommand{\lcalc}{$\lambda$-calculus}
\newcommand{\pic}{$\pi$-calculus}
\newcommand{\rhoc}{$\rho$-calculus}
\newcommand{\rhol}{$\rho$-logic}
\newcommand{\hcalc}{highwire calculus}
\newcommand{\dcalc}{data calculus}
%XML should be all caps, not small caps. --cb
%\newcommand{\xml}{\textsc{xml}}
\newcommand{\xml}{XML} 

\newcommand{\papertitle}{A logic for a reflective higher-order calculus}
% use static date to preserve date of actual publication
\newcommand{\paperversion}{Draft Version 0.1 - December 19, 2004}

\newenvironment{toc}
{
\begin{list}{}{
   \setlength{\leftmargin}{0.4in}
   \setlength{\rightmargin}{0.6in}
   \setlength{\parskip}{0pt}
 } \item }
{\end{list}}

\newenvironment{narrow}
{
\begin{list}{}{
   \setlength{\leftmargin}{0.4in}
   \setlength{\rightmargin}{0.6in}
 } \item }
{\end{list}}

\def\lastname{Meredith and Radestock}
%%% ----------------------------------------------------------------------
\begin{document}

\begin{frontmatter}
	\title{A logic for a reflective higher-order calculus}

	\author{ L.G. Meredith \thanksref{lgmemail} }
	\address{ CTO, Djinnisys Corporation\\ 505 N72nd St, Seattle, WA 98103 }
	\author{ Matthias Radestock \thanksref {mremail} }	
	\address{ CTO, LShift, Ltd.\\ 6 Rufus St, London N1 6PE }
	\thanks[lgmemail]{ lgreg.meredith@gmail.com }
	\thanks[mremail]{ matthias@lshift.net }

%%% ----------------------------------------------------------------------

	\begin{abstract}
	The {\pic} is not a closed theory, but rather a theory dependent upon
	some theory of names. Taking an operational view, one may think of the
	{\pic} as a procedure that when handed a theory of names provides
	a theory of processes that communicate over those names. This openness
	of the theory has been exploited in {\pic} implementations, where
	ancillary mechanisms provide a means of interpreting of names, e.g. as
	tcp/ip ports. But, foundationally, one might ask if there is a closed
	theory of processes, i.e. one in which the theory of names arises from
	and is wholly determined by the theory of processes.
	
	Here we present such a theory in the form of an asynchronous
	message-passing calculus built on a notion of quoting. Names are
	quoted processes, and as such represent the code of a process, a
	reification of the syntactic structure of the process as an object for
	process manipulation. Name-passing in this setting becomes a way of
	passing the code of a process as a message. In the presence of a
	dequote operation, turning the code of a process into a running
	instance, this machinery yields higher-order characteristics without
	the introduction of process variables.

	We also present a logic.
	\end{abstract}

	\begin{keyword}
		concurrency, message-passing, process calculus, reflection, program logic
	\end{keyword}

\end{frontmatter}

\section{Introduction}

The {\pic} (\cite{milner91polyadicpi}) is not a closed theory, but
rather a theory dependent upon some theory of names. Taking an
operational view, one may think of the {\pic} as a procedure that when
handed a theory of names provides a theory of processes that
communicate over those names. This openness of the theory has been
exploited in {\pic} implementations, like the execution engine in
Microsoft's Biztalk \cite{biztalk}, where an ancillary binding
language providing a means of specifying a `theory' of names; e.g.,
names may be tcp/ip ports or urls or object references, etc. But,
foundationally, one might ask if there is a closed theory of
processes, i.e. one in which the theory of names arises from and is
wholly determined by the theory of processes. Behind this question
lurk a whole host of other exciting and potentially enlightening
questions regarding the role of names with structure in calculi of
interaction and the relationship between the structure of names and
the structure of processes.

Speaking provocatively, nowhere in the tools available to the computer
scientist is there a countably infinite set of \textit{atomic}
entities. All such sets, e.g. the natural numbers, the set of strings
of finite length on some alphabet, etc., are \textit{generated} from a
finite presentation, and as such the elements of these sets inherit
\textit{structure} from the generating procedure. As a theoretician
focusing on some aspects of the theory of processes built from such a
set, one may temporarily forget that structure, but it is there
nonetheless, and comes to the fore the moment one tries to build
\textit{executable} models of these calculi. 

To illustrate the point, when names have structure, name equality
becomes a computation. But, if our theory of interaction is to provide
a basis for a theory of computation, then certainly this computation
must be accounted for as well. Moreover, the fact that any realization
of these name-based, mobile calculi of interaction must come to grips
with names that have structure begs the question: would the
theoretical account of interaction be more effective, both as a theory
in its own right and as a guide for implementation, if it included an
account of the relationships between the structure of names and the
structure of processes?

\subsection{Overview and contributions} 

Here we present a theory of an asynchronous message-passing calculus
built on a notion of quoting. Names are quoted processes, and as such
represent the code of a process, a reification of the syntactic
structure of the process (up to some equivalence). Name-passing, then
becomes a way of passing the code of a process as a message. In the
presence of a dequote operation, turning the code of a process into a
running instance, this machinery yields higher-order characteristics
without the introduction of process variables.

As is standard with higher-order calculi, replication and/or recursion
is no longer required as a primitive operation. Somewhat more
interestingly, the introduction of a process constructor to
dynamically convert a process into its code is essential to obtain
computational completeness, and simultaneously supplants the function
of the $\nu$ operator. In fact, we give a compositional encoding of
the $\nu$ operator into the calculus, making essential use of dynamic
quote as well as dequote. Following the tradition started by Smith and
des Rivieres, \cite{lisp3} we dub this ability to turn running code
into data and back again, reflection; hence, the name
\textit{r}eflective,
\textit{h}igher-\textit{o}rder calculus, or rho-calculus, for short,
or $\rho$-calculus for even shorter.

Certainly, the paper presents a concrete calculus that may be used to
model a variety of computations and highlights a number of interesting
phenemona in those computations. We take the view, however, that the
main contribution is that the calculus provides an instrument to bring
to life a set of questions regarding the role of names in calculi of
interaction. These questions include the calculation of name equality
as a computation to be considered within the framework of interaction
and the roles of name equality in substitution versus
synchronization. These questions don't really come to life, though,
without the instrument in hand. So, we turn immediately to the
presentation of the calculus.

%%% ----------------------------------------------------------------------

\section{The calculus}

\subsubsection{Notation}

We let ${P, Q, R}$ range over processes and ${x, y, z}$ range over names.

\begin{grammar}
\mbox{\rhoc}		& {P, Q}		& \bc	& \pzero & \mbox{null process} \\
%				&					& \bm	& {x}\id{[}{y}\id{]} & \mbox{output} \\
				&					& \bm	& {x}\id{(}{y}\id{)} \concat {P} & \mbox{input} \\
				&					& \bm	& \lift{{x}}{{P}} & \mbox{lift} \\
				&					& \bm	& \dropn{{x}} & \mbox{ drop } \\
				&					& \bm	& {P} \juxtap {Q} & \mbox{ parallel } \\
				& {x, y}  		& \bc	& \quotep{{P}} & \mbox{ quote } \\
\end{grammar}

\subsubsection{Quote}

Working in a bottom-up fashion, we begin with names. The technical
detail corresponding to the {\pic}' parametricity in a theory of
\textit{names} shows up in standard presentations in the grammar
describing terms of the language: there is no production for names;
names are taken to be terminals in the grammar. Our first point of
departure from a more standard presentation of an asynchronous mobile
process calculus is here. The grammar for the terms of the language
will include a production for names in the grammar. A name is a
\textit{quoted} process, $\quotep{P}$.

\subsubsection{Parallel}

This constructor is the usual parallel composition, denoting
concurrent execution of the composed processes.

\subsubsection{Lift and drop}
Despite the fact that names are built from (the codes of) processes,
we still maintain a careful disinction in kind between process and
name; thus, name construction is not process construction. So, if one
wants to be able to generate a name from a given process, there must
be a process constructor for a term that creates a name from a
process. This is the motivation for the production $\lift{x}{P}$,
dubbed here the \emph{lift} operator. The intuitive meaning of this
term is that the process $P$ will be packaged up as its code,
$\quotep{P}$, and ultimately made available as an output at the port
$x$.

A more formal motivation for the introduction of this operator will
become clear in the sequel. But, it will suffice to say now that
$\quotep{P}$ is impervious to substitution. In the $\rho$-calculus,
substitution does not affect the process body between quote marks. On
the other hand, $\lift{x}{P}$ is susceptible to substitution and as
such constitutes a dynamic form of quoting because the process body
ultimately quoted will be different depending on the context in which
the $\lift{x}{P}$ expression occurs.

Of course, when a name is a quoted process, it is very handy to have a
way of evaluating such an entity. Thus, the $\dropn{x}$ operator,
pronounced \textit{drop} $x$, (eventually) extracts the process from a
name. We say `eventually' because this extraction only happens when a
quoted process is substituted into this expression. A consequence of
this behavior is that $\dropn{x}$ is inert except under and input
prefix. One way of saying this is that if you want to get something
done, sometimes you need to drop a name, but it should be the name of
an agent you know.

\begin{remark} %{No $\nu$ is good news}
The lift operator turns out to play a role analogous to $(\nu \;
x)P$. As mentioned in the introduction, it is essential to the
computational completeness of the calculus, playing a key role in the
implementation of replication. It also provides an essential
ingredient in the compositional encoding of the $\nu$ operator.
\end{remark}

\begin{remark} %{Dropping replication}
It is well-known that replication is not required in a higher-order
process algebra \cite{SangiorgiWalker}. While our algebra is
\textit{not} higher-order in the traditional sense (there are not
formal process variables of a different type from names) it has all
the features of a higher-order process algebra. Thus, it turns out
that there is no need for a term for recursion. To illustrate this we
present below an encoding of $!P$ in this calculus. Intuitively, this
will amount to receiving a quoted form of a process, evaluating it,
while making the quoted form available again. The reader familiar with
the $\lambda$-calculus will note the formal similarity between the
crucial term in the encoding and the paradoxical combinator
\cite{Barendregt84}.
\end{remark}

\subsubsection{Input and output}

The input constructor is standard for an asynchronous name-passing
calculus. Input blocks its continuation from execution until it
receives a communication. Lift is a form of output which -- because
the calculus is asynchronous -- is allowed no continuation. It also
affords a convenient syntactic sugar, which we define here.

\begin{eqnarray}
	{x}\id{[}{y}\id{]} \triangleq \lift{x}{\dropn{y}} \nonumber
\end{eqnarray}

\subsubsection{The null process}

As we will see below, the null process has a more distinguished role
in this calculus. It provides the sole atom out of which all other
processes (and the names they use) arise much in the same way that the
number $0$ is the sole number out of which the natural numbers are
constructed; or the empty set is the sole set out of which all sets
are built in $ZF$-set theory \cite{Krivine-TheCurryHowardCorre}; or
the empty game is the sole game out of which all games are built in
Conway's theory of games and numbers \cite{ONAG}. This analogy to
these other theories draws attention, in our opinion, to the
foundational issues raised in the introduction regarding the design of
calculi of interaction.

\subsection{The name game}

Before presenting some of the more standard features of a mobile
process calculus, the calculation of free names, structural
equivalence, etc., we wish to consider some examples of processes and
names. In particular, if processes are built out of names, and names
are built out of processes, is it ever possible to get off the ground?
Fortunately, there is one process the construction of which involves
no names, the null process, $0$. Since we have at least one process,
we can construct at least one name, namely $\quotep{0}$ \footnote{pun
gratefully accepted ;-)}. Armed with one name we can now construct at
least two new processes that are evidently syntactically different
from the $0$, these are $\quotep{0}\id{[}{\quotep{0}}\id{]}$ and
$\quotep{0}\id{(}{\quotep{0}}\id{)} \concat 0$. As we might expect,
the intuitive operational interpretation of these processes is also
distinct from the null process. Intuitively, we expect that the first
outputs the name $\quotep{0}$ on the channel $\quotep{0}$, much like
the ordinary {\pic} process $x\id{[}x\id{]}$ outputs the name $x$ on
the channel $x$, and the second inputs on the channel $\quotep{0}$,
much like the ordinary {\pic} process $x\id{(}x\id{)} \concat 0$
inputs on the channel $x$.

Of course, now that we have two more processes, we have two more
names, $\quotep{\quotep{0}\id{[}{\quotep{0}}\id{]}}$ and
$\quotep{\quotep{0}\id{(}{\quotep{0}}\id{)} \concat 0}$. Having three
names at our disposal we can construct a whole new supply of processes
that generate a fresh supply of names, and we're off and running. It
should be pointed out, though, that as soon as we had the null process
we also had $0 \juxtap 0$ and $0 \juxtap 0 \juxtap 0$ and
consequently, we had the names $\quotep{ 0 \juxtap 0 }$, and $\quotep{
0 \juxtap 0 \juxtap 0}$, and .... But, since we ultimately wish to
treat these compositions as merely other ways of writing the null
process and not distinct from it, should we admit the codes of these
processes as distinct from $\quotep{0}$?

This question leads to several intriguing and apparently fundamental
questions. Firstly, if names have structure, whether this derives from
the structure of processes or something else, what is a reasonable
notion of equality on names? How much computation, and of what kind,
should go into ascertaining equality on names? Additionally, what
roles should name equality play in a calculus of processes? In
constructing this calculus we became conscious that substitution and
synchronization identify two potentially very different roles for name
equality to play in name-passing calculi. That these are very
different roles is suggested by the fact that they may be carried out
by very different mechanisms in a workable and effective theory. We
offer one choice, but this is just one design choice among infinitely
many. Most likely, the primary value of this proposal is to raise the
question. Likewise, we offer a proposal regarding the calculation of
name equality that is just one of many and whose real purpose is to
make the question vivid. We wish to turn to the core mechanics of the
calculus with these questions in mind.

\subsection{Free and bound names}

The syntax has been chosen so that a binding occurrence of a name is
sandwiched between round braces, $\id{(} \cdot \id{)}$. Thus, the
calculation of the free names of a process, $P$, denoted
$\freenames{P}$ is given recursively by

%\begin{multicols}{2}
%{
	\begin{eqnarray}
		\freenames{\pzero} & = & \emptyset \nonumber\\
%		\freenames{x \id{[}y\id{]}} & = & \{ x, y \} \\
		\freenames{x \id{(}y\id{)} \concat P} & = & \{ x \} \cup (\freenames{P} \setminus \{ y \}) \nonumber\\
		\freenames{\lift{x}{P}} & = & \{ x \} \cup \freenames{P} \nonumber\\
		\freenames{\binpar{P}{Q}} & = & \freenames{P} \cup \freenames{Q} \nonumber\\
		\freenames{\dropn{x}} & = & \{ x \} \nonumber
	\end{eqnarray}
%}
%\end{multicols}

An occurrence of $x$ in a process $P$ is \textit{bound} if it is not
free. The set of names occurring in a process (bound or free) is
denoted by $\names{P}$.

\subsection{Structural congruence}

The {\em structural congruence} of processes, noted $\scong$, is the
least congruence, containing $\alpha$-equivalence, $\alphaeq$, that
satisfies the following laws:

\begin{eqnarray}
	{P} \juxtap \pzero	
		&  \scong \; {P} \; \scong & 
			\pzero \juxtap {P} \nonumber\\
	{P} \juxtap {Q}	
		& \scong & 
			{Q} \juxtap {P} \nonumber\\
	({P} \juxtap {Q}) \juxtap {R}
		& \scong & 
			{P} \juxtap ({Q} \juxtap {R}) \nonumber
\end{eqnarray}

\subsection{Name equivalence}

We now come to one of the first real subtleties of this calculus. Both
the calculation of the free names of a process and the determination
of structural congruence between processes critically depend on being
able to establish whether two names are equal. In the case of the
calculation of the free names of an input-guarded process, for
example, to remove the bound name we must determine whether it is in
the set of free names of the continuation. Likewise, structural
congruence includes $\alpha$-equivalence. But, establishing
$\alpha$-equivalence between the processes $x \id{(} z \id{)}\concat \lift{w}{y
\id{[}z\id{]}}$ and $x \id{(} v \id{)}\concat \lift{w}{y
\id{[}v\id{]}}$, for instance, requires calculating a substitution, e.g. $x \id{(} v \id{)}\concat \lift{w}{y
\id{[}v\id{]}} \id{\{}z / v \id{\}}$. But this calculation requires, in turn, being able to
determine whether two names, in this case the name in the object
position of the output, and the name being substituted for, are equal.

As will be seen, the equality on names involves structural equivalence
on processes, which in turn involves alpha equivalence, which involves
name equivalence. This is a subtle mutual recursion, but one that
turns out to be well-founded. Before presenting the technical details,
the reader may note that the grammar above enforces a strict
alternation between quotes and process constructors. Each question
about a process that involves a question about names may in turn
involve a question about processes, but the names in the processes the
next level down, as it were, are under fewer quotes. To put it another
way, each `recursive call' to name equivalence will involve one less
level of quoting, ultimately bottoming out in the quoted zero process.

Let us assume that we have an account of (syntactic) substitution and
$\alpha$-equivalence upon which we can rely to formulate a notion of
name equivalence, and then bootstrap our notions of substitution and
$\alpha$-equivalence from that. We take name equivalence, written
$\nameeq$, to be the smallest equivalence relation generated by the
following rules.


\infrule[Quote-drop]
{  }
{ \quotep{\dropn{x}} \nameeq x }

\infrule[Struct-equiv]
{ P \scong Q }
{ \quotep{P} \nameeq \quotep{Q} }

\subsection{Syntactic substitution}

Now we build the substitution used by $\alpha$-equivalence. We use
$\Proc$ for the set of processes, $\QProc$ for the set of names, and
$\id{\{}\vec{y} / \vec{x} \id{\}}$ to denote partial maps, $s : \QProc
\rightarrow \QProc$. A map, $s$ lifts, uniquely, to a map on process terms, $\widehat{s} :
\Proc \rightarrow \Proc$ by the following equations.

\begin{eqnarray}
(0) \psubstp{Q}{P}         & = &    0 \nonumber\\
(R \juxtap S) \psubstp{Q}{P}
		& = &    
		(R)\psubstp{Q}{P} \juxtap (S) \psubstp{Q}{P} \nonumber\\
(x \id{(} y\id{)}\concat R) \psubstp{Q}{P}    
		& = &    
		(x)\substp{Q}{P} \id{(}z\id{)}\concat( (R \psubstn{z}{y}) \psubstp{Q}{P} ) \nonumber\\
%(x\id{[}y\id{]}) \psubstp{Q}{P}
%		& = &    
%		(x)\substp{Q}{P} \id{[}(y)\substp{Q}{P} \id{]} \\
(\lift{x}{R}) \psubstp{Q}{P}  
 		& = &
 		\lift{(x)\substp{Q}{P}}{ R \psubstp{Q}{P} } \nonumber\\
(\dropn{x})  \psubstp{Q}{P}       
		& = & 
		\left\{ 
			\begin{array}{ccc} 
				\dropn{\quotep{Q}} & & x \nameeq \quotep{P} \\
                              	\dropn{x} & & otherwise \\
			\end{array}
		\right. \nonumber
\end{eqnarray}
 

where

\begin{eqnarray}
(x)\id{\{} \lpquote Q \rpquote / \lpquote P \rpquote \id{\}}            = 
		\left\{ 
			\begin{array}{ccc}
				\lpquote Q \rpquote & & x \nameeq \lpquote P \rpquote \\
                                x & & otherwise \\
			\end{array}
		\right. \nonumber
\end{eqnarray}

and $z$ is chosen distinct from $\quotep{P}$, $\quotep{Q}$, the free
names in $Q$, and all the names in $R$. Our $\alpha$-equivalence will
be built in the standard way from this substitution.

But, given these mutual recursions, the question is whether the calculation
of $\nameeq$ (respectively, $\scong$, $\alphaeq$) terminates. To
answer this question it suffices to formalize our intuitions regarding
level of quotes, or quote depth, $\#(x)$, of a name $x$ as
follows.

\begin{eqnarray}
	\#(\quotep{P}) & = & 1 + \#(P) \nonumber\\
    \#(P) & = & \left\{
			   \begin{array}{ccc}
				 max\{ \#(x) : x \in \names{P} \} & & \names{P} \neq \emptyset \\
				 0 & & otherwise \\
			   \end{array}
			\right. \nonumber
\end{eqnarray}

The grammar ensures that $\#(\quotep{P})$ is bounded. Then the
termination of $\nameeq$ (respectively, $\scong$, $\alphaeq$) is an
easy induction on quote depth.

\subsection{ Dynamic quote: an example }

Anticipating something of what's to come, consider applying the
substitution, $\widehat{\id{\{}u / z \id{\}}}$, to the following pair
of processes, $\lift{w}{y \id{[}z\id{]}}$ and $w\id{[} \lpquote
y\id{[}z\id{]} \rpquote \id{]}$.

\begin{eqnarray}
	\lift{w}{y \id{[}z\id{]}}\widehat{\id{\{}u / z \id{\}}}
		& = &
		\lift{w}{y \id{[}u\id{]}} \nonumber\\
	w\id{[} \lpquote y\id{[}z\id{]} \rpquote \id{]} \widehat{ \id{\{}u / z \id{\}} }
		& = &
		w\id{[} \lpquote y\id{[}z\id{]} \rpquote \id{]} \nonumber
\end{eqnarray}

Because the body of the process between quotes is impervious to
substitution, we get radically different answers. In fact, by
examining the first process in an input context, e.g. $x \id{(} z
\id{)}\concat \lift{w}{y
\id{[}z\id{]}}$, we see that the process under the lift operator may
be shaped by prefixed inputs binding a name inside it. In this sense,
the lift operator will be seen as a way to dynamically construct
processes before reifying them as names.

\subsection{Semantic substitution}

The substitution used in $\alpha$-equivalence is really only a device
to formally recognize that binding occurrences do not depend on the
specific names. It is not the engine of computation. The proposal here
is that while synchronization is the driver of that engine, the real
engine of computation is a semantic notion of substitution that
recognizes that a dropped name is a request to run a process. Which
process? Why the one whose code has been bound to the name being
dropped. Formally, this amounts to a notion of substitution that
differs from syntactic substitution in its application to a dropped
name.

\begin{eqnarray}
(\dropn{x})  \psubstp{Q}{P}       
		& = & 
		\left\{ 
			\begin{array}{ccc} 
				Q & & x \nameeq \quotep{P} \\
                              	\dropn{x} & & otherwise \\
			\end{array}
		\right. \nonumber
\end{eqnarray}

In the remainder of the paper we will refer to semantic and syntactic
substitutions simply as substitutions and rely on context to
distinguish which is meant. Similarly, we will abuse notation and
write $\substn{y}{x}$ for $\psubstn{y}{x}$.

Finally equipped with these standard features we can present the
dynamics of the calculus.

\subsection{Operational Semantics}
The reduction rules for {\rhoc}  are
% \infrule[Comm]
% { {{x}_0 \nameeq {x}_3} }
% {{{{x}_{0}\id{[}{x}_{2}\id{]}}} \juxtap {{x}_{3}\id{(}{x}_{1}\id{)} \concat {P}}
% \red {P}\id{\{}{x}_{2} \id{/} {x}_{1} \id{\}}}

%\infrule[Comm]
%{ {x}_{0} \nameeq {x}_{1} }
%{{{ x_{0} \id{[} z \id{]}}} \juxtap { x_{1} \id{(} y \id{)} \concat {P}}
%\red {P}\id{\{} z \id{/} y \id{\}}}

\infrule[Comm]
{ {x}_{0} \nameeq {x}_{1} }
{{\lift{x_{0}}{Q} \juxtap { x_{1} \id{(}{y} \id{)} \concat {P}}}
\red {P}\id{\{}\quotep{Q} \id{/} {y} \id{\}}}

In addition, we have the following context rules:

\infrule[Par]{{P} \red {P}'}{{{P} \juxtap {Q}} \red {{P}' \juxtap {Q}}}

\infrule[Equiv]{{{P} \scong {P}'} \andalso {{P}' \red {Q}'} \andalso {{Q}' \scong {Q}}}{{P} \red {Q}}

The context rules are entirely standard and we do not say much about
them, here. The communication rule does what was promised, namely make
it possible for agents to synchronize and communicate processes
packaged as names. For example, using the comm rule and name
equivalence we can now justify our syntactic sugar for output.

\begin{eqnarray}
	& {x}\id{[}{z}\id{]} \juxtap x \id{(}y\id{)} \concat P & \nonumber\\
	= 
	& \lift{x}{\dropn{z}} \juxtap x \id{(}y\id{)} \concat P & \nonumber\\
	\red 
	& P \substn{\quotep{\dropn{z}}}{y} & \nonumber\\
	\scong & P \substn{z}{y} & \nonumber
\end{eqnarray}

But, it also provides a scheme that identifies the role of name
equality in synchronization. There are other relationships between
names with structure that could also mediate
synchronization. Consider, for example, a calculus identical to the
one presented above, but with an alternative rule governing
communication.

\infrule[Comm-annihilation]
{ \forall R . [ P_{channel} \juxtap Q_{channel} \red^* R ] \Rightarrow R \red^* 0 }
{{\lift{\quotep{Q_{channel}}}{Q} \juxtap { {\quotep{P_{channel}}} \id{(}{y} \id{)} \concat {P}}}
\red {P}\id{\{}\quotep{Q} \id{/} {y} \id{\}}}

Intuitively, it says that the codes of a pair of processes,
$P_{channel}$, $Q_{channel}$, stand in channel/co-channel relation
just when the composition of the processes always eventually reduces
to $0$, that is, when the processes annihilate one another.  This rule
is well-founded, for observe that because $0 \scong 0 \juxtap 0$, $0
\juxtap 0 \red^* 0$. Thus, $\quotep{0}$ serves as its own
co-channel. Analogous to our generation of names from $0$, with one
such channel/co-channel pair, we can find many such pairs. What we
wish to point out about this rule is that we can see precisely an
account of the calculation of the channel/co-channel relationship as
deriving from the theory of interaction. We do not know if the
computation of name equality has a similar presentation, driving home
the potential difference of those two roles in calculi of interaction.

We mention, as a brief aside, that there is no reason why $0$ is
special in the scheme above. We posit a family of calculi, indexed by
a set of processes $\{ S_{\alpha} \}$, and differing only in their
communication rule each of which conforms to the scheme below.

\infrule[Comm-annihilation-S]
{ \forall R . [ P_{channel} \juxtap Q_{channel} \red^* R ] \Rightarrow R \red^* R' \scong S_{\alpha} }
{{\lift{\quotep{Q_{channel}}}{Q} \juxtap { {\quotep{P_{channel}}} \id{(}{y} \id{)} \concat {P}}}
\red {P}\id{\{}\quotep{Q} \id{/} {y} \id{\}}}

We explore this family of calculi in a forthcoming paper. For the rest
of this paper, however, we restrict our attention to the calculus with
the less exotic communication rule, using $\red$ for reduction
according to that system and $\wred$ for $\red^*$.

\section{Replication}

As mentioned before, it is known that replication (and hence
recursion) can be implemented in a higher-order process algebra
\cite{SangiorgiWalker}. As our first example of calculation with the
machinery thus far presented we give the construction explicitly in
the {\rhoc}.

\begin{eqnarray}
	D(x) & \triangleq & \prefix{x}{y}{(\binpar{\outputp{x}{y}}{\dropn{y}})} \nonumber\\
	\bangp{P}(x) & \triangleq & \binpar{\lift{x}{\binpar{D(x)}{P}}}{D(x)} \nonumber
\end{eqnarray}

\begin{eqnarray}
	\bangp{P}(x) & & \nonumber\\
	=
	& \lift{x}{(\prefix{x}{y}{(\outputp{x}{y} \juxtap \dropn{y})) \juxtap P}} 
	      \juxtap \prefix{x}{y}{(\outputp{x}{y} \juxtap \dropn{y})} & \nonumber\\
	\red
	& (\outputp{x}{y} \juxtap \dropn{y})\substn{\quotep{(\prefix{x}{y}{(\dropn{y} \juxtap \outputp{x}{y})) \juxtap P}}}{y} & \nonumber\\
	=
	& \outputp{x}{\quotep{(\prefix{x}{y}{(\outputp{x}{y} \juxtap \dropn{y})) \juxtap P}}}
	  \juxtap {(\prefix{x}{y}{(\outputp{x}{y} \juxtap \dropn{y})) \juxtap P}} & \nonumber\\
	\red
	& \ldots & \nonumber\\
	\red^*
	& P \juxtap P \juxtap \ldots & \nonumber
\end{eqnarray}

Of course, this encoding, as an implementation, runs away, unfolding
$\bangp{P}$ eagerly. A lazier and more implementable replication
operator, restricted to input-guarded processes, may be obtained as follows.

\begin{eqnarray}
\bangp{\prefix{u}{v}{P}} 
	\triangleq 
	\binpar{\lift{x}{\prefix{u}{v}{(\binpar{D(x)}{P})}}}{D(x)} \nonumber
\end{eqnarray}

It is worth noting that the lift operator is essential to get
computational completeness. A similar calculus equipped with only a
static quote enjoys a computational expressiveness at least equivalent
to context-free grammars, but short of context-sensitive. This fact is
established and exploited in a forthcoming paper on a type system for
the {\rhoc}.

\section{Bisimulation}

Having taken the notion of restriction out of the language, we
carefully place it back into the notion of observation, and hence into
the notion of program equality, i.e. bisimulation. That is, we
parameterize the notion of barbed bisimulation by a set of names over
which we are allowed to set the barbs. The motivation for this choice
is really comparison with other calculi. The set of names of the
{\rhoc} is \textit{global}. It is impossible, in the grammar of
processes, to guard terms from being placed into contexts that can
potentially observe communication. So, we provide a place for
reasoning about such limitations on the scope of observation in the
theory of bisimulation.

\begin{definition}
An \emph{observation relation}, $\downarrow_{\mathcal N}$, over a set
of names, $\mathcal N$, is the smallest relation satisfying the rules
below.

\infrule[Out-barb]{y \in {\mathcal N}, \; x \nameeq y}
		  {\outputp{x}{v} \downarrow_{\mathcal N} x}
\infrule[Par-barb]{\mbox{$P\downarrow_{\mathcal N} x$ or $Q\downarrow_{\mathcal N} x$}}
		  {\binpar{P}{Q} \downarrow_{\mathcal N} x}

We write $P \Downarrow_{\mathcal N} x$ if there is $Q$ such that 
$P \wred Q$ and $Q \downarrow_{\mathcal N} x$.
\end{definition}

Notice that $\prefix{x}{y}{P}$ has no barb.  Indeed, in {\rhoc} as well
as other asynchronous calculi, an observer has no direct means to
detect if a message sent has been received or not.

\begin{definition}
%\label{def.bbisim}
An  ${\mathcal N}$-\emph{barbed bisimulation} over a set of names, ${\mathcal N}$, is a symmetric binary relation 
${\mathcal S}_{\mathcal N}$ between agents such that $P\rel{S}_{\mathcal N}Q$ implies:
\begin{enumerate}
\item If $P \red P'$ then $Q \wred Q'$ and $P'\rel{S}_{\mathcal N} Q'$.
\item If $P\downarrow_{\mathcal N} x$, then $Q\Downarrow_{\mathcal N} x$.
\end{enumerate}
$P$ is ${\mathcal N}$-barbed bisimilar to $Q$, written
$P \wbbisim_{\mathcal N} Q$, if $P \rel{S}_{\mathcal N} Q$ for some ${\mathcal N}$-barbed bisimulation ${\mathcal S}_{\mathcal N}$.
\end{definition}

\section{Interpreting \pic}

Here we provide an encoding of the pure asynchronous {\pic} into the
{\rhoc}. Since all names are global in the {\rhoc}, we encounter a
small complication in the treatment of free names at the outset. There
are several ways to handle this. One is to insist that the translation
be handed a closed program (one in which all names are bound either by
input or by restriction). This alternative feels inelegant. Another is
to provide an environment, $r : \mathcal{N}_{\pi} \rightarrow \QProc$,
for mapping the free names in a {\pic} process into names in the
{\rhoc}. Maintaining the updates to the environment, however, obscures
the simplicity of the translation. We adopt a third alternative.

To hammer home the point that the {\pic} is parameterized in a theory
of names, we build a {\pic} in which the names are the names of
{\rhoc}. This is no different than building a {\pic} using the natural
numbers, or the set of URLs as the set of names. Just as there is no
connection between the structure of these kinds of names and the
structure of processes in the {\pic}, there is no connection between
the processes quoted in the names used by the theory and the processes
generated by the theory, and we exploit this fact.

\subsection{\pic}

More formally,

\begin{grammar}
\mbox{\pic}		& {P, Q}			& \bc	& \pzero \\
				&					& \bm	& {x}\id{[}{y}\id{]} \\
				&					& \bm	& {x}\id{(}{y}\id{)} \concat {P} \\
				&					& \bm	& \newp{{x}}{{P}} \\
				&					& \bm	& {P} \juxtap {Q} \\
				&					& \bm	& \bangp{{P}} \\
				& {x, y}    		& \bc	& x, y \in \QProc 
\end{grammar} 

Note well: names are quoted \emph{$\rho$-calculus} processes.

\subsection{Structural congruence}

\begin{definition}
The {\em structural congruence}, $\equiv$, between processes is 
the least congruence closed with respect to
alpha-renaming, satisfying the abelian monoid laws for 
parallel (associativity, commutativity and $\pzero$ 
as identity), and the following axioms:
\begin{enumerate}
\item the scope laws:
\begin{eqnarray}
 \newp{x}\pzero  	& \equiv & \pzero, \nonumber\\
 \newp{x}{\newp{x}{P}} 	& \equiv & \newp{x}{P}, \nonumber\\
 \newp{x}{\newp{y}{P}} 	& \equiv & \newp{y}{\newp{x}{P}}, \nonumber\\
 \binpar{P}{\newp{x}{Q}} & \equiv & \newp{x}{\binpar{P}{Q}}, \; \mbox{\textit{if} }x \not\in \freenames{P} \nonumber
\end{eqnarray}
\item
the recursion law:
\begin{eqnarray}
 \bangp{P} \equiv \binpar{P}{\bangp{P}} \nonumber
\end{eqnarray}
\item
the name equivalence law:
\begin{eqnarray}
 P \equiv P \substn{x}{y}, \; \mbox{\textit{if} }x \nameeq y \nonumber
\end{eqnarray}
\end{enumerate}
\end{definition}

\subsection{Operational semantics} The operational semantics is standard.

\infrule[Comm]
{ }
{{{{x}\id{[}{z}\id{]}}} \juxtap {{x}\id{(}{y}\id{)} \concat {P}}
\red {P}\id{\{}{z} \id{/} {y} \id{\}}}

In addition, we have the following context rules:

\infrule[Par]{{P} \red {P}'}{{{P} \juxtap {Q}} \red {{P}' \juxtap {Q}}}

\infrule[New]{{P} \red {P}'}{{\newp{{x}}{{P}}} \red {\newp{{x}}{{P}'}}}

\infrule[Equiv]{{{P} \scong {P}'} \andalso {{P}' \red {Q}'} \andalso {{Q}' \scong {Q}}}{{P} \red {Q}}

Again, we write $\wred$ for $\red^*$, and rely on context to
distinguish when $\red$ means reduction in the {\pic} and when it
means reduction in the {\rhoc}. The set of {\pic} processes will be
denoted by $\Proc_{\pi}$.

\subsection{The translation}

The translation will be given by a function, $\meaningof{-}( -, - ) :
\Proc_{\pi} \times \QProc \times \QProc \red \Proc$. The guiding
intuition is that we construct alongside the process a distributed memory
allocator, the process' access to which is mediated through the second argument
to the function. The first argument determines the shape of the memory
for the given allocator.

Given a process, $P$, we pick $n$ and $p$ such that $n \neq p$ and
distinct from the free names of $P$. For example, $n = \quotep{\Pi_{m
\in \freenames{P}}\outputp{m}{\quotep{\pzero}}}$ and $p =
\quotep{\Pi_{m \in
\freenames{P}}\prefix{m}{\quotep{\pzero}}{\pzero}}$. Then

\begin{eqnarray}
	\meaningof{P} = \meaningof{P}_{2nd}( n, p ) \nonumber
\end{eqnarray}

where

\begin{eqnarray}
   	\meaningof{\pzero}_{2nd} (  n, p )
   		& = &
   		 \pzero \nonumber\\
   	\meaningof{x \id{[} y \id{]}}_{2nd} (  n, p ) 
  		& = & 
  		x \id{[} y \id{]} \nonumber\\
   	\meaningof{x \id{(} y \id{)} \concat P}_{2nd} (  n, p ) 
   		& = & 
 		x \id{(} y \id{)} \concat \meaningof{P}_{2nd} (  n, p ) \nonumber\\
   	\meaningof{P \juxtap Q}_{2nd} (  n, p ) 
   		& = & 
 		\meaningof{P}_{2nd} ( n^{l}, p^{l} )
   			 \juxtap \meaningof{Q}_{2nd} ( n^{r}, p^{r} ) \nonumber\\
   	\meaningof{\id{!} P}_{2nd} (  n, p )
   		& = & \binpar{\lift{x}{\meaningof{P}_{3rd}( n^{r}, p^{r} )}}
		             {\binpar{D(x)}{\binpar{\outputp{n^{r}}{n^{l}}}{\outputp{p^{r}}{p^{l}}}}} \nonumber\\
   	\meaningof{\id{(}\nu \; x \id{)} P}_{2nd} (  n, p ) 
   		& = & 
 		\prefix{p}{x}{\binpar{\meaningof{P}_{2nd} ( n^{l}, p^{l} )}{\outputp{p}{n}}} \nonumber
\end{eqnarray}

and

\begin{eqnarray}
	x^{l} & \triangleq & \quotep{\outputp{x}{x}} \nonumber\\
	x^{r} & \triangleq & \quotep{\prefix{x}{x}{\pzero}} \nonumber\\
	\meaningof{P}_{3rd}( n'', p'' ) 
		& \triangleq & 
			\prefix{n''}{n}{\prefix{p''}{p}{(\binpar{\meaningof{P}_{2nd}(  n, p )}
							        {(\binpar{D(x)}{\binpar{\outputp{n''}{n^{l}}}{\outputp{p''}{p^{l}}}})})}} \nonumber
\end{eqnarray}
	
\begin{remark}
	It is also noteworthy that the translation is dependent on how
	the parallel compositions in a process are
	associated. Different associations will result in different
	bindings for $\nu$-ed names. This will not result in different
	behavior, however, as the bindings will be consistent
	throughout the translation of the process.
\end{remark}

\begin{theorem}[Correctness]	
	$P \wbbisim_{\pi} Q \iff \ldb P \rdb \wbbisim_{r(\texttt{FN}(P))} \ldb Q \rdb$.
\end{theorem}

\emph{Proof sketch}: An easy structural induction.

One key point in the proof is that there are contexts in the {\rhoc}
that will distinguish the translations. But, these are contexts that
can see the fresh names, $n$, and the communication channel, $p$, for
the `memory allocator'. These contexts do not correspond to any
observation that can be made in the {\pic} and so we exclude them in
the {\rhoc} side of our translation by our choice of ${\mathcal N}$
for the bisimulation. This is one of the technical motivations behind
our introduction of a less standard bisimulation.

\begin{example}
	In a similar vein consider, for an appropriately chosen $p$ and $n$ we have
	\begin{eqnarray}
		\meaningof{\newp{v}{\newp{v}{\outputp{u}{v}}}} = 
			\binpar{\prefix{p}{v}{(\binpar{(\prefix{\quotep{\outputp{p}{p}}}{v}{\outputp{u}{v}})}
					              {(\outputp{\quotep{\outputp{p}{p}}}{\quotep{\outputp{n}{n}}})})}}
			       {\outputp{p}{n}} \nonumber
	\end{eqnarray}
	and
	\begin{eqnarray}
		\meaningof{\newp{v}{\outputp{u}{v}}} = 
				\binpar	{\prefix{p}{v}{\outputp{u}{v}}}
			       		{\outputp{p}{n}} \nonumber
	\end{eqnarray}

	Both programs will ultimately result in an output of a single
	fresh name on the channel $u$. But, the former program will
	consume more resources. Two names will be allocated; two memory
	requests will be fulfilled. The {\rhoc} can see this, while the
	{\pic} cannot. In particular, the {\pic} requires that
	$\newp{x}{\newp{x}{P}} \equiv \newp{x}{P}$.

	Implementations of the {\pic}, however, having the property that
	${\newp{x}{P}}$ involves the allocation of memory for the
	structure representing the channel $x$ come to grips with the
	implications this requirement has regarding memory management. If
	memory is allocated upon encountering the $\nu$-scope, there are
	situations where the left-hand side of the equation above will
	fail while the right-hand will succeed. Remaining faithful to the
	equation above requires that such implementations are
	\textit{lazy} in their interpretation of ${\newp{x}{P}}$, only
	allocating the memory for the fresh channel at the first moment
	when that channel is used.

	Having a detailed account of the structure of names elucidates
	this issue at the theoretical level and may make way to offer
	guidance to implementations.
\end{example}

\subsection{Higher-order {\pic}}

As noted above, the lift and drop operators of the {\rhoc} effectively
give it features of a higher-order calculus \cite{hop},
\cite{facile}. The translation of the higher-order {\pic} is quite
similar to the translation for {\pic}. Of course, the higher-order
{\pic} has application and one may wonder how this is
accomplished. This is where the susceptibility of lift to
substitutions comes in handy. For example, to translate the parallel
composition of a process that sends an abstraction, $\absp{v}{P}$, to
a process that receives it and applies it to the values, $v$ we
calculate

\begin{eqnarray}
	\meaningof{\binpar{\outputp{x}{\absp{v}{P}}}{(\prefix{x}{Y}{\applyp{Y}{v}})}}(z)
		= 
	\binpar{(\prefix{z}{v}{\lift{x}{\meaningof{P}(z')}})}
	       {(\binpar{\prefix{x}{y}{\dropn{y}}}{\outputp{z}{\meaningof{v}(z'')}})} \nonumber
\end{eqnarray}

where the translation is parameterized in a channel, $z$, for sending
values, and $z'$ and $z''$ are constructed from $z$ in some manner
analogous to what is done with $n$ and $p$ above.

More generally, one may seek to understand the trade-offs between a
presentation of higher-order capability in the higher-order {\pic} and
the {\rhoc}. A detailed study is a subject worthy of an entire paper,
but at a high level of description one may note that the same argument
levied with the ordinary {\pic} applies here: the higher-order {\pic}
does not \emph{offer} a theory of names, but rather \emph{depends}
upon one being provided. An investigator interested in the
higher-order {\pic} as an executable language must still address
computation on names, such as calculating name equality in
substitution or synchronization, outside of the framework of the
theory. Additionally, the higher-order {\pic} has a larger inventory
of moving parts: process variables, for sending and receiving
processes, as well as names. On both counts the {\rhoc} is more
minimalist, needing neither a theory of names, nor the machinery of
process variables. On the other hand, minimalism does not always align
with ease of use. Experience shows that when writing specifications in
the {\rhoc} of any reasonable size one quickly adopts conventions that
make the calculus resemble a more traditional higher-order calculus.

\section{Logic}

In this section we present a logic for the calculus.

\begin{grammar}
\mbox{reflective logic}	& {\phi, \psi}	& \bc	& \ptrue & \mbox{verity} \\
				&					& \bm	& \pzero & \mbox{nullity} \\
				&					& \bm	& \neg \phi & \mbox{negation} \\
				&					& \bm	& \phi \& \psi & \mbox{conjunction} \\
				&					& \bm	& \phi \juxtap \psi & \mbox{simultaneity} \\
				&					& \bm	& \pdropf{b} & \mbox{descent} \\
				&					& \bm	& \plift{a}{\phi} & \mbox{elevation} \\
				&					& \bm	& \pprefix{a}{b}{\phi} & \mbox{activity} \\
				& {a}  		& \bc	& \quotep{\phi} & \mbox{indication} \\
				&					& \bm	& b & \mbox{...} \\
				& {b}  		& \bc	& \quotep{P} & \mbox{nomination} \\
\end{grammar}

We denote by $\Formula$ the set of formulae generated by this grammar.

The satisfaction relation, $\models \subseteq \Proc \times \Formula$, between processes and formulae is given by

\begin{eqnarray}
	P & \models & \ptrue \nonumber \\ 
    P & \models & \pzero \riff P \scong \pzero \nonumber \\ 
    P & \models & \neg \phi \riff P \not\models \phi \nonumber\\
    P & \models & \phi \& \psi \riff P \models \phi \; \& P \; \models \psi \nonumber\\
    P & \models & \binpar{\phi}{\psi} \riff \exists P_0, P_1.P \scong \binpar{P_0}{P_1}
      \; \& \; P_0 \models \phi \; \& \;  P_1 \models \psi \nonumber\\
    P & \models & \pdropf{b} \riff \exists Q, P'.P \scong \binpar{Q}{\dropn{x}}
	    \; \& \; x \models b \nonumber\\	
	P & \models & \plift{a}{\phi} \riff \exists Q, P'.P \scong \binpar{Q}{\lift{x}{P'}} 
		\; \& \; x \models a 
		\; \& \; P' \models \phi \nonumber\\
	P & \models & \pprefix{a}{b}{\phi} \riff \exists Q, P'.P \scong \binpar{Q}{\prefix{x}{y}{P'}} \nonumber\\
	  & & x \models a \nonumber\\
	  & & \forall c . \exists z . {P'}\substn{z}{y} \models {\phi}\substn{c}{b} \nonumber\\
    \quotep{P} & \models & \quotep{\phi} \riff P \models \phi \nonumber\\
    \quotep{P} & \models & \quotep{Q} \riff \quotep{P} \nameeq \quotep{Q} \nonumber
\end{eqnarray}

\begin{theorem}[Equivalence]
	$P \wbbisim Q \riff \forall \phi . P \models \phi \riff Q \models \phi .$
\end{theorem}

\emph{Proof sketch}: tbd.

\subsubsection{Syntactic sugar}

\begin{eqnarray}
	\phi \Rightarrow \psi & \triangleq & \neg ( \phi \& \neg \psi ) \nonumber\\
	\phi \vee \psi & \triangleq & \neg ( \neg \phi \& \neg \psi ) \nonumber
\end{eqnarray}

\subsection{Examples}

\subsubsection{Regenerating processes and persistent states}

...

\subsubsection{Namespaces and namespace exploration}

...

\section{Conclusions and future work}

We studied an asynchronous message-passing calculus built out of a
notion of quote. We showed that the calculus provides a workable,
effective theory of computation capable of encoding the {\pic} with a
compositional account of the $\nu$-operator, as well as the
higher-order {\pic}. These encodings bring to light interesting
computational phenomena that implementations of the {\pic} have had to
face. Additionally, the development of the calculus highlights several
intriguiging aspects of the relationships between the structure of
processes and the structure of names.

We note that this work is situated in the larger context of a growing
investigation into naming and computation. Milner's studies of action
calculi led not only to reflexive action calculi
\cite{DBLP:conf/tacs/Milner97}, but to Power's and Hermida's work on
name-free accounts of action calculi
\cite{DBLP:conf/concur/HermidaP95} as well as Pavlovic's \cite{PavlovicD:CLNA}. 
Somewhat farther afield, but still
related, is Gabbay's theory of freshness \cite{GabbayMJ:picfm}. Very
close to the mark, Carbone and Maffeis observe a tower of
expressiveness resulting from adding very simple structure to names
\cite{polysync}. In some sense, this may be viewed as approaching the
phenomena of structured names `from below'. By making names be
processes, this work may be seen as approaching the same phenomena
`from above'. But, both investigations are really the beginnings of a
much longer and deeper investigation of the relationship between
process structure and name structure.

Beyond foundational questions concerning the theory of interaction,
such an investigation may be highly warranted in light of the recent
connection between concurrency theory and biology. In particular,
despite the interesting results achieved by researchers in this field,
there is a fundamental difference between the kind of synchronization
observed in the {\pic} and the kind of synchronization observed
between molecules at the bio-molecular level. The difference is that
interactions in the latter case occur at sites with extension and
behavior of their own \cite{Fontana}. An account of these kinds of
phenomena may be revealed in a detailed study of the relationship
between the structure of names and the structure of processes.

\paragraph{Acknowledgments.}
The authors wish to thank Robin Milner for his thoughtful and
stimulating remarks regarding earlier work in this direction, and
Cosimo Laneve for urging us to consider a version of the calculus
without heating rules.

% ------------------------------------------------------------------------
%GATHER{Xbib.bib}   % For Gather Purpose Only
%GATHER{Paper.bbl}  % For Gather Purpose Only
\bibliographystyle{plain}
\bibliography{ex_nihilo_logic}

% ------------------------------------------------------------------------

% ------------------------------------------------------------------------

\end{document}
% ------------------------------------------------------------------------
