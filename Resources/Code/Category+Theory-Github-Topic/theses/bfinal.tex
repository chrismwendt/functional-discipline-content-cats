\oldchapter{Lekensamenvatting}
    
De volgende twee technische vraagstukken
    in de wiskundige theorie van quantum computers
    liggen ten grondslag aan dit proefschrift.
\begin{enumerate}
\item 
    Breidt de stelling van Stinespring uit naar
        alle von Neumann algebra's?
\item
    Is de categorie van von Neumann algebra's te axiomatiseren?
\end{enumerate}
Het eerste vraagstuk wordt beantwoord: ja, de stelling breidt uit.
Het tweede vraagstuk blijft helaas voor een groot deel onbeantwoord.
De belangrijkste bijdragen van dit proefschrift aan de wetenschap
    zijn niet per se deze directe resultaten,
    maar de bijvangst die het onderzoek erna heeft opgeleverd:
    er worden verscheidene nieuwe stellingen bewezen
        en begrippen ingevoerd
    die toepassing hebben buiten deze vraagstukken
    (zoals bijvoorbeeld een uitbreiding van de stelling
        van Kaplansky en het begrip~$\diamond$-geadjungeerdheid).

Maar waar gaan deze twee originele vraagstukken eigenlijk over?
Wat is die stelling van Stinespring?
Wat zijn von Neumann algebra's?
En wat is een quantum computer \"uberhaupt?

\oldsection*{Quantum computers}
Een quantum computer gebruikt
       bepaalde quantum mechanische eigenaardigheden
    om sommige berekeningen veel effici\"enter uit
    te voeren dan een traditionele computer.
Een zorgelijk voorbeeld is
    dat quantum computer
    erg goed is in het ontbinden van getallen in priemfactoren.
Het grootste deel van de moderne cryptografie
        die ons beveiligt is gestoeld op de aanname
        dat het vinden van zulke ontbindingen lastig is
        en deze wordt dan ook makkelijk door
        een voldoende grote quantum computer gebroken.
Gelukkig zijn de huidige quantum computers nog erg klein
    en hebben wij de tijd om onze cryptografie aan te passen.
Aan de positieve kant zijn
    quantum computer erg geschikt voor het simuleren
    van het gedrag van moleculen,
    waaronder het lokale effect van potenti\"ele medicijnen.

In functie is een quantum computer te vergelijken met
        een grafische kaart van een computer:
    voor de meeste berekeningen is een normale CPU het best,
            maar voor sommige berekeningen
        is een grafische kaart veel effici\"enter.
Dit komt doordat een normale CPU gemaakt is om zo snel mogelijk
    losse rekenstappen op afzonderlijke getallen
    achter elkaar uit te voeren,
    terwijl een grafische kaart rekenstappen
    uitvoert op miljoenen getallen tegelijkertijd.
Niet elke berekening heeft baat bij dat grote parallelisme.
Voor diegene die dat wel hebben (zoals het
    weergeven van een 3D-wereld in een computerspel)
    is het lastig om het originele algoritme aan te passen
    om optimaal gebruik te maken van de grafische kaart.
    Programmeren voor een grafische kaart vraagt om een
    heel andere \emph{mindset}.
Ook de theoretische analyse van algoritmes voor grafische kaarten
    is van een andere aard dan die voor een traditionele computer.

De situatie bij een quantum computer is vergelijkbaar, maar extremer:
    de effici\"entiewinst tussen een quantum computer en PC
    is groter dan die tussen een CPU en grafische kaart.
Helaas is het vinden en analyseren van algoritmes voor quantum computers
    ook vele malen ingewikkelder.
Er is ook geen methode of zelfs vuistregel bekend
    om te bepalen of een berekening effici\"enter uit te voeren
    is op een quantum computer of niet.

\oldsection*{Von Neumann algebra's}

Het gedrag van een algoritme voor een traditionele computer,
    zoals diegene die het kwadraat van een telgetal berekent,
    wordt vaak beschreven met wiskundige functie,
    zoals in dit geval de functie~$f\colon \N \to \N$
        met~$f(n) = n^2$.
De notatie~$f\colon \N \to \N$
    betekent dat de functie~$f$ als invoer
    een telgetal~$\N \equiv \{0, 1, 2, \ldots \}$
    neemt en ook een telgetal als uitvoer geeft.
Een echt programma draait op een computer
    dat maar een beperkte hoeveelheid geheugen heeft:
        er is dus een grootste telgetal dat in de computer past.
    Dat we het algoritme beschrijven
    met willekeurig grote telgetallen is een idealisering van de situatie.
Het zou echter onhandig zijn om alle generiek toepasbare algoritme
    te beschrijven als wiskundige functies tussen
    eindige verzamelingen~(zoals~$\{0, 1, 2, 3\}$)
    in plaats van tussen oneindige verzamelingen zoals~$\N$.

Dit is helaas precies de situatie voor de meeste beschrijvingen van
    algoritmes voor quantum computers.
Het probleem is dat het analoog van~$\N$
    en de wiskundige functies daartussen
    vele malen complexer van aard zijn
    dan de wiskundige functies die het gedrag
    beschrijven voor het eindige geval.
Ook al dit eindige geval op zichzelf is
    stukken ingewikkelder dan de simpele functies
    tussen eindige verzamelingen voor de normale algoritmes.

\emph{Von Neumann algebra's} zijn de wiskundige structuren
    die gebruikt worden om oneindige quantumdatatypes
    te beschrijven zoals het analoog van~$\N$.
Zogenoemde~\emph{ncp-functies} tussen von Neumann algebras
    zijn het analoog van de normale wiskundige functies
    tussen verzamelingen.
Het zijn deze von Neumann algebra's met bijbehorende
    ncp-functies, die in dit proefschrift onderzocht worden
    om ze beter te begrijpen en eenvoudiger toe te kunnen passen.

\oldsection*{De stelling van Stinespring}
Het is lastig om grip te krijgen
    op een willekeurige ncp-functie
    tussen von Neumann algebra's.
De stelling van Stinespring helpt hierbij:
    deze zegt dat elke ncp-functie~$f\colon \scrA \to \scrB_I$
    tussen von Neumann algebra's~$\scrA$ en~$\scrB_I$
        (waar~$\scrB_I$ van een speciale soort is)
        in feite de samenstelling is
        van twee veel simpelere ncp-functies:
        eerst~$f_1\colon \scrA \to \scrP$
            en dan~$f_2\colon \scrP \to \scrB_I$.
    Deze bijzondere opsplitsing wordt
        ook wel een \emph{dilatie} genoemd
            en wordt gebruikt als cruciale stap in het
    bewijs van vele stellingen, niet alleen over ncp-functies zelf,
    maar ook over bijvoorneeld quantum protocollen en quantum informatie.
Het probleem is dat de stelling van Stinespring alleen toepasbaar
    is als~$\scrB_I$ van de speciale soort \emph{type I} is.
Daarmee zijn de stellingen die ermee worden bewezen vaak ook maar toepasbaar
    op zulke speciale von Neumann algebra's.
In het eerste hoofdstuk van het proefschrift
    wordt een generalisatie van de stelling van Stinespring
    bewezen, die toepasbaar is op alle ncp-functies.

\oldsection*{Axiomatisatie}
Waarom zijn juist von Neumann algebra's en ncp-functies
    geschikt voor oneindige quantum datatypes?
In feite is het een wiskundige gok:
    von Neumann algebra's en ncp-functies
    lijken de eenvoudigste structuren die
    de voorbeelden die we kennen goed beschrijven.
Eenvoud hier is relatief: de ingewikkelde definitie
    van een von Neumann algebra
    staat ver van de intu\"itie van een
    informaticus en natuurkundige.
Het valt ook te betwijfelen of elke von Neumann
    algebra een realistisch quantum mechanisch systeem beschrijft,
    laat staan een quantum datatype.

Een mogelijke manier om deze kwestie te beslechten is de volgende:
    wij zoeken een lijst van kenmerkende
    (en voor onze toepassing relevante) eigenschappen waaraan
        von Neumann algebra's samen met ncp-functies aan voldoen.
Als er bewezen kan worden dat alleen von Neumann algebra's
    met ncp-functies aan deze lijst van eigenschappen kan voldoen
    (en geen ander soort wiskundige structuur)
    dan hebben wij met deze lijst van eigenschappen
        (nu verheven tot axioma's)
    een nieuw inzicht in de aard van von Neumann algebra's.

Een dergelijke aanpak is eerder al succesvol gebruikt
    om de regels van quantum theorie voor eindige systemen beter te begrijpen.
Een voorbeeld hiervan zijn de axioma's van Chiribella et al,
    die quantum theorie karakteriseren door
    de wijze waarop informatie verwerkt wordt.

In het tweede hoofdstuk van dit proefschrift
    wordt gezocht naar een dergelijke axiomatisatie.
        Een van de kernaxioma's is het bestaan
            van een soort \emph{omkering}, een zogenoemde~\emph{dagger},
            op de speciale ncp-functies die aan de rechterkant
            voorkomen van een dilatie,
            wat de twee hoofdstuken met elkaar verbindt.
Helaas wordt een volledige axiomatisatie niet bereikt.
Het verhaal eindigt hier niet:
    mijn college van de Wetering
        heeft recent bewezen dat met een paar toevoegingen,
        de axioma's uit dit proefschrift
        zogeheten Euclidische Jordan algebra's (EJA's) karakteriseren.
Hiermee is de kous nog niet af:
    EJA's zijn niet geschikt voor oneindige quantum datatypes.
Dit resultaat suggereert wel de mogelijkheid
    dat JBW-algebra's,
        welke een kruising zijn van von Neumann en Jordan algebra's,
    uiteindelijk de juiste algebra's zijn voor onze toepassing.

Hoewel ons tweede vraagstuk niet beantwoord is, heeft de zoektocht
    naar geschikte axioma's een aantal nieuwe begrippen aan het licht
    gebracht (zoals~$\diamond$-geadjun\-geerdheid, hoeken en filters), die een
    nieuw inzicht verschaffen in von Neumann algebra's
        en daarmee quantum computers in het algemeen.

\oldchapter{About the author}

Bas Westerbaan, born in 1988,
    enrolled to the Radboud Universiteit
     as a physics student in 2007.
He received his bachelor's degree in
    mathematics with a thesis on computability theory
    supervised by~dr.~Wim Veldman.
He continued his study of logic and foundational computer science
    during his master's eduction and developed a keen interest for,
     what he thought to be, an unrelated mathematical field:
        functional analysis.
These interests however, are conveniently combined in the present thesis
    which continues the research of his master's thesis,
    which was supervised by prof.~Bart Jacobs as well.
During his undergraduate studies,
    Bas also served as secretary of the board of the 150 member student club
        Karpe Noktem and he was elected to
        the student board of the faculty of science.
His master's degree in mathematics was awarded cum laude in 2013.

While perfoming the research as a \emph{promovendus}
    at the Digital Security group that led to this thesis,
    Bas coauthered several security audits
        including those of the \emph{DigiD-} and \emph{BerichtenBox}-app;
        the Dutch digital identity and mail system. 
In the year between the submission and defense
    of his thesis, Bas was employed as a post-doctoral researcher
    where he assisted with the development of a cryptographic system
    for polymorphic pseudonymisation and encryption.
During that same year he also taught a quarter-year course on
    computer networking
        and was kindly invited by dr.~Heunen
        to speak about his research at the University of Edinburgh.

% vim: ft=tex.latex
