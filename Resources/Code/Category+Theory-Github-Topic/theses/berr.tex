\begin{erratum}{physics-stinespring}%
The second part of the exercise starts with
    ``Conclude that any quantum channel~$\Phi\colon \scrB(\scrK) \to \scrB(\scrH)$ \ldots''.
This should have read
    ``Conclude that any quantum channel~$\Phi$ from~$\scrH$ to itself \ldots''
\end{erratum}
\begin{erratum}{ess-uniq-pur}%
The third-to-last sentence reads
``Derive from the latter
that for each~$y \in \scrK'$ and rank-one projector~$e \in \scrH$,
there is a~$y' \in \scrK'$
with~$U_0 (e \otimes y) = e \otimes y'$.''
This should have been
``Derive from the latter
that for each~$y \in \scrK'$ and unit vector~$e \in \scrH$,
there is a~$y' \in \scrK'$
with~$U_0 (e \otimes y) = e \otimes y'$.''.
\end{erratum}
\begin{erratum}{moved-dfn-selfdual}%
    A pre-Hilbert~$\scrB$-module~$X$ is \Define{self dual}
    if every bounded~$\scrB$-linear~$\tau \colon X \to \scrB$
    is of the form~$\tau(x) = \langle t, x\rangle$ for some~$t \in X$.
In print the condition that~$\tau$ should be bounded
        was accidentally lost when it was moved from~\sref{dils-selfdual}.
\end{erratum}
\begin{erratum}{dils-uniform-spaces-basics}%
The assumption  that the net~$x_\alpha$ is Cauchy in point 3
    is superfluous.
\end{erratum}
\begin{erratum}{hilbmod-adjoint-exists}%
The map~$T$ should be presumed to be bounded.
\end{erratum}
\begin{erratum}{hilmod-fixed-on-V}%
The C$^*$-algebra~$\scrB$ isn't assumed to be a von Neumann algebra,
    which it should have been (for otherwise~\sref{dils-completion} wouldn't
    be applicable.)
\end{erratum}
\begin{erratum}{err159IV}%
The displayed
    inequality
\begin{equation*}
    |f(T - p_S T p_S)| \ \leq \  |f(\, (1-p_S) T | +| p_S T (1-p_S) \,)|
\end{equation*}
    should have read
    $|f(T - p_S T p_S)| \ \leq \  |f(\, (1-p_S) T \,) | +| f(\, p_S T (1-p_S) \,)|$.
\end{erratum}
\begin{erratum}{dils-stand-filter}%
The standard filter~$c_b$ for ~$b \in \scrB$
    is a map~``$\ceil{b} \scrB \ceil{b} \to \scrB$''
    instead of a map~``$\scrB \to \ceil{b} \scrB \ceil{b}$''.
\end{erratum}
\begin{erratum}{quotient-basics}%
Not every zero map is a quotient; only those into~$0$.
\end{erratum}
\begin{erratum}{compr-basics}%
Not every zero map is a comprehension; only those from~$0$.
\end{erratum}
\begin{erratum}{asrt-pristine-reverse}%
Also in the last two points of the exercise,
    the map~$h$ should be presumed to be pristine.
\end{erratum}
\begin{erratum}{eff-inv-lemma}%
``commutant'' was meant, but ``commutator'' was written.
\end{erratum}
\begin{erratum}{exc-purec-no-biproduct}%
The condition
    ``$f \after \varphi = \langle x, (\,\cdot\,) x\rangle$''
        is a typo.
        What was meant is~``$f (\varphi(T, c)) = \langle x, T x\rangle $
        for all~$c \in \scrC$
        and~$T \in \scrB(\scrH)$''.
The same mistake was made writing
    ``$\pi_1 \after \varphi = \langle x, (\,\cdot\,) x\rangle$'' and
    ``$\pi_2 \after \varphi = \langle y, (\,\cdot\,) y\rangle$''.
\end{erratum}

% vim: se ft=tex.latex :
