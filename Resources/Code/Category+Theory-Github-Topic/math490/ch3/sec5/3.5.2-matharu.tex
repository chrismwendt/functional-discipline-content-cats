\documentclass[main.tex]{subfiles}
\begin{document}
	
	
	\paragraph{}
	\begin{exercise}
		Prove that the colimit of any small functor $ \func{F}{\sC}{\Set} $ is
		isomorphic to the set $ \pi_0(\int F) $ of connected components of the 
		category of elements of $ F. $ What is the colimit cone?
	\end{exercise}
	
	\begin{proof}
		Since $ F $ is a small functor by assumption, we can construct the 
		colimit of
		$ F $ in the usual way. Define $ \func{\io_c}{Fc}{\coprod_{c\in 
			\ob\sC}Fc/\sim},$ where $\sim $ is the smallest equivalence 
			relation 
		such that for all $\func{f}{c}{c'} $ in $ \sC $ and $ x\IN Fc,$ we have 
		$\io_c(x)\sim\io_{c'}Ff(x).$  
		Recall that the objects of $\int F$ are ordered pairs $(c,x),$ where 
		$c$ is in $ \sC \AND 
		x$ is in $ Fc, $ and morphisms are defined as maps
		 $ \func{f}{c}{c'} \IN \sC$ such that $ Ff(x)=x'. $ 
		 If we restrict ourselves to the set of connected components of 
		 $ \int F, $ we get $ \pi_0(\int F). $
		 
		 First note that two elements of in the colimit of $ F $ are 
		 related if and only if $ \io_c(x)\sim \io_{c'}$ for all $ 
		 \func{f}{c}{c'} \IN\sC.$ Next two objects $A\AND B\IN \ob\int F $ are 
		 related if and only if there is zig zag of morphisms between them.
		 Now observe the that between any two objects $ (Fj,x)\AND (Fk,x') $
		 there is a map $ \func{f}{x}{x'} $ such that
		 $\xymatrix{(Fj,x)\ar[r]^{f}& (Fk,x')}.$
		This diagram tells us that two objects in $\pi_0(\int F)$
		are connected under by $Ff.$ This relationships puts the 
		generators of each equivalence relation in one-to-one correspondence
		via the map $ [(c,x)]\mapsto [\io_c(x)]. $ Since the generators
		are in bijection with each other the equivalence relations
		generated by them are also in bijection. The colimit cone is a 
		natural transformation from $ F $ to $ \colim F.$ Each leg of the
		cone is a homomorphism that preserves connected components. 
		\end{proof}
	
\end{document}