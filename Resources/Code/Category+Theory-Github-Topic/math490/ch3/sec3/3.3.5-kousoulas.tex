\documentclass[main.tex]{subfiles}
\begin{document}

\paragraph{}
\begin{exercise}
	Show that the forgetful functors \(\func{U}{\Setp}{\Set}\) and
	\(\func{U}{\Topp}{\Top}\) fail to preserve coproducts and explain why this
	results demonstrates that the connectedness hypothesis in Proposition
	3.3.8(ii) is necessary.
\end{exercise}

\begin{proof}
	Consider first pointed sets \((A,a)\) and \((B,b)\) where neither is a
	singleton. Letting \(C\defeq (A\setminus\{a\})\amalg(B\setminus\{b\})\),
	then \((C,*)\), where \(*\) is some available symbol, realizes the coproduct
	of \((A,a)\) and \((B,b)\) in \(\Setp\). The inclusion maps here are defined
	as they are for disjoint unions normally augmented by taking either \(a\) or
	\(b\) to \(*\) as necessary. However, under the forgetful functor \(U\), the
	coproduct of \(A\) and \(B\) is simply the disjoint union \(A\amalg B\),
	which (with some fussing) properly contains \(C\).

	That this problem also affects topological spaces can been seen merely by
	noting that in taking finite sets with the discrete topology we have
	essentially recreated the category \(\Fin\) because all maps from a discrete
	space are continuous. Given more realistic topological spaces, the coproduct
	of two pointed spaces involves stitching them together at that point while
	the coproduct of the underlying spaces has them essentially floating free of
	each other.

	To see why the connectedness hypothesis is necessary it is instructive to
	examine pushouts: where coproducts are the colimit of a diagram from a
	discrete category \(\bullet\quad\bullet\), pushouts are the colimit of a
	connected diagram \(\bullet\gets\bullet\to\bullet\). When we take a diagram
	of this form in \(\Setp\), we have two functions that preserve the points of
	each object: \(f(c)=a\) and \(g(c)=b\).
	\begin{equation*}\begin{tikzcd}
			(C,c) \ar[r, "f"] \ar[d, "g"] & (B,b) \\
			(A,a)
	\end{tikzcd}\end{equation*}
	When we apply \(U\) and ``forget'' about the points, the functions \(f\) and
	\(g\) preserve their importance. For a set \(Z\) to be the colimit of this
	diagram in \(\Set\) with inclusions \(\iota_A\) and \(\iota_B\), it must be
	the case \(\iota_Af(c)=\iota_Bg(c)\) which forces \(\iota_A(a)=\iota_B(b)\).
	This is precisely the point at which \(U\) failed to preserve the coproduct.

	% Given another object \((Z,z)\) with maps
	% \(\func{f}{(A,a)}{(Z,z)}\) and \(\func{f}{(B,b)}{(Z,z)}\), it is easy to see
	% that we may factor each
\end{proof}
\end{document}
