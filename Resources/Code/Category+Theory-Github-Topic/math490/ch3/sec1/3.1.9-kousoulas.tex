\documentclass[main.tex]{subfiles}

\begin{document}

\paragraph{}
\begin{exercise}
	Show that if \(\sJ\) has an initial object, then the limit of any functor
	indexed by \(\sJ\) is the value of that functor at an initial object. Apply
	the dual of this result to describe the colimit of a diagram indexed by a
	successor ordinal.
\end{exercise}

\begin{proof}
	We begin by showing that an initial object of \(\sJ\) gives a limit for any
	functor \(\func{F}{\sJ}{\sC}\) where \(\sC\) is some category. Letting \(*\)
	be an initial object of \(\sJ\), for any object \(j\) in \(\sJ\) there is
	precisely one map \(\func{\lm_j}{*}{j}\). For \(*\) to form a cone over
	\(F\) we need a family of morphisms from \(F*\) to \(Fj\) for each object
	\(j\) in \(sJ\). An obvious (but not necessarily required) choice is to take
	\(F\lm_j\). Now we need only check that for every morphism
	\(\func{f}{j}{k}\) in \(\sJ\) the identity \(FfF\lm_j=F\lm_k\). Because
	\(f\lm_j=\lm_k\) this is precisely the condition that \(F\) is
	functorial.\footnote{Note that this is weaker than saying that any triangle
		in the image of \(F\) must commute.
		\(\begin{tikzcd}[ampersand replacement=\&, cramped, column sep=.5em]
			\&F* \ar[dr] \ar[dl] \\ Fj \ar[rr] \&\& Fk
		\end{tikzcd}\) It is entirely possible to create new composable pairs if
		\(F\) is not injective on objects. Even if \(Fj=Fj'\), \(F\lm_j\) and
		\(F\lm_{j'}\) may be distinct. Given a map \(\func{f}{j}{k}\), it need
		not be true that \(FfF\lm_{j'}=F\lm_k\), but this is immaterial to the
		naturality of the \(F\lm_j\).}

	Now suppose that \(c\) is the apex of a cone over \(F\), so there is a
	distinguished map \(\func{\rho_j}{c}{Fj}\) for each \(j\) in \(\sJ\)
	satisfying the naturality condition of a cone. For \(F*\) to be the limit of
	\(F\), there must be a unique map \(\func{\psi}{c}{F*}\) such that
	\(\rho_j=F\lm_j\psi\). In particular we need that \(\rho_*=F\lm_*\psi\),
	however, \(\lm_*\) is the identity on \(*\), so we have \(\rho_*=\psi\).
	Thus the leg of the cone from \(c\) to \(F*\) gives us the unique map we
	require.

	\[\begin{tikzcd}
			c \ar[d, "\rho_*" left] \ar[dr, "\rho_j"] \\
			F* \ar[r, "F\lm_j" below] & Fj
	\end{tikzcd}\]

	Note that both limits and initial objects are unique up to a canonical
	isomorphism, which conforms with our result. If \(\sJ\) had multiple initial
	objects, by the argument above all of these would give limits of a cone
	based on \(\sJ\), and the unique isomorphism between two initial objects
	would transfer to \(\sC\) to give us the canonical isomorphism between
	limits.

	\subparagraph{}
	Now consider the case of a successor ordinal category \(\mbb{n}\) and a
	functor \(\func{F}{\mbb{n}}{\sC}\). Note first that a successor ordinal
	\(n\) by definition is the successor of some other ordinal denoted
	\(n-1\).\footnote{In contrast a limit ordinal
		like \(\om\) or \(2\om\) is not the successor of any other ordinal and
	is instead the least upper bound of an otherwise unbounded collection.}
	Considered as a poset, any ordinal is a well-ordered chain:\footnote{This
		chain need not be finite. The ordinal \(\om+1=\om\cup\set{\om}\) is a
	successor ordinal with maximum element \(\om\).} the set of all strictly
	smaller ordinals ordered under inclusion. If \(\mbb{n}\) is a successor
	ordinal, then this poset has maximum element \(n-1\), i.e. \(m\subset n-1\)
	for any \(m\in\mbb{n}\).

	Considered as a category, a poset has as maps the elements of a relation, in
	this case \(\subset\), and as objects the elements of the poset. Thus the
	condition that \(n-1\) is a maximum element is precisely saying that \(n-1\)
	is a terminal object in \(\mbb{n}\).

	The dual of the above result gives precisely that the colimit of a functor
	indexed by a diagram with a terminal object is the value of the functor at
	the terminal object. Instead of considering an apex \(c\) with maps leading
	into the image of our diagram we instead consider a nadir with maps leading
	from our diagram to \(c\). It is clear that a terminal object will satisfy
	the argument above with the arrows reversed. This means that the functor
	\(\func{F}{\mbb{n}}{\sC}\) has as a colimit \(F(n-1)\).
\end{proof}
\end{document}
