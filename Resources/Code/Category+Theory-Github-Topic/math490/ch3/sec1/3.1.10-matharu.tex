\documentclass[main.tex]{subfiles}
\begin{document}

\paragraph{}
\begin{exercise}
	If $ (a,b) $ are positive integers satisfying the universal property of
	(3.1.20) show that the pair $ (-a,-b) $ also satisfies the same universal
	property. Explain why this observation does not imply that the pullback is
	ill-defined.
\end{exercise}
\begin{proof}
	\textbf{Mark how do I get the angle on this diagram}
	For reference (3.1.20) is the following commutative diagram.
	\begin{equation*}\begin{tikzcd}
			\ZZ\ar[r, dashrightarrow, "b"] \ar[d, dashrightarrow, "a"]
			\ar[dr, phantom, "\lrcorner", very near start] &
			\ZZ\ar[d, "n"] \\ \ZZ \ar[r, "m"] &\ZZ
	\end{tikzcd}\end{equation*}
	We know from undergraduate algebra the least common multiple of two integers is
	only unique up to multiplication by a unit. Therefore, $ (-a,-b) $ should
	satisfy the above diagram as well.
	The pullback is unique \footnote{Like everything else in category theory
	probably} up to isomorphism. Since -1 is an automorphism on $ \ZZ $ that takes$
	(a,b)$ to $ (-a,-b), $  so in either case the diagrams are essentially the
	same thing as far as we are concerned.
\end{proof}
\end{document}
