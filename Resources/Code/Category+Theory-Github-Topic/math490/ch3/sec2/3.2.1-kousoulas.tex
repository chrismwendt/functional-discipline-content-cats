\documentclass[main.tex]{subfiles}
\begin{document}

\paragraph{}
\begin{exercise}
	A \textbf{small category} can be redefined to be a particular diagram in
	\Set. The data is given by a pair of suggestively-named sets with functions
	\[\begin{tikzcd}[cramped]
			\mor\sC \ar[r, shift left=2, "\dom"]
			\ar[r,shift right=2, "\cod" below] &
			\ob\sC \ar[l, "\ID" description, inner sep=0.1ex]
	\end{tikzcd}\]
	together with a ``composition function'' yet to be defined. Use a pullback
	to define the set of composable pairs of morphisms, which serves as the
	domain for the composition function, and formulate the axioms for a category
	using commutative diagrams in \Set. When \Set is replace by a category
	\(\sE\) with pullbacks, this defines a \textbf{category internal to
	\(\sE\)}.
\end{exercise}

\subparagraph{}
For the sake of concision we introduce the aliases: \(\cO=\ob\sC\) and
\(\cM=\mor\sC\).
Pullbacks in \Set are realized by subsets of the Cartesian product on which
two functions agree. In this case we have
\[\begin{tikzcd}
		\cM\by_\cO\cM \ar[d, "\pi_1" left] \ar[r, "\pi_2"]
		\ar[d, phantom, shift left=4, "\lrcorner", very near start] &
		\cM \ar[d, "\dom"] \\
		\cM \ar[r, "\cod"] &
		\cO
\end{tikzcd}\]
where \[\cM\by_\cO\cM=\set{(f,g)\in\mor\sC^2|\cod(f)=\dom(g)},\] and
\(\pi_1\) and \(\pi_2\) are the expected projection maps out of
\(\cM\by_\cO\cM\). This set contains precisely the ordered pairs \((f,g)\) for
which the composition \(gf\) is defined. Composition is thus rendered as a
function \(\func{\mu}{\cM\by_\cO\cM}{\cM}\). We can then describe the axioms
of a category as commutative diagrams involving \(\mu\), cod, dom, id, and
products thereof.

First, given composable morphisms \(f\) and \(g\) as above, we have
\(\dom(f)=\dom(gf)\) and \(\cod(g)=\cod(gf)\). In other words \({\dom\pi_1=\dom\mu}\) and
\({\cod\pi_2=\cod\mu}\).
\[\begin{tikzcd}
		\cM\by_\cO\cM \ar[r, shift left, "\pi_1"]
		\ar[r, shift right, "\mu" below] &
		\cM \ar[r, "\dom"]  &
		\cO
	\end{tikzcd}
	\qtextq{and}
	\begin{tikzcd}
		\cM\by_\cO\cM \ar[r, shift left, "\pi_2"]
		\ar[r, shift right, "\mu" below] &
		\cM \ar[r, "\cod"] &
		\cO
\end{tikzcd}\]

Similarly, given an object \(c\), the identity \(\id_c\) must satisfy
\(\dom(\id_c)=c=\cod(\id_c)\), i.e. \(\dom\ID=\id_\cO=\cod\ID\).
\[\begin{tikzcd}
		\cO \ar[dr, "\ID"] \ar[rr, "\id_\cO"] && \cO \\ &
		\cM \ar[ur, shift right, "\cod" below] \ar[ur, shift left, "\dom"]
\end{tikzcd}\]

To state the next two axioms we require more sophisticated pullbacks. First,
we need ternary pullbacks. Let
\[\cM\by_\cO\cM\by_\cO\cM=\set{(f,g,h)|\cod(f)=\dom(g)\AND\cod(g)=\dom(h)},\]
i.e. sets of composable triples, with projections \(\pi_1\), \(\pi_2\), and
\(\pi_3\) into \(\cM\). It is easily seen that this is the pullback in two
different diagrams.
\[\begin{tikzcd}
		\cM\by_\cO\cM\by_\cO\cM \ar[d, "\pi_1\by\pi_2" left] \ar[r, "\pi_3"]
		\ar[d, phantom, shift left=4, "\lrcorner", very near start] &
		\cM \ar[d, "\dom"] \\
		\cM\by_\cO\cM \ar[r, "\cod\pi_2"] &
		\cO
	\end{tikzcd}\qtextq{and}
	\begin{tikzcd}
		\cM\by_\cO\cM\by_\cO\cM \ar[d, "\pi_1" left] \ar[r, "\pi_2\by\pi_3"]
		\ar[d, phantom, shift left=4, "\lrcorner", very near start] &
		\cM\by_\cO\cM \ar[d, "\dom\pi_1"] \\
		\cM \ar[r, "\cod"] &
		\cO
\end{tikzcd}\]
Because of our restriction the composite projections \(\pi_1\by\pi_2\) and
\(\pi_2\by\pi_3\) give us valid composable pairs, so each is a legitimate map
onto \(\cM\by_\cO\cM\).

The condition of associativity states that \(h(gf)=(hg)f\) for any composable
triple \((f,g,h)\), which we can now express diagrammatically.
\[\begin{tikzcd}
		\cM\by_\cO\cM\by_\cO\cM \ar[d, "\mu\by\pi_3" left]
		\ar[r, "\pi_1\by\mu"] &
		\cM\by_\cO\cM \ar[d, "\mu"] \\
		\cM\by_\cO\cM \ar[r, "\mu"] &
		\cM
\end{tikzcd}\]

Finally, given any morphism \(f\) we have \(\ID(\cod(f))f=f=f\ID(\dom(f))\).
For this law we must consider the composable pairs containing identity
morphisms. These can be isolated as the image of the following two pullbacks
inside \(\cM\by_\cO\cM\).
\[\begin{tikzcd}
		\cM\by_\cO\cO \ar[d, "\pi_1" left] \ar[r, "\pi_2"]
		\ar[d, phantom, shift left=4, "\lrcorner", very near start] &
		\cO \ar[d, "\id_\cO"] \\
		\cM \ar[r, "\cod"] &
		\cO
	\end{tikzcd}\qtextq{and}
	\begin{tikzcd}
		\cO\by_\cO\cM \ar[d, "\pi_1" left] \ar[r, "\pi_2"]
		\ar[d, phantom, shift left=4, "\lrcorner", very near start] &
		\cM \ar[d, "\dom"] \\
		\cO \ar[r, "\id_\cO"] &
		\cO
\end{tikzcd}\]
Equipped with these objects, we can express the identity law by the following
commutative diagram.
\[\begin{tikzcd}
		\cM\by_\cO\cO \ar[r, "\pi_1\by\ID"] \ar[dr, "\pi_1" below] &
		\cM\by_\cO\cM \ar[d, "\mu"] &
		\cO\by_\cO\cM \ar[l, "\ID\by\pi_2" above] \ar[dl, "\pi_2"] \\ &
		\cM
\end{tikzcd}\]
\end{document}
