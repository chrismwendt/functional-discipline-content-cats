\documentclass[main.tex]{subfiles}
\begin{document}

\paragraph{}
\begin{exercise}
	What is a functor between groups, regarded as one-object categories?
\end{exercise}

\begin{proof}
	Recall that a group as a category has a single object \(x\), and that each
	element of the group is a morphism in the category.  All domains and
	codomains are that object \(x\).  There is one identity morphism \(1_x\),
	which is the identity element in the group.  Composition is the same as
	multiplication in this context. \\ \\ A functor between groups \(\sC\) and
	\(D\) with respective objects \(x_1\) and \(x_2\) must trivially be such
	that \(Fx_1 = x_2\).  Our primary concern is the behavior of the functor
	on the morphisms. We require that for a functor \(F1_x = 1_{Fx}\) for
	all objects \(x \in \ob\sC\), which in this case just implies that \(1_{x_1}\)
	is taken to \(1_{x_2}\).  Additionally, we require \(F(\dom(f)) =
	\dom(Ff)\) and \(F(\cod(f)) = \cod(Ff)\) for all morphisms \(f\) in the
	first category.  This is a trivial requirement, as \(F(\dom(f)) = \dom(Ff)
	= F(\cod(f))= \cod(Ff) = x_2\) regardless of \(f\).  Finally we require
	that if \(f\) and \(g\) are a composable pair of morphisms in \(\sC\), then
	\(F(fg) = FfFg\). However, all morphisms in \(\sD\) are composable, and
	this implies that \(F(f*g) = Ff*Fg\) in the notation of groups with
	operation \(*\).  This property and the preservation of identities are
	directly the definition of a group homomorphism, so this functor is simply a
	group homomorphism.
\end{proof}

\end{document}
