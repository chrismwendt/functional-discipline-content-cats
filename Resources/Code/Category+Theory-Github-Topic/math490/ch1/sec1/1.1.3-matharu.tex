\documentclass[main.tex]{subfiles}
\begin{document}

\paragraph{}
\begin{exercise}
For any category \(\sC\) and any object \(c\) in \(\sC\) show that:
\begin{enumerate}
	\item There is a category \(c/\sC\) whose objects are morphisms
		\(\func{f}{c}{x}\) with domain \(c\) and in which a morphism from
		\(\func{f}{c}{x}\) to \(\func{g}{c}{y}\) is a map \(\func{h}{x}{y}\)
		between the codomains so that the triangle \[ \xymatrix{&c\ar[dr]^g
		\ar[dl]_f& \\ x \ar[rr]_h && y } \] \tbf{commutes}, i.e., so that
		\(g=hf.\)

	\item There is a category \(\sC/c\) whose objects are morphisms
		\(\func{f}{x}{c}\) with codomain \(c\) and in which a morphism from
		\(\func{f}{x}{c}\) to \( \func{g}{y}{c} \) between the domains so that
		the triangle \[ \xymatrix{x \ar[rr]^h\ar[dr]_f && y\ar[dl]^g\\ &c & } \]
		\tbf{commutes}, i.e., so that \(f=gh\)

		The categories \(c/\sC\AND\sC/c\) are called \tbf{slice categories} of
		\(\sC\) \tbf{under} and \tbf{over} \(c,\) respectively.
\end{enumerate}
\end{exercise}

\begin{proof}
	First we must determine the form of the objects and morphisms in \(c/\sC.\)
	The objects of \(c/\sC\) are diagrams of the following form.
	\[ \xymatrix{c\ar[r]^f& x} \]
	The morphisms in \(c/\sC\) are diagrams of the following form.
	\[ \xymatrix{&c\ar[dr]^g \ar[dl]_f& \\ x \ar[rr]_h && y } \]
	Though this is notation is by no means standard, to help distinguish between
	morphisms in \(\sC\) and morphisms in the slice categories, we will
	define \(h'\) as a short hand for the diagram with the morphism \(h\) as the
	bottom arrow (or top in \(\sC/c\)). Notice that both the objects are
	commutative diagrams in \(\sC.\) We could also think of the objects as
	functors\footnote{Functors are defined in section 1.3. Right now the diagrams here are just helpful tools to keep track of equations. Diagrams are made formal in section 1.6.} from  the category \(\mathbb{2}\) and the morphisms as functors
	from the category \(\mathbb{3}.\) By the way we have defined morphisms the
	only reasonable choices for the domain and codomain of \(h'\) are \(f\) and
	\(g\) respectively.

	We can see how to compose two compatible morphisms \(\func{i'}{e}{f}\) and
	\(\func{h'}{f}{g}\) in \(c/\sC\) by looking at the following diagram in
	\(\sC.\)
	\[ \xymatrix{&c\ar[dr]\ar[dr]^g\ar[dl]\ar[dl]_e\ar[d]_f\\
	z\ar[r]_{i}&x\ar[r]_h&y } \]
	Since \((hi)e=h(ie)=hf=g\) in \(\sC\) the diagram commutes, and the
	composition \(hi\) can be thought of as a member of \(c/\sC\) denoted as
	\((hi)'\) with domain and codomain \(e\) and \(g\), respectively. Using the diagram
	notation \((hi)'\) is denoted as follows.

	\[ \xymatrix{&c\ar[dr]^g \ar[dl]_e& \\ x \ar[rr]_{hi} && y } \]
	Because we have defined composition in \(c/\sC\) in terms of composition
	in \(\sC,\) \(c/\sC\) inherits the associativity of \(\sC\) That
	is for composable morphisms  \(h',\;i',\AND j' \IN \sC\) we have
	\[ (h'i')j'=(hi)j=h(ij)=h'(i'j'). \]

	We need to obtain the identity morphism of each object \(f\) in
	\(c/\sC.\) To do so notice that the follow diagram commutes, because
	\(\sC\) is a category.
	\[ \xymatrix{c\ar[d]_f\ar[dr]^f\\ x\ar[r]_{1_x}&x } \]
	Looking at the same diagram from a different perspective we see that
	\(1_x\) actually acts as the identity morphism for \(f\) in \(c/\sC.\)
	Since we were careful when defining the morphisms in \(c/\sC,\) this
	identity is well defined. If we had defined the morphisms in \(c/\sC\)
	to be anything less than a commutative diagrams, it would seem as
	\(1'_x\) could serve as the identity for multiple objects in \(c/\sC.\)
	This issue is not restricted to identity morphisms, but this ambiguity is most
	obvious in the case of identity morphisms. However, since we defined
	morphisms appropriately, we can use the notation defined earlier to write
	\(1'_{x}=1_{f}:f\to f\) without any ambiguity. This notion can be used to
	obtain the left and right identities by considering the following
	commutative diagram in \(\sC\)
	\[\xymatrix{&&c\ar[dr]_g\ar[drr]^g\ar[dl]^f\ar[dll]_f\\
	x\ar[r]_{1_x}&x\ar[rr]_h&&y\ar[r]_{1_y}&y }\]
	Translating the above diagram into the slice category notation we have
	that \(h'1_{f}= h'=1_{g}h'.\) We have shown that \(c/\sC\) satisfies all
	the axioms of a category.

	We can use the same procedure to show that \(\sC/c\) is also a category.
	The only difference is direction of each arrow. Hence this proof will be
	relatively terse. The objects in \(c/\sC\) are the following diagram.
	\[\xymatrix{x\ar[r]^f& c}\]
	A morphisms  \(h'\) in \(\sC/c\) has the form of the following diagram.
	\[\xymatrix{x \ar[rr]^h\ar[dr]_f && y\ar[dl]^g\\ &c & }\]
	The domain of \(h'\) is \(f\) and the codomain of \(h\) is \(g.\) We
	define composition on \(\sC/c\) by taking compatible in \(\sC/c\)
	morphisms \(i':e\to f\AND h':f\to g\) and observing the following
	diagram in \(\sC.\)

	\[\xymatrix{z\ar[r]^i\ar[dr]_e &x\ar[d]^f\ar[r]^h&y\ar[dl]^g\\ &c & }\]
	Again this diagram commutes since \(g(hi)=(gh)i=fi=e\) in \(\sC.\)
	We see that \((ih)\) is member of \(\sC/c\) with domain \(e\) and
	codomain \(g\) and is denoted as follows.
	\[\xymatrix{z \ar[rr]^{ih}\ar[dr]_e && y\ar[dl]^g\\ &c & }\]
	Just like we did in the previous case, we have defined composition in
	terms of the composition in \(\sC.\) Hence the associativity is
	inherited. That is given composable morphisms \(i',h',\AND j'\) we have
	\[(j'h')i'=(jh)i=j(hi)=j'(h'i').\]
	We can obtain the identity element for each object \(f\IN \sC/c\) in the
	exact same way as before.
	\[\xymatrix{c\ar[d]_f\ar[dr]^f\\ x\ar[r]_{1_x}&x }\]
	Since the above diagram commutes we can write \(1'_x=1_{f}:f\to f\) To
	get an identity for an arbitrary element observe that the diagram below
	commutes and gives us that \(1_{f}h'=h'=h'1_{g}\).
	\[\xymatrix{x\ar[r]^{1_{x}}&x\ar[rr]^h&&y\ar[r]^{1_{y}}&y
							 \\&&c\ar[urr]_g\ar[ur]^g\ar[ul]_f\ar[ull]^f}\]
	Therefore both \(c/\sC\) and \(\sC/c\) are categories in their own right.
\end{proof}

If you looked over to the next page and read the definition of opposite
categories, you should notice that \(((c/(\sC)^{\op}))^{\op}=(\sC/c).\)
If we knew about opposite categories beforehand we could have just
proved that the \(c/\sC\) is a category and then cited this result and
been done (since the opposite category is category, it's in the name
after all), without all the extra tedium of swapping arrows.
\end{document}
