\documentclass[main,tex]{subfiles}
\begin{document}

\paragraph{}
\begin{exercise}
	A functor \(F\) defines a \textbf{subfunctor} of \(G\) if there is a
	natural transformation \(\alpha \colon F \Rightarrow G \) whose
	components are monomorphisms. In the case of \(G \colon \sC^{\op} \to
	\msf{Set}\), a subfunctor is given by a collection of subsets \(Fc
	\subseteq Gc\) so that each \(Gf \colon Gc \to Gc'\) restricts to a
	function \(Ff \colon Fc \to Fc'\). Characterize those subsets that
	assemble into a subfunctor of the representable functor
	\(\msf{C}(-,c)\).
\end{exercise}
\begin{proof}
	First let us clarify what the exercise is actually
	asking us to do. For some \(c \in \ob\sC\), we want to choose subsets
	\(Fd \subseteq \msf{C}(d,c)\) for every \(d \in \ob\sC\), such that:
	\begin{itemize}
		\item \(F\) is a subfunctor of \(\msf{C}(-,c)\); that is, for every \(f \colon d' \to d \in \sC\), the following commutes for some monomorphisms \(\iota', \iota\).
			$$
			\xymatrix{
				Fd\ar[d]_{Ff}\ar[r]^{\iota} & \msf{C}(d, c)\ar[d]^{- \circ f}\\
				Fd'\ar[r]_{\iota'} & \msf{C}(d', c)
			}
			$$
		\item The function \(- \circ f: \msf{C}(d, c) \to \msf{C}(d', c)\) restricts to \(Ff\); that is, \((- \circ f)(g) = Ff(g)\) for all \(g \in Fd\).

	\end{itemize}
	Notice that since every \(Fd \subseteq \msf{C}(d,c)\), we automatically
	have an inclusion morphism from \(Fd\) to \(\msf{C}(d,c)\). Since every
	\(Fd\) and \(\msf{C}(d,c)\) are sets, this morphism is trivially monic.
	So let \(\iota \colon Fd \to \msf{C}(d,c)\) in the diagram be the
	inclusion of \(Fd\) into \(\msf{C}(d,c)\) (and define \(\iota'\)
	similarly.)

	In order for this diagram to commute, for any \(g \in Fd\), \((- \circ
	f)\iota g\) (going along the top) must be equal to \(\iota'Ffg\) (going
	along the bottom). But \(\iota\) and \(\iota'\) represent inclusion, so
	\(\iota' Ffg = Ffg\) and \(\iota g = g\). So for the diagram to
	commute, it must be the case that \((- \circ f)g = Ffg\) for any \(g
	\in Fd\); in other words \(Ff\) must equal \((- \circ f)\) when applied
	to elements of \(Fd\). But this is the same as saying that \((- \circ
	f)\) restricts to \(Ff\)! So the first and second conditions are
	equivalent -- in order for \(F\) to be a subfunctor of
	\(\msf{C}(-,c)\), for any \(f \colon d' \to d \in C\), \(Ff \colon Fd
	\to Fd'\) must be exactly the restriction of \((- \circ f)\) to \(Fd\).

	Therefore it must be the case that for every \(g \colon d
	\to c \in Fd, f \colon d' \to d \in \msf{C}\), \(Ff(g) = (- \circ f)g =
	gf\). But in order for this to be possible, \(gf\) must always be in
	\(Fd'\) independent of the choice of \(g\) and \(f\). On the other
	hand, if
	that is the case, than obviously \((- \circ f)\) is defined in
	\(Fd'\) for all \(f \colon d' \to d\); so the restriction of \((- \circ
	f)\) to \(Fd\) -- that is to say, \(Ff\) -- is defined for every \(f\)!
	So we can always create a subfunctor \(F\) if \(Fd\) and \(Fd'\)
	fulfill this condition (for every \(Fd'\),) and we cannot create a
	subfunctor if they do not.  More formally:

	A family of subsets \(\{Fd \subseteq \msf{C}(d, c) \ | \ d \in
	\ob\msf{C}\}\) can be assembled into a subfunctor \(F\) of
	\(\msf{C}(-,c)\) if and only if: For every \(d' \in \msf{C}\), every
	\(f \colon d' \to d \in \msf{C}\) and every \(g \colon d \to c \in
	Fd\), \(gf \in Fd'\). So this condition characterizes all of the
	subsets \(Fd\) requested by the exercise.
\end{proof}

\end{document}
