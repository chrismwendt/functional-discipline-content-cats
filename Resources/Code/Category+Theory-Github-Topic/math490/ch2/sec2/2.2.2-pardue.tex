\documentclass[main.tex]{subfiles}
\begin{document}

\paragraph{}

\begin{exercise}
Explain why the Yoneda lemma does not dualize to classify natural transformations from an arbitrary set-valued functor to a represented functor.
\end{exercise}

In the argument in Exercise 2.2.i, we replace $\sC$ by $\sC^\op$ changing the directions of the morphisms, but not the direction of natural transformations.
That is a reason based on the argument given.

But, here is an example showing that there is no such lemma with natural
transformations from $F$ to $\sC(c,-)$. Let $\sC=\Set$, $F=1_\Set$ and
$c=\emptyset$. Then $\Set(\emptyset,-)$ is a constant functor taking every set
$x$ to $\{\emptyset\}$ and every $f\colon x\rightarrow y$ to the identity function
for this set. There is a unique natural transformation
$\alpha\colon 1_\Set\Rightarrow\Set(\emptyset,-)$. This is not in bijection with
$1_\Set\emptyset=\emptyset$.
\end{document}
