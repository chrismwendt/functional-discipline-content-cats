\setvariables [document] [assessment={Test 1}]
\setvariables [document] [date=2020-02-04]

\startcomponent *

\product  prd-MATH1550


% \startchapter [title={2020-01-31}]

\startexercise
	Evaluate the limits. Use \m{+∞} or \m{-∞} when appropriate.
	\startitemize [i, columns, joinedup]
		\item  \m{\lim_{x → 4}  \frac{\sqrt{2 x + 3} - 3}{x - 3}}
		\item  \m{\lim_{x → 5^+} \frac{-2}{x - 5}}
		\item  \m{\lim_{x → 5^+} \frac{x - 7}{x(x - 5)^2}}
		\item  \m{\lim_{x → ∞} \frac{51743 - 7 x - 8 x^3}{4 x^4 + 14}}
		\item  \m{\lim_{x → -∞} \frac{\sqrt{3 x^2 - 9}}{2 x + 8}}
		\item  \m{\lim_{x → -∞} \frac{2 x - \sin(x)}{x + 3}}
	\stopitemize
	\startformula

	\stopformula  
\stopexercise

\startexercise
	Use the Intermediate Value Theorem to prove that \m{f(x) = x^2 + 2x - 6} has a root in the interval \m{[1, 2]}.
\stopexercise

\startexercise
	A ball is thrown directly upward from the edge of a cliff and travels in such a way that \m{t} seconds later, its height in meters is given by the position function \m{s(t) = -16 t^2 + 56 t + 24}. Find the average velocity over the time interval \m{[1,2]}.
\stopexercise

\startexercise
	Find the value of the constant \m{C} which makes \m{f} continuous everywhere,
	\startformula
		\text{where } \qquad f(x) = 
		\startmathcases
			\NC  4 x^2 - 2  \MC  \text{if } x < 1  \NR
			\NC  C x   + 3  \MC  \text{if } x ≥ 1  \NR
		\stopmathcases .
	\stopformula 
\stopexercise

\startexercise
	\startitemize [i, joinedup]
		\item  Use the limit definition of derivative to find \m{f'(x)}, where \m{f(x) = 1 - 2 x + 3 x^2}.
		\item  Find the equation of the tangent line to \m{f(x)} at \m{x = -4}.
		\item  What is the instantaneous rate of change of \m{f(x)} at \m{x = 2}?
	\stopitemize
\stopexercise

\startexercise
	Find the vertical and horizontal asymptotes of the function
	\startformula
		f(x) = \frac{3 x^2 + 2}{2 x^2 + x - 3} .
	\stopformula
\stopexercise

% \startsolution
% 	\blank[16*big]
% \stopsolution

% \stopchapter
\stopcomponent