\setvariables [document] [assessment={Test 3}]
\setvariables [document] [date=2020-03-13]

\startcomponent *

\product  prd-MATH1550


% \startchapter [title={2020-01-31}]

\startexercise
	Find the absolute extrema (maxima and minima) for the function \m{f(x) = 2 x^3 - 3 x^2 - 12 x + 5} on the interval \m{[0, 4]}, and determine where those values occur.
\stopexercise
\blank[12*big]

\startexercise
	For the function \m{f(x) = 2 x^3 + 9 x^2 + 12 x + 36}
	\startitemize [i]
		\item  Identify the critical points.
			\blank[6*big]

		\item  Find the intervals on which the function is increasing or decreasing.
			\blank[6*big]

		\item  Identify all local extrema.
			\blank[6*big]

		\item  Identify the inflection points.
			\blank[6*big]

		\item  Find the interval on which the function is concave up or concave down.
			\blank[6*big]

	\stopitemize
\stopexercise

\startexercise
	Verify that the hypotheses of the mean-value theorem are satisfied for \m{f(x) = x^3} on the interval \m{[-3, 5]}, and find all values of \m{c} that satisfies the conclusion of the theorem.
\stopexercise
\blank[16*big]

\startexercise
	A landscape architect wishes to enclose a rectangular garden of area \m{1000 m^2} on one side by a brick wall costing \m{80 \$ m^{-1}} and on the other side by a metal fence costing \m{20 \$ m^{-1}}. Which dimensions minimize the total cost?
\stopexercise
\blank[16*big]

\startexercise
	Find the limits. Show your work.
	\startitemize [i, columns, three]
		\item  \m{\lim_{x → 0} \frac{e^x - 1}{x^2 + 3x}}
		\item  \m{\lim_{x → 0^+} (x + \cos(x))^{\frac{1}{x}}}
		\item  \m{\lim_{x → 0} x \csc(x)}
	\stopitemize
\stopexercise
\blank[16*big]

% \stopchapter
\stopcomponent