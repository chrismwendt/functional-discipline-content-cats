\startchapter [title={Discrete Probability Spaces}]

	\setuptabulate [
		split=yes,
		header=text,
		frame=on,
		title={\tfa Notations}]

	\starttabulate [|l|l|M|m|]
		\FL  \FL
		\NC  Term  \NC  Description  \NC  \text{Symbol/Idea}  \NC  \text{Coin toss Example}  \NR
		\ML  \ML

		\NC  sample space  \NC  set of outcomes  \NC  Ω  \NC  \bcrl{H, T}  \NR  \ML

		\NC  outcome  \NC  arbitrary outcome  \NC  ω ∈ Ω  \NC  H  \NR  \ML

		\NC  event  \NC  subset of sample space  \NC  E  \NC  ∅, \bcrl{H}, \bcrl{T}, \bcrl{H, T}  \NR  \ML

		\NC  mutually exclusive  \NC  events with  \NC  E_1 ∩ E_2 = ∅  \NC  \bcrl{H} and \bcrl{T}  \NR
		\NC  events  \NC  empty intersection  \NC  \NC  \NR  \ML

		\NC  probability mass  \NC  weightage of  \NC  p: Ω → [0, 1], \text{ with}  \NC  p(H) = \frac13, p(T) = \frac23  \NR
		\NC  function  \NC   each outcome  \NC  ∑_ω p(ω) = 1  \NC  \NR  \ML

		\NC  probability  \NC  (of an event)  \NC  ℙ: 2^Ω → [0, 1],  \NC  ℙ(∅) = 0, ℙ(\bcrl{H, T}) = 1  \NR
		\NC  \NC  \NC  ℙ(E) = ∑_{ω ∈ Ω} p(ω)  \NC  \NR  \ML

		\NC  random variable  \NC  a function  \NC  X: Ω → ℝ  \NC  X(H) = 1, X(T) = 0  \NR  \ML
		\BL
	\stoptabulate

\stopchapter


\startchapter [title={The algebra of sets}]
	
	Suppose \m{Ω} be a set, and \m{E, F, G} are subsets of \m{Ω}. We have the following operators:
	\startitemize [1, joinedup]
		\item  complement(\m{E^∁})
		\item  union (\m{E ∪ F})
		\item  intersection(\m{E ∩ F})
		\item  set difference (\m{E ∖ F = E ∩ F^∁})
	\stopitemize

	Then the following \emph{laws} hold.
	\startitemize [1, joinedup]
		\item  Commutativity of union (\m{E ∪ F = F ∪ E}) and intersection (\m{E ∩ F = F ∩ E}).
		\item  Associativity of union (\m{(E ∪ F) ∪ G = E ∪ (F ∪ G)}) and intersection (\m{(E ∩ F) ∩ G = E ∩ (F ∩ G)}).
		\item  Distributivity of union over intersection (\m{E ∩ (F ∪ G) = (E ∩ F) ∪ (E ∩ G)}).
		\item  Distributivity of intersection over union (\m{E ∪ (F ∩ G) = (E ∪ F) ∩ (E ∪ G)}).
		\item  Idempotence: \m{E ∪ E = E}, \m{E ∩ E = E}.
		\item  Domination: \m{Ω ∪ E = Ω}, \m{Ω ∩ E = E}.
		\item  Absorption: \m{E ∪ (E ∩ F) = E}, \m{E ∩ (E ∪ F) = E}.
		\item  De Morgan: \m{(E ∪ F)^∁ = E^∁ ∩ F^∁}, \m{(E ∩ F)^∁ = E^∁ ∪ F^∁}.
		\item  Involution: \m{(E^∁)^∁ = E}.
		\item  ⋯
	\stopitemize
	
	



\stopchapter


\startchapter [title={Axiomatic probability theory}]

	\startdefinition [title={Probability axioms}]
		A \emph{non-negative valued} function \m{ℙ} defined on the set of events is called a \emph{probability measure} if the following hold.
		\startitemize [n, nowhite, after]
			\item  (null empty set)  \m{ℙ(∅) = 0}.
			\item[countable-additivity]  (countable additivity)  For any sequence of mutually exclusive events \m{E_1, E_2, ⋯}, we have \m{ℙ\brnd{⨆_{n=1}^∞ E_n} = ∑_{n=1}^∞ ℙ(E_n)}.
			\item  (probability)  \m{ℙ(Ω) = 1}.
		\stopitemize
	\stopdefinition

	
	\comment{Draw Venn diagrams for all of the following.}

	\startproposition
		\m{ℙ(E^∁) = 1 - ℙ(E)}.
	\stopproposition

	\startproof
		Since \m{E ∩ E^∁ = ∅}, by \in{Axiom}[countable-additivity] we have \m{1 = ℙ(Ω) = ℙ(E ⊔ E^∁) = ℙ(E) + ℙ(E^∁)}.
	\stopproof


	\startproposition
		If \m{E ⊂ F}, then \m{ℙ(E) ≤ ℙ(F)}.
	\stopproposition

	\startproof
		Note that \m{F = E ⊔ (F ∖ E)}. So by \in{Axiom}[countable-additivity] we have \m{ℙ(F) = ℙ(E ⊔ (F ∖ E)) = ℙ(E) + ℙ(F ∖ E)}. Therefore, \m{ℙ(F) - ℙ(E) = ℙ(F ∖ E)}, which is non-negative since probability is a non-negative set function.
	\stopproof


	\startproposition [title={Inclusion-Exclusion}]
		\m{ℙ(E ∪ F) = ℙ(E) + ℙ(F) - ℙ(E ∩ F) .}.
	\stopproposition

	\startproof
		\startitemize [n, nowhite, after]
			\item  \m{E ∪ F = (E ∖ F) ⊔ (F ∖ E) ⊔ (E ∩ F)}, so \m{ℙ(E ∪ F) = ℙ(E ∖ F) + ℙ(F ∖ E) + ℙ(E ∩ F)}.
			\item  \m{E = (E ∖ F) ⊔ (E ∩ F)}, so \m{ℙ(E) = ℙ(E ∖ F) + ℙ(E ∩ F)}, and similarly
			\item  \m{F = (F ∖ E) ⊔ (E ∩ F)}, so \m{ℙ(F) = ℙ(F ∖ E) + ℙ(E ∩ F)}.
		\stopitemize
		Combining the above,
		\startformula  \startalign[n=2]
			\NC  ℙ(E ∪ F)  \NC =  ℙ(E ∖ F) + ℙ(F ∖ E) + ℙ(E ∩ F)  \NR
			\NC  \NC =  (ℙ(E) - ℙ(E ∩ F)) + (ℙ(F) - ℙ(E ∩ F)) + ℙ(E ∩ F)  \NR
			\NC  \NC =  ℙ(E) + ℙ(F) - ℙ(E ∩ F) .
		\stopalign  \stopformula
	\stopproof

\stopchapter
