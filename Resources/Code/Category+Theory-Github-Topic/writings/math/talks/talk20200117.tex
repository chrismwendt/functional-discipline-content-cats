%%%%%%%%%%%%%
% Variables %
%%%%%%%%%%%%%

% Language
\setvariables [document] [language=en]

% Version and mode
\setvariables [document] [version=final]    % {final, concept, temporary}
\setvariables [document] [mode=presentation]    % {presentation, manuscript, handout}

% Colors
% https://coolors.co/app
\setvariables [document] [color-link-external=royalblue]
\setvariables [document] [color-link-internal=violetred]
\setvariables [document] [color-background-0=white]
\setvariables [document] [color-background-1=mistyrose]
\setvariables [document] [color-background-2=whitesmoke]
\setvariables [document] [color-foreground-0=black]
\setvariables [document] [color-foreground-1=firebrick4]
\setvariables [document] [color-foreground-2=darkslategray]

% Fonts
% Options: palatino, xitsbidi, euler
\setvariables [document] [font=palatino]
\setvariables [document] [fontsize-presentation=38pt]
\setvariables [document] [fontsize-document=12pt]

% Information
\setvariables [document] [title={Stochastic Differential Equations with Anticipating Initial Conditions}]
% \setvariables [document] [subtitle={Joint Mathematics Meetings 2020}]
\setvariables [document] [author={Sudip Sinha}]
\setvariables [document] [coauthors={Hui-Hsiung Kuo \blank[small] Jiayu Zhai}]
\setvariables [document] [event={Joint Mathematics Meetings 2020}]
\setvariables [document] [place={Denver, CO, USA}]
\setvariables [document] [date={2020-01-17}]
\setvariables [document] [keyword={mathematics, probability, stochastic, integral, calculus, anticipating, applications}]

% Logo
% \setvariables [document] [logo=MC_Logo_Symbol_672x668.png]

% Environment
\environment env-talks


%%%%%%%%%%%%%%%%%%%%%%%%%%%%%%%%%%%%
% This is where the document starts.
%%%%%%%%%%%%%%%%%%%%%%%%%%%%%%%%%%%%

\starttext

%%%%%%%%%%%%%%%%
% Front matter %
%%%%%%%%%%%%%%%%
\startfrontmatter

% \setupbackgrounds [page] [
% 	background={color,backgraphics,foreground},
% 	backgroundcolor=\getvariable{document}{color-background-0}]
% \startcolor [\getvariable{document}{color-foreground-0}]    % text color


% \startmode [handout]
% \startcolumns
% \stopmode


% Introduction
\startmode [presentation]

\startslide
	\startalign [middle]

		{\tfd \getvariable{document}{title}}

		\blank[2*line]

		{\tfb \getvariable{document}{author}}

		\blank[line]

		\comment{Joint work with}

		{\tfa \getvariable{document}{coauthors}}

		\blank[2*line]

		\getvariable{document}{event}

		\getvariable{document}{place}

		\getvariable{document}{date}
	\stopalign
\stopslide


% Table of contents
\startslide [title={Outline}]
	\placecontent
\stopslide
\stopmode

% \startmode [manuscript]

% This presentation is going to be on two topics:
% \startitemize[n,nowhite,after]
% 	\item  Generalization of stochastic integrals developed primarily by Professor H.-H. Kuo
% 	\item  Applications of generalization in large deviations theory
% \stopitemize

% \stopmode

\stopfrontmatter



%%%%%%%%%%%%%%%
% Body matter %
%%%%%%%%%%%%%%%
\startbodymatter

% \setupbackgrounds [page] [
% 	background={color,backgraphics,foreground},
% 	backgroundcolor=\getvariable{document}{color-background-0}]
% \startcolor [\getvariable{document}{color-foreground-0}]    % text color

\startsection [title={Introduction and motivation}, reference=sec:intro]
\startmode [presentation]

\startslide [title={Review and notations}]

	\startitemize [4]

		\item  Let \m{T ∈ (0, ∞)}, \m{t ∈ [0, T]}, and \m{B_t} is a Brownian motion.

		\item  The \good{Itô integral} for an \good{adapted} process \m{X_t} w.r.t. \m{B_t} is defined as
			\startformula
				∫_0^t X_s \d B_s = \ℙlim_{\norm[Δ_n] → 0}  ∑_{j = 0}^{n - 1} X_{t_j} Δ B_j , \ \text{ where \m{Δ_n} is a partition of \m{[0, t]} and \m{Δ B_j = B_{t_{j + 1}} - B_{t_j}}} .
			\stopformula
		% \startitemize [n, joinedup]
		% 	\item  For \good{adapted} step processes \m{X_t(ω) = ∑_{j = 0}^{n-1} ξ_j(ω) 𝟙_{[t_j, t_{j + 1})}(t)}, define \m{∫_0^T X_t \d B_t  =  ∑_{j = 0}^{n-1} ξ_j Δ B_j}.
		% 	\item  For \good{adapted} process \m{X_t} such that \m{∫_0^T \abs[X_t]^2 \d t < ∞} a.s., use adapted step processes approximating \m{X} to extend the integral using limit in probability.
		% \stopitemize

		\item  The process \m{Y_t = ∫_0^t X_s \d B_s} is a martingale.

		\item  An \good{Itô process} is an \good{adapted} process of the form \m{X_t = X_0 + ∫_0^t m_s \d s + ∫_0^t σ_s \d B_s}.

			Equivalently expressed as \m{\d X_t = m_t \d t + σ_t \d B_t}.

		\item  \good{Itô formula}:  If \m{X_t} is an Itô process and \m{f(t, x) ∈ C^{1, 2}(ℝ × ℝ)}, then \m{Y_t = f(t, X_t)} is also an Itô process given by
			\startformula
				\d Y_t = \d f(t, X_t)  =  ∂_t f (t, X_t) \d t  +  ∂_x f (t, X_t) \d X_t  +  \frac12 ∂_x^2 f (t, X_t) \, (\d X_t)^2 .
			\stopformula

	\stopitemize
\stopslide

\startslide [title={Remarks on the Itô integral}]

	\startitemize [4]

		\item  The integrand must an \good{adapted} stochastic process.

		\item  Iterated integrals: \m{∫_0^T ∫_0^T \d B_u \d B_v \ugly{= ∫_0^T B_T \d B_v = ?}}

		% \item  Note that \m{𝔼(B_t^2) = \ugly{t ≠ 0}}, so \ugly{no martingale property} \frown.

		\item  What about anticipating stochastic differential equations like

			\startcombination [2*1]
				{\startframed[width=0.5\textwidth, frame=off]
					\startformula
						\startmathcases
							\NC   \d X_t =  \MC  X_t \d B_t \NR
							\NC  \ \ X_0 =  \MC  \okay{B_T}  \NR
						\stopmathcases  \qquad \qquad  \text{or}
					\stopformula
				\stopframed}{}
				{\startframed[width=0.5\textwidth, frame=off]
					\startformula
						\startmathcases
							\NC   \d Y_t =  \MC  \okay{B_T} \d B_t \NR
							\NC  \ \ Y_0 =  \MC  1  \NR
						\stopmathcases ?
					\stopformula
				\stopframed}{}
			\stopcombination
		\item  Problem: We want to define \m{∫_0^T Z_t \d B_t}, where \m{Z_t} is not (necessarily) adapted.

		\item  Some approaches
		\startitemize [5, joinedup, columns]
			\item  Enlargement of filtration \cite[short][Itô1978]
			\item  White noise theory
			\item  Malliavin calculus
			\item  Numerous others
		\stopitemize
	\stopitemize
\stopslide


\startsection [title={The generalized integral}, reference=sec:2-sided-int]
\startmode [presentation]

\startslide [title={Definition of the integral \cite[short][AyedKuo2008, AyedKuo2010]}]
	
	\startitemize [4]

		\item  A process \m{Y^t} and filtration \m{ℱ_t} are called \okay{instantly independent} if \m{Y^t} and \m{ℱ_t} are independent \m{∀ t}.

			Example: The process \m{B_T - B_t} is instantly independent of the filtration generated by \m{B_t}.

		\item  Idea
			\startitemize[n, joinedup]

				\item  Decompose the integrand into \good{adapted} and \okay{instantly independent} parts.

				\item  Evaluate the \good{adapted} and the \okay{instantly independent} parts at the \good{left} and \okay{right} endpoints.

			\stopitemize

		\item  Consider two continuous stochastic processes, \good{\m{X_t} adapted} and \okay{\m{Y^t} instantly independent} w.r.t. \m{ℱ_t}. Then the integral \m{∫_0^T \good{X_t} \okay{Y^t} \d B_t} is \important{defined} as
			\startformula
				∫_0^T \good{X_t} \okay{Y^t} \d B_t  ≜  \ℙlim_{\norm[Δ_n] → 0}  ∑_{j = 0}^{n - 1} \good{X_{t_{j}}} \okay{Y^{t_{j+1}}} ΔB_j .
			\stopformula
			% provided that the limit exists in probability.

		\item  Now, for any stochastic process \m{Z(t) = ∑_{k = 1}^n \good{X_t^{(k)}} \okay{Y^t_{(k)}}} we extend the definition by linearity.

			This is well-defined \cite[short][HwangKuoSaitôZhai2016].
	\stopitemize
\stopslide

\startslide [title={A simple example}]
	
	\startformula \startalign
		\NC  ∫_0^t B_T \d B_s
			\NC =  ∫_0^t (\good{B_s} + \okay{(B_T - B_s)}) \d B_s
			    =  \good{∫_0^t B_s \d B_s}  +  \okay{∫_0^t (B_T - B_s) \d B_s}
		\NR \NC
			\NC =  \good{\Ltwolim_{\norm[Δ_n] → 0} ∑_{j = 0}^{n - 1} B_{t_j} Δ B_j}
				+  \okay{\Ltwolim_{\norm[Δ_n] → 0} ∑_{j = 0}^{n - 1} (B_T - B_{t_{j + 1}}) Δ B_j}
		\NR \NC
			\NC =  \Ltwolim_{\norm[Δ_n] → 0} ∑_{j = 0}^{n - 1} \brnd[B_T - Δ B_j] Δ B_j
		\NR \NC
			\NC =  B_T ⋅ \Ltwolim_{\norm[Δ_n] → 0} ∑_{j = 0}^{n - 1} Δ B_j - \Ltwolim_{\norm[Δ_n] → 0} ∑_{j = 0}^{n - 1} (Δ B_j)^2
		\NR \NC
			\NC =  B_T B_t - t
	\stopalign \stopformula

	% \startitemize [4]

	% 	\item  Note that \m{𝔼(B_T B_t - t) = 0}.

	% 	\item  In general, \m{𝔼 ∫_0^t Z(s) \d B_s = 0}. \good{\smile}
	% \stopitemize
\stopslide

\startslide [title={The near-martingale property}]

	\startitemize [4]

		\item  Question: What is the analogues of the martingale property?

		% \item  Answer for martingales: near-martingales \cite[short][KuoSaeTangSzozda2012].

		\item  Example: \m{𝔼(B_T B_t - t ∣ ℱ_s) = B_s^2 - s ≠ B_T B_s - s}. \frown

			But \m{𝔼(B_T B_s - s ∣ ℱ_s) = B_s^2 - s}. \smile

		\item  Let \m{Z_t} be a process such that \m{𝔼\abs[Z_t] < ∞ \ ∀ t}, and \m{0 ≤ s ≤ t ≤ T}. Then \m{Z_t} is called a \important{near-martingale} if \m{𝔼(Z_t ∣ ℱ_s) = 𝔼(Z_s ∣ ℱ_s)}.
			% \startitemize [5, joinedup]
			% 	\item  \important{near-martingale} if \m{𝔼(Z(t) ∣ ℱ_s) = 𝔼(Z(s) ∣ ℱ_s)},
			% 	\item  \important{near-submartingale} if \m{𝔼(Z(t) ∣ ℱ_s) ≥ 𝔼(Z(s) ∣ ℱ_s)},
			% 	\item  \important{near-supermartingale} if \m{𝔼(Z(t) ∣ ℱ_s) ≤ 𝔼(Z(s) ∣ ℱ_s)}.
			% \stopitemize
	\stopitemize

	\starttheorem [title={\cite[short][KuoSaeTangSzozda2012]}]
		Let \m{f} and \m{ϕ} be continuous functions on \m{ℝ}. Under integrability conditions, the processes \m{X_t = ∫_0^t f(B_t) ϕ(B_T - B_t) \d B_t} and \m{Y^t = ∫_t^T f(B_t) ϕ(B_T - B_t) \d B_t} are near-martingales.
		% \startformula
		% 	X_t = ∫_0^t f(B_t) ϕ(B_T - B_t) \d B_t
		% 	\qquad \text{ and } \qquad
		% 	Y^t = ∫_t^T f(B_t) ϕ(B_T - B_t) \d B_t
		% \stopformula
		% are near-martingales.
	\stoptheorem

	\starttheorem [title={\cite[short][HwangKuoSaitôZhai2017]}]
		Let \m{Z_t} be a stochastic process bounded in \m{L^1}, and \m{X_t = 𝔼(Z_t ∣ ℱ_t)}. Then \m{X_t} is a martingale if and only if \m{Z_t} is a near-martingale.
	\stoptheorem
\stopslide

\startslide [title={The general Itô formula \cite[short][HwangKuoSaitôZhai2016]}]

	\starttabulate [|c|m|m|]
		\NC  Process
		\NC  \text{Definition}
		\NC  \text{Representation}

		\NR
		\FL

		\NC  \good{Itô}
		\NC  \good{X_t = X_0 + ∫_0^t m_s \d s + ∫_0^t σ_s \d B_s}
		\NC  \good{\d X_t = \ \ m_t \d t + σ_t \d B_t}

		\NR

		\NC  \okay{instantly independent}
		\NC  \okay{Y^t = Y^T + ∫_t^T η^s \d s + ∫_t^T ς^s \d B_s}
		\NC  \okay{\d Y^t = - η^t \d t - ς^t \d B_t}

		\NR
		\BL
	\stoptabulate

	Here \m{η^t} and \m{ς^t} are instantly independent such that \m{Y^t} is also instantly independent.

	\starttheorem[title={\cite[short][HwangKuoSaitôZhai2016]}]
		Let \good{\m{\d X_t = m_t \d t + σ_t \d B_t}} be an \good{Itô} process, and \okay{\m{\d Y^t = - η^t \d t - ς^t \d B_t}} be a \okay{instantly independent} process. If \m{f(t, x, y) ∈ C^{1, 2, 2}(ℝ × ℝ × ℝ)}, then
		\startformula \startalign[n=3]
			\NC  \d f(t, X_t, Y^t)  = \NC  ∂_t f (t, X_t, Y^t) \d t  \NC  +  \good{∂_x f (t, X_t, Y^t) \d X_t + \frac12 ∂_x^2 f (t, X_t, Y^t) (\d X_t)^2}  \NR
			\NC  \NC  \NC  + \okay{∂_y f (t, X_t, Y^t) \d Y^t - \frac12 ∂_y^2 f (t, X_t, Y^t) (\d Y^t)^2} .  \NR
		\stopalign \stopformula
	\stoptheorem
\stopslide



\startsection [title={Conditional expectation}, reference=sec:conditional-expectation-sde]
\startmode [presentation]

\startslide [title={Linear stochastic differential equations}]

	\startdefinition
		Define the \important{exponential process} with parameters \m{α} and \m{β} by
		\startformula
			ℰ^{(α, β)}_t = \exp\brnd[∫_0^t α_s \d B_s + ∫_0^t {\brnd[β_s - \frac12 α_s^2]} \d s] .
		\stopformula
	\stopdefinition
	
	\starttheorem [title={\cite[short][HwangKuoSaitôZhai2016]}]
		The solution of the stochastic differential equation
		\startformula
			\startmathcases
				\NC   \d X_t =  \MC  α_t X_t \d B_t + β_t X_t \d t \NR
				\NC  \ \ X_0 =  \MC  ψ(B_T)  \NR
			\stopmathcases
		\stopformula
		is given by \m{X_t  =  ψ\brnd[B_T - \important{∫_0^t α_s \d s}] ℰ^{(α, β)}_t}. 
	\stoptheorem
\stopslide

\startslide [title={Motivating question}]
	
	What can we say about the conditional expectation of the solution of the stochastic differential equation
	\startformula
		\startmathcases
			\NC   \d X_t =  \MC  α_t X_t \d B_t + β_t X_t \d t \NR
			\NC  \ \ \, X_0 =  \MC  ψ(B_T)  \NR
		\stopmathcases ?
	\stopformula

	In particular, if \m{Y_t = 𝔼(X_t ∣ ℱ_t)}, can we expect \m{Y_t} to be the solution of the stochastic differential equation
	\startformula
		\startmathcases
			\NC   \d Y_t =  \MC  α_t Y_t \d B_t + β_t Y_t \d t \NR
			\NC  \ \ \, Y_0 =  \MC  𝔼ψ(B_T)  \NR
		\stopmathcases
		?
	\stopformula
\stopslide

\startslide [title={Unexpected behaviour}]
	
	\starttheorem [title={\cite[short][KuoSinhaZhai2018]}]
		Suppose \m{α_t ∈ L^2[0, T]}, \m{β_t} is adapted with \m{𝔼∫_0^T \abs[β_t]^2 \d t < ∞}, and \m{ψ: ℝ → ℝ} has power series expansion at \m{0} with infinite radius of convergence, and \m{ψ'} denotes the derivative of \m{ψ}.

		Consider the two stochastic differential equations
		
		\startcombination [2*1]
			{\startframed[width=0.5\textwidth, frame=off]
				\startformula
					\startmathcases
						\NC   \d X_t =  \MC  α_t X_t \d B_t + β_t X_t \d t \NR
						\NC  \ \ X_0 =  \MC  ψ(B_T)  \NR
					\stopmathcases  \qquad \text{and}
				\stopformula
			\stopframed}{}
			{\startframed[width=0.5\textwidth, frame=off]
				\startformula
					\startmathcases
						\NC   \d \overline{X}_t =  \MC  α_t \overline{X}_t \d B_t + β_t \overline{X}_t \d t \NR
						\NC  \ \ \overline{X}_0 =  \MC  ψ'(B_T)  \NR
					\stopmathcases .
				\stopformula
			\stopframed}{}
		\stopcombination
		Denote \m{Y_t = 𝔼(X_t ∣ ℱ_t)} and \m{\overline{Y}_t = 𝔼(\overline{X}_t ∣ ℱ_t)}.

		Then \m{Y_t} satisfies the stochastic differential equation
		\startformula
			\startmathcases
				\NC   \d Y_t =  \MC  α_t Y_t \d B_t + β_t Y_t \d t + \important{\overline{Y}_t \d B_t} \NR
				\NC  \ \ Y_0 =  \MC  𝔼ψ(B_T)  \NR
			\stopmathcases .
		\stopformula
	\stoptheorem
\stopslide

\startslide [title={A brief detour: Hermite polynomials}]
	
	\startitemize [4]

		\item  An Hermite polynomial of degree \m{n} with parameter \m{ρ} is given by
			\startformula
				H_n(x; ρ) = (-ρ)^n e^{\frac{x^2}{2ρ}} ∂_x^n e^{-\frac{x^2}{2ρ}} .
			\stopformula

		% \item  Some useful equalities for Hermite polynomials:
		% \startitemize [m, joinedup]
		% 	\item  \m{∂_x H_n(x; ρ) = n H_{n−1}(x; ρ)}
		% 	\item  \m{∂_x^2 H_n(x; ρ) = -2 ∂_ρ H_n(x; ρ)}
		% 	\item  \m{H_n(x + y; ρ) = ∑_{k=0}^n \binom{n}{k} H_{n−k}(x; ρ) y^k}
		% \stopitemize

		\item  The first few Hermite polynomials are: \m{1, x, x^2 - ρ, x^3 - 3ρx, x^4 - 6 ρ x^2 + 3 ρ^2, …}.

		\item  Hermite polynomials form an orthonormal basis of \m{L^2(ℝ, γ)}, where \m{γ} is the Gaussian measure with mean \m{0} and variance \m{ρ}.

		\item  For fixed \m{n ∈ ℕ}, the stochastic process \m{X_t = H_n(B_t; t)} is a martingale, and %\m{\d X_t = n H_{n-1}(B_t; t) \d B_t}.
			\startformula
				\d X_t = n H_{n-1}(B_t; t) \d B_t .
			\stopformula
	\stopitemize
\stopslide

\startslide [title={Initial condition: Hermite polynomials}]
	
	\starttheorem [title={\cite[short][KuoSinhaZhai2018]}]
		Suppose \m{α_t ∈ L^2[0, T]}, \m{β_t} is adapted with \m{𝔼∫_0^T \abs[β_t]^2 \d t < ∞}, and let \m{n} be a fixed natural number. Let \m{X_t} be the solution of
		\startformula
			\startmathcases
				\NC   \d X_t =  \MC  α_t X_t \d B_t + β_t X_t \d t  \NR
				\NC  \ \ X_0 =  \MC  H_n(B_T;\ T) ,  \NR
			\stopmathcases
		\stopformula
		and \m{Y_t = 𝔼(X_t ∣ ℱ_t)}.

		Then \m{Y_t} satisfies the stochastic differential equation
		\startformula
			\startmathcases
				\NC   \d Y_t =  \MC  α_t Y_t \d B_t + β_t Y_t \d t + \important{n H_{n-1}{\brnd[B_t - ∫_0^t α_s \d s;\ t]} ℰ^{(α, β)}_t \d B_t} \NR
				\NC  \ \ Y_0 =  \MC  0  \NR
			\stopmathcases ,
		\stopformula
		and is explicitly given by
		\startformula
			Y_t = H_n \brnd[B_t - \important{∫_0^t α_s \d s};\ t] ℰ^{(α, β)}_t .
		\stopformula
	\stoptheorem
\stopslide

\startslide [title={Initial condition: differentiable function in \m{L^2(ℝ, γ)}}]
	
	\starttheorem [title={\cite[short][KuoSinhaZhai2018]}]
		Suppose \m{α_t ∈ L^2[0, T]}, \m{β_t} is adapted with \m{𝔼∫_0^T \abs[β_t]^2 \d t < ∞}.

		Let \m{ψ(x) = ∑_{n = 0}^∞ c_n H_n(x; T)} be a differentiable function in \m{L^2(ℝ, γ)}.

		Consider the two stochastic differential equations
		
		\startcombination [2*1]
			{\startframed[width=0.5\textwidth, frame=off]
				\startformula
					\startmathcases
						\NC   \d X_t =  \MC  α_t X_t \d B_t + β_t X_t \d t \NR
						\NC  \ \ X_0 =  \MC  ψ(B_T)  \NR
					\stopmathcases  \qquad \text{and}
				\stopformula
			\stopframed}{}
			{\startframed[width=0.5\textwidth, frame=off]
				\startformula
					\startmathcases
						\NC   \d \overline{X}_t =  \MC  α_t \overline{X}_t \d B_t + β_t \overline{X}_t \d t \NR
						\NC  \ \ \overline{X}_0 =  \MC  ψ'(B_T)  \NR
					\stopmathcases .
				\stopformula
			\stopframed}{}
		\stopcombination
		Denote \m{Y_t = 𝔼(X_t ∣ ℱ_t)} and \m{\overline{Y}_t = 𝔼(\overline{X}_t ∣ ℱ_t)}.

		Then \m{Y_t} satisfies the stochastic differential equation
		\startformula
			\startmathcases
				\NC   \d Y_t =  \MC  α_t Y_t \d B_t + β_t Y_t \d t + \important{\overline{Y}_t \d B_t} \NR
				\NC  \ \ Y_0 =  \MC  𝔼ψ(B_T)  \NR
			\stopmathcases ,
		\stopformula
		and is explicitly given by
		\startformula
			Y_t = ∑_{n = 0}^∞ c_n H_n \brnd[B_t - \important{∫_0^t α_s \d s};\ t] ℰ^{(α, β)}_t .
		\stopformula
	\stoptheorem
\stopslide

\stopmode

\stopsection


\startsection [title={A larger class of initial conditions}, reference=sec:ic-Wiener]
\startmode [presentation]

\startslide [title={Initial condition: Wiener integral}]
	
	\bold{Question}: Can we extend the class of initial conditions?

	\starttheorem [title={\cite[short][KuoSinhaZhai2018]}]
		Let \m{α_t ∈ L^2[0, T], β_t ∈ L^1[0, T], h_t ∈ L^2[0, T], ψ(t) ∈ C^2(ℝ)}. Then the (unique) solution of the stochastic differential equation
		\startformula
			\startmathcases
				\NC   \d X_t =  \MC  α_t X_t \d B_t + β_t X_t \d t  \NR
				\NC  \ \ X_0 =  \MC  ψ\brnd[∫_0^T h_s \d B_s]  \NR
			\stopmathcases
		\stopformula
		is given by
		\startformula
			X_t = ψ\brnd[∫_0^T h_s \d B_s - \important{∫_0^t α_s h_s \d s}] ℰ^{(α, β)}_t .
		\stopformula
	\stoptheorem
\stopslide

\startslide [title={An example}]
	
	Consider the stochastic differential equation
	\startformula
		\startmathcases
			\NC   \d X_t =  \MC  X_t \d B_t  \NR
			\NC  \ \ X_0 =  \MC  ψ\brnd[∫_0^T B_s \d s]  \NR
		\stopmathcases .	
	\stopformula

	Using Itô lemma, we rewrite \m{∫_0^T B_s \d s = ∫_0^T (T - s) \d B_s}.
	
	Now using the previous theorem, we get
	\startformula
		X_t = ψ\brnd[∫_0^T B_s \d s - {\brnd[Tt - \frac12 t^2]}] e^{B_t - \frac12 t} .
	\stopformula
\stopslide

\stopmode

\stopsection

\stopbodymatter



%%%%%%%%%%%%%%%
% Back matter %
%%%%%%%%%%%%%%%
\startbackmatter

% \setupbackgrounds [page] [
% 	background={color,backgraphics,foreground},
% 	backgroundcolor=\getvariable{document}{color-background-0}]
% \startcolor [\getvariable{document}{color-foreground-0}]    % text color

\startmode [presentation]

\startslide

	\startalign [middle]

		\blank[4*line]

		{\tfd Thank you!}

	\stopalign
\stopslide

\startsubject [title={Appendix}]

\startslide [title={Iterated integrals}]

	\starttheorem[title={\cite[short][Itô1951MWI]}]
		Let \m{f ∈ L^2([0, T]^n)} and \m{\bad{\hat{f}}} be its symmetrization. Then
		\startformula
			∫_{[0, T]^n} f(t_1, …, t_n) \d B_{t_1} … \d B_{t_n}  =  \bad{n!} ∫_0^T ⋯ ∫_0^{\bad{t_{n-2}}} \brnd[∫_0^{\bad{t_{n-1}}} \bad{\hat{f}}(t_1, …, t_n) \d B_{t_n}] \d B_{t_{n-1}} … \d B_{t_1} .
		\stopformula
		% \m{\hat{f}(t_1, …, t_n)  =  \frac{1}{n!} ∑_{σ ∈ S_n} f(t_{σ(1)}, …, t_{σ(n)})}
	\stoptheorem

	\starttheorem[title={\cite[short][AyedKuo2010]}]
		Let \m{f ∈ L^2([0, T]^n)}. Then
		\startformula
			∫_{[0, T]^n} f(t_1, …, t_n) \d B_{t_1} … \d B_{t_n}  =  ∫_0^T ⋯ ∫_0^T f(t_1, …, t_n) \d B_{t_n} … \d B_{t_1} .
		\stopformula
	\stoptheorem

	Example\cite[short][HwangKuoSaitôZhai2016]: For the new integral, \m{∫_0^T \brnd[∫_0^T B_u \d u] \d B_v = ∫_0^T \brnd[∫_0^T B_u \d B_v] \d u}.
\stopslide

\startslide [title={A generalization of Itô isometry}]

	\starttheorem [title={\cite[short][KuoSaeTangSzozda2012]}]
		Let \m{ϕ} be an analytic function on \m{ℝ}. Then under integrability conditions and for each \m{t},
		\startformula
			𝔼\bsqr[{\brnd[∫_0^t ϕ(B_T - B_s) \d B_s]}^2]  =  ∫_0^t 𝔼\bsqr[{\brnd[ϕ(B_T - B_s)]}^2] \d s
		\stopformula
	\stoptheorem

	\starttheorem [title={\cite[short][KuoSaeTangSzozda2013]}]
		Let \m{f} and \m{ϕ} be \m{C^1} functions on \m{ℝ}. Then
		\startformula \startalign
			\NC  𝔼\bsqr[{\brnd[∫_0^T f(B_t) ϕ(B_T - B_t) \d B_t]}^2]
			\NC  =  ∫_0^T 𝔼\bsqr[{\brnd[f(B_t) ϕ(B_T - B_t)]}^2] \d t
			\NR  \NC  \NC  \quad + 2 ∫_0^T∫_0^{\bad{t}} 𝔼\bsqr[f(B_s) ϕ'(B_T - B_s) f'(B_s) ϕ(B_T - B_s)] \d s \d t .
		\stopalign \stopformula
	\stoptheorem
\stopslide

\startslide [title={A generalization of Girsanov theorem}]

	\starttheorem[title={\cite[short][KuoPengSzozda2013b]}]
		Let \m{X_t} and \m{Y^t} be continuous square-integrable stochastic processes such that \m{X_t} is adapted and \m{Y^t} is instantly independent.

		Then the translated stochastic process \m{W_t = B_t - ∫_0^t (X_t + Y^t) \d t} is a near-martingale under the probability measure \m{\widetilde{ℙ}} defined by the Radon-Nikodym derivative \m{\frac{\d \widetilde{ℙ}}{\d ℙ} = 𝓔^{(X + Y, 0)}_T}.
	\stoptheorem
\stopslide
\stopsubject

\startslide

	\startalign [middle]

		\blank[4*line]

		{\tfd Thank you!}

	\stopalign
\stopslide

\startslide [title={Bibliography}]
	\placelistofpublications
\stopslide
\stopmode
\stopbackmatter

\stoptext
