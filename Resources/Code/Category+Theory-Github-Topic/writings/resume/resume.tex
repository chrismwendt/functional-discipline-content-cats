%%  Useful links
% https://wiki.contextgarden.net/Command/framed
% https://wiki.contextgarden.net/Framed
% https://wiki.contextgarden.net/Overlays
% https://wiki.contextgarden.net/Tables_Overview
% https://wiki.contextgarden.net/TextBackground
% https://tex.stackexchange.com/questions/486078/is-there-any-alternative-for-tcolorbox-in-context
% https://tex.stackexchange.com/questions/135790/how-to-wrap-text-around-framed-floats-in-context
% https://tex.stackexchange.com/questions/486078/is-there-any-alternative-for-tcolorbox-in-context
% https://tex.stackexchange.com/questions/30427/simplest-way-to-overlay-a-text-rectangle-label-an-image/30435
% https://meeting.contextgarden.net/2008/talks/2008-08-22-hartmut-metapost/mptut-context2008.pdf
% http://www.pragma-ade.com/general/manuals/details.pdf
% https://mailman.ntg.nl/pipermail/ntg-context/2012/069802.html
% https://en.wikipedia.org/wiki/X11_color_names

\setupcolors [state=start]       % otherwise you get greyscale
\usecolors [x11]    % X11 colors, use \showcolor[x11] to get a list

% % See wiki/Command/setupinteraction
% \setupinteraction [
% 	state=start,  % make hyperlinks active
% 	menu=on,
% 	click=yes,
% 	style=normal,
% 	color=\getvariable{document}{color-link-external},   % Seems like this is the only color working
% 	contrastcolor=\getvariable{document}{color-link-internal},   % This is not working
% 	openaction=FitWidth,    % Make the PDF open in 'fit width' mode
% 	focus=standard,    % stop switching to 'fit page' mode when clicking an interdocument hyperlink
% 	title={Résumé},
% 	subtitle={Industrial},
% 	author={Sudip Sinha},
% 	date={2019-08-31},
% 	keyword={résumé, curriculum vitae}]
% \setupinteractionscreen [option=bookmark]    % wiki/Command/setupinteractionscreen


% % \setupframed[offset=5pt]

% \starttext

% \noindent % leave vertical mode
% \startframed[align=normal,width=0.75\textwidth,location=lohi]
% 	\framed[]{\tfd \sc Sudip Sinha}
% 	\quad
% 	\startframed[width=local, align=normal]
% 		Random bullshit
% 	\stopframed
	
% 	% \externalfigure[LaTeX/pic.jpg][width=.1\hsize,frame=on,location=lohi]
% 	% \startframed[align=middle,width=.8\hsize,location=lohi]
% 	% 	Lorem Ipsum

% 	% 	Neque porro quisquam est qui dolorem
% 	% \stopframed
	
% 	\startframed[align=normal,width=.95\hsize]
% 		Lorem ipsum dolor sit amet, consectetur adipiscing elit. Cras
% 		luctus augue augue, vitae vulputate ligula mattis vel. Praesent
% 		efficitur massa et neque vehicula, quis ultricies mi
% 		vestibulum. Donec facilisis, nibh eu mollis consectetur, justo
% 		elit faucibus leo, ut viverra dolor ante sit amet lorem
% 	\stopframed
% \stopframed
% \quad
% \startframed[align=normal,width=.2\textwidth,location=lohi]
% 	\starttabulate[|c|c|]
% 		\NC Lorem \NC Ipsum \NC\NR\HL
% 		\NC Lorem \NC Ipsum \NC\NR\HL
% 		\NC Lorem \NC Ipsum \NC\NR\HL
% 		\NC Lorem \NC Ipsum \NC\NR\HL
% 		\NC Lorem \NC Ipsum \NC\NR\HL
% 	\stoptabulate
% \stopframed
% \par\noindent
% \startframed[align=normal,width=\textwidth]
% 	Lorem ipsum dolor sit amet, consectetur adipiscing elit. Cras luctus
% 	augue augue, vitae vulputate ligula mattis vel. Praesent efficitur
% 	massa et neque vehicula, quis ultricies mi vestibulum. Donec
% 	facilisis, nibh eu mollis consectetur, justo elit faucibus leo, ut
% 	viverra dolor ante sit amet lorem
% \stopframed


% \page

% \hbox{
% \startframedtext
% 	[width=0.75\textwidth, align=left, frame=off,
% 	background=color,backgroundcolor=beige,
% 	foreground=color,foregroundcolor=brown]
	
% 	\startsubject[title={Education}]
% 		Here goes my educational details
% 	\stopsubject

% \stopframedtext

% \startframed [width=0.25\textwidth]
% 	Skills
% \stopframed
% }

% \hbox
% {
% 	\framed[width=2cm, align=middle, location=line] {Education}
% 	\framed[width=2cm, align=middle, location=line] {Skills}
% }

% % \ruledhbox
% %   {A
% %    \framed[width=2cm,align=middle,location=low]    {location\\equals\\low}
% %    \framed[width=2cm,align=middle,location=line]   {location\\equals\\line}
% %    \framed[width=2cm,align=middle,location=high]   {location\\equals\\high}
% %    B}

% \stoptext


%  Randomized frame
% \startuseMPgraphic{background:mathematicalargumentframe}
% 	path p;
% 	for i = 1 upto nofmultipars :
% 		p = (multipars[i]  topenlarged 1pt  bottomenlarged 1pt)  randomized 1pt;
% 		fill p withcolor green + 0.9375 * white;    % hard to change (works: red, green, blue, white, lightgray + linear combinations)
% 		draw p withcolor \MPvar{linecolor}
% 		withpen pencircle scaled \MPvar{linewidth};
% 	endfor;
% \stopuseMPgraphic

% \definetextbackground [mathematicalargumentframe] [
% 	mp=background:mathematicalargumentframe,
% 	location=paragraph,
% 	rulethickness=0.5pt,
% 	framecolor=green,    % hard to change (works: red, green, blue, white, lightgray)
% 	width=broad,
% 	leftoffset=4pt,
% 	rightoffset=4pt,
% 	before={\testpage[3]\blank[small]},
% 	after={\blank[small]}]



% %  https://mailman.ntg.nl/pipermail/ntg-context/2012/069802.html
% \usemodule[tikz]
% \usetikzlibrary[decorations.pathmorphing]

% \def\TIKZdecorationoverlay {
% 	\starttikzpicture
% 		\expanded{
% 			\draw[\overlaylinecolor, line width=\overlaylinewidth, decorate, decoration={\framedparameter{decoration}}]
% 			(0, 0)--(\overlaywidth, 0) -- (\overlaywidth, \overlayheight) -- (0, \overlayheight) -- cycle;}
% 	\stoptikzpicture}

% \defineoverlay[decoration][\TIKZdecorationoverlay]

% % \setupframed
% %    [decoration=snake]

% \starttext

% \framed[frame=off, rulethickness=3bp, background=decoration,
%           backgroundoffset=3mm, framecolor=red, align=normal]
%           {\input ward \endgraf}

% \blank[2*big]

% \externalfigure
%    [cow]
%    [frame=off, frameoffset=5pt, backgroundoffset=frame,
%    background=decoration,
%    decoration={coil, amplitude=4pt, segment length=5pt}]





% %  https://tex.stackexchange.com/questions/486078/is-there-any-alternative-for-tcolorbox-in-context
% \startuseMPgraphic{mp:cvblock}
% 	path p ; numeric w, h, o ;
% 	w := OverlayWidth ; h := OverlayHeight ; o := BodyFontSize*1/2 ;

% 	% Full box
% 	p := ((0,0) -- (0,h+2o) -- (w,h+2o) -- (w,0) -- cycle) cornered (o) ;
% 	fill p withcolor OverlayColor ;
% 	draw p withcolor OverlayLineColor withpen pencircle scaled OverlayLineWidth ;

% 	% Title box
% 	p := ((0,h) -- (0,h+2o) -- (w,h+2o) -- (w,h) -- cycle) cornered (o) ;
% 	filldraw p -- cycle withcolor OverlayLineColor ;
% 	draw textext.rt(\MPstring{cvblock}) shifted (o,h+o) withcolor OverlayColor ;

% 	setbounds currentpicture to OverlayBox ;
% \stopuseMPgraphic

% \defineoverlay[cvblock][\useMPgraphic{mp:cvblock}]

% \define\setframetitle
%   {\setMPtext{cvblock}{\strut\framedparameter{title}}}

% \defineframedtext [cvsection] [
% 	frame=off,
% 	width=\textwidth,
% 	background=cvblock,
% 	backgroundcolor=whitesmoke,
% 	framecolor=firebrick4,
% 	% rulethickness=1pt,
% 	extras=\setframetitle]

% \defineframedtext [cvsubsection] [
% 	frame=off,
% 	width=\textwidth,
% 	background=cvblock,
% 	backgroundcolor=white,
% 	framecolor=darkblue,
% 	% rulethickness=1pt,
% 	extras=\setframetitle]


% \definetextbackground[secondary][
%   location=paragraph,
%   background=color,
%   backgroundcolor=lightgray,
%   leftoffset=.5\bodyfontsize,
%   rightoffset=.5\bodyfontsize,
%   topoffset=.5\bodyfontsize,
%   bottomoffset=.5\bodyfontsize,
%   before={\startnarrower\switchtobodyfont[small]},
%   after={\stopnarrower},
%   frame=off,]



% \starttext

% \startcvsection [title={Education},% width=0.75\textwidth,
% 	backgroundcolor=lightgray]

% 	\startcvsubsection [title={LSU}]
% 		\startitemize [n, joinedup]
% 			\item  Specializing in stochastic analysis and mathematical finance
% 			\item  Vice President of the LSU SIAM chapter
% 			\item  Teaching assistant for Capstone Course, Probability Theory, Interest Theory, Linear Algebra, Calculus and Pre-calculus. The capstone course is a upper level undergraduate course in which the aim is to solve real-world problems. Previous projects include understanding kinesiology data using recurrence quantification analysis, building a dashboard for analyzing data for athletes, and analyzing 3D body shapes to get health metrics.
% 		\stopitemize
% 	\stopcvsubsection
	
% 	\samplefile{knuth}

% \stopcvsection

% \stoptext

% \input knuth

% \startsecondary [title={Title}]
% \input knuth
% \stopsecondary

% \input knuth

% \definehead [slide] [subject] [    % default formatting = subject formatting
% 	% number=no,
% 	% page=yes,
% 	style=\tfb,
% 	alternative=middle,
% 	before=,    % Negates the effect of setuphead [section].
% 	aligntitle=yes]

% \definecolor [transparentred]  [r=1,t=.05,a=1]    % TODO: remove this with a proper color
% \definetextbackground
% 	[intro]
% 	[state=start,
% 	backgroundcolor=transparentred,
% 	% backgroundoffset=.25cm,
% 	frame=off,
% 	location=paragraph,
% 	color=red]

% \starttext

% \startslide [title={Education}]
% 	A rather common way to draw attention to a passage, is to add a background. In this chapter we will therefore discuss how to enhance your document with \starttextbackground [subintro] those colorful areas that either or not follow the shape of your paragraph. \stoptextbackground\ Be warned: this chapter has so many backgrounds that you might start to dislike them.
% \stopslide

% \starttextbackground[intro]
% 	A rather common way to draw attention to a passage, is to add a background. In this chapter we will therefore discuss how to enhance your document with \starttextbackground [subintro] those colorful areas that either or not follow the shape of your paragraph. \stoptextbackground\ Be warned: this chapter has so many backgrounds that you might start to dislike them.
% \stoptextbackground



\defineparagraphs[AsymmetricColumns][n=2]

% The width of the second paragraph is automatically calculated if unspecified.
\setupparagraphs[AsymmetricColumns][1][width=.66\textwidth, rule=on]
% \setupparagraphs[AsymmetricColumns][2][rule=on]

\starttext
\startAsymmetricColumns
	\startsubject [title={Education}]
		\startitemize [n, joinedup]
			\item  Specializing in stochastic analysis and mathematical finance
			\item  Vice President of the LSU SIAM chapter
			\item  Teaching assistant for Capstone Course, Probability Theory, Interest Theory, Linear Algebra, Calculus and Pre-calculus. The capstone course is a upper level undergraduate course in which the aim is to solve real-world problems. Previous projects include understanding kinesiology data using recurrence quantification analysis, building a dashboard for analyzing data for athletes, and analyzing 3D body shapes to get health metrics.
		\stopitemize
	\stopsubject
	\quotation{...{\it why is there something rather than nothing?}
	For nothing is simpler and easier than something.
	Furthermore, assuming that things must exist,
	one must be able to explain
	{\it why they must exist thus,}
	and not otherwise.}
\nextAsymmetricColumns
	G. W. Leibniz, {\it\fr Principes de la nature et de la grâce fondés en raison,} 1714.
\stopAsymmetricColumns
\stoptext