\section{Fundamentals}

\begin{definition}[Category]
    A \textit{category} contains:
    \begin{enumerate}
        \item \textbf{Objects}
        \item \textbf{Arrows} which go between objects.
    \end{enumerate}
\end{definition}

\begin{definition}[Arrow (morphism)]
    Given two objects $A$ and $B$ from a category, an \textit{arrow} (or \textit{morphism}) $f$ goes from one to the other.
    
    \begin{figure}[H]
        \centering
        \begin{tikzpicture}
            \node [phantomblock] (A) {$A$};
            \node [phantomblock, right=2em of A] (B) {$B$};
            \draw [->] (A) edge node [above] {$f$} (B);
        \end{tikzpicture}
        \caption{Arrow starting at $A$ and ending at $B$}
        \label{fig:arrow}
    \end{figure}
\end{definition}

\begin{remark}
    Arrows can be thought of as \textit{functions}: a function taking an input of type \Code{int} and returning some value of type \Code{bool} can be considered as an arrow between $\MCode{int} \to \MCode{bool}$.
\end{remark}

\begin{definition}[Composition of Arrows]
    Arrows (morphisms) can be \textit{composed}. Given the arrows $f$ and $g$ where
    \begin{equation}
        \begin{aligned}
            A \xrightarrow{f} B \xrightarrow{g} C
        \end{aligned}
    \end{equation}
    
    Then their composition $g \circ f$ goes from $A$ to $C$.
    
    \begin{figure}[H]
        \centering
        \begin{tikzcd}
            A \arrow[r, "f", swap] \arrow[rr, bend left, "g \circ f"] 
            & B \arrow[r, "g", swap] 
            & C
        \end{tikzcd}
        \caption{Composition of Arrows}
        \label{fig:arrow-composition}
    \end{figure}
    
    Such composition between two arrows must satisfy the two properties:
    \begin{enumerate}
        \item \textbf{Associativity}.
        For any three composable arrows $f$, $g$, $h$, it is required that
        \begin{equation}
            h \circ (g \circ f) \equiv (h \circ g) \circ f \equiv h \circ g \circ f
        \end{equation}
        Allowing parentheses to be dropped.
        \item \textbf{Identity}. For every object $A$ there exists an \textit{unit of composition}, which is an \textit{identity} arrow $id_A$ which loops back to the same object $A$. When the identity arrow $id_A$ is composed with any arrow $\gamma$ which begins or ends at $A$, then the result is $\gamma$. Given some arrow $f$ where $A \xrightarrow{f} B$ then
        \begin{align}
            f \circ id_A &= f \\
            id_B \circ f &= f
        \end{align}
    \end{enumerate}
    
    Graphically,
    \begin{figure}[H]
        \centering
        \begin{subfigure}[t]{0.45\textwidth}
            \centering
            \begin{tikzcd}
                A \arrow[r, "f", swap] \arrow[rrr, bend left, "h \circ g \circ f"]
                & B \arrow[r, "g", swap]
                & C \arrow[r, "h", swap]
                & D
            \end{tikzcd}
            \caption{Associative Composition}
        \end{subfigure}
        \begin{subfigure}[t]{0.45\textwidth}
            \centering
            \begin{tikzcd}
                A \arrow[r, "f"] \arrow[loop left, "id_A"]
                & B \arrow[loop right, "id_B"]
            \end{tikzcd}
            \caption{Identity Morphisms}
        \end{subfigure}
        \caption{Composition Properties}
        \label{fig:composition-properties}
    \end{figure}
    
\end{definition}

\begin{remark}
    The composition between $f$ and $g$ is denoted $g \circ f$ with $f$ applied first -- the order of composition is \textit{right-to-left}, with the rightmost arrow, $f$, applied first.
    
    In C, this is effectively
    \inputminted{c}{content/code-listings/composition.c}
    
    In Haskell,
    \inputminted{hs}{content/code-listings/composition.hs}
    
    Creating a new composed function from $f$ and $g$ in JavaScript
    \inputminted{js}{content/code-listings/composition.js}
    
    Note that a new anonymous function taking an input \Code{x} is created, which is the composed function $g \circ f$.
    
    The identity morphism in Haskell is simply
    \inputminted{hs}{content/code-listings/identity.hs}
\end{remark}
