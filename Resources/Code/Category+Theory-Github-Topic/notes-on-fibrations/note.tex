\documentclass{article}

\usepackage{amssymb,amsmath,amsthm}
\usepackage{cmll}
\usepackage{stmaryrd}
\usepackage{mathpartir}
\usepackage{supertabular}
\usepackage{color}
\usepackage{fullpage}
\usepackage{verbatim}
\usepackage[textwidth=2cm]{todonotes}
\usepackage{enumitem}
\usepackage{hyperref}

\usepackage[all]{xypic}
\usepackage[barr]{xy}

%% This renames Barr's \to to \mto.  This allows us to use \to for imp
%% and \mto for a inline morphism.
\let\mto\to
\let\to\relax
\newcommand{\to}{\rightarrow}
\newcommand{\ndto}[1]{\to_{#1}}
\newcommand{\ndwedge}[1]{\wedge_{#1}}

% Commands that are useful for writing about type theory and programming language design.
%% \newcommand{\case}[4]{\text{case}\ #1\ \text{of}\ #2\text{.}#3\text{,}#2\text{.}#4}
\newcommand{\interp}[1]{\llbracket #1 \rrbracket}
\newcommand{\normto}[0]{\rightsquigarrow^{!}}
\newcommand{\join}[0]{\downarrow}
\newcommand{\redto}[0]{\rightsquigarrow}
\newcommand{\nat}[0]{\mathbb{N}}
\newcommand{\fun}[2]{\lambda #1.#2}
\newcommand{\CRI}[0]{\text{CR-Norm}}
\newcommand{\CRII}[0]{\text{CR-Pres}}
\newcommand{\CRIII}[0]{\text{CR-Prog}}
\newcommand{\subexp}[0]{\sqsubseteq}
%% Must include \usepackage{mathrsfs} for this to work.

\date{}

\let\b\relax
\let\d\relax
\let\t\relax
\let\r\relax
\let\c\relax
\let\j\relax
\let\wn\relax
\let\H\relax

% Cat commands.
\newcommand{\powerset}[1]{\mathcal{P}(#1)}
\newcommand{\cat}[1]{\mathcal{#1}}
\newcommand{\func}[1]{\mathsf{#1}}
\newcommand{\iso}[0]{\mathsf{iso}}
\newcommand{\H}[0]{\func{H}}
\newcommand{\J}[0]{\func{J}}
\newcommand{\catop}[1]{\cat{#1}^{\mathsf{op}}}
\newcommand{\Hom}[3]{\mathsf{Hom}_{\cat{#1}}(#2,#3)}
\newcommand{\limp}[0]{\multimap}
\newcommand{\colimp}[0]{\multimapdotinv}
\newcommand{\dial}[1]{\mathsf{Dial_{#1}}(\mathsf{Sets^{op}})}
\newcommand{\dialSets}[1]{\mathsf{Dial_{#1}}(\mathsf{Sets})}
\newcommand{\dcSets}[1]{\mathsf{DC_{#1}}(\mathsf{Sets})}
\newcommand{\sets}[0]{\mathsf{Sets}}
\newcommand{\obj}[1]{\mathsf{Obj}(#1)}
\newcommand{\mor}[1]{\mathsf{Mor(#1)}}
\newcommand{\id}[0]{\mathsf{id}}
\newcommand{\lett}[0]{\mathsf{let}\,}
\newcommand{\inn}[0]{\,\mathsf{in}\,}
\newcommand{\cur}[1]{\mathsf{cur}(#1)}
\newcommand{\curi}[1]{\mathsf{cur}^{-1}(#1)}
\newcommand{\m}[1]{\mathsf{m}_{#1}}
\newcommand{\n}[1]{\mathsf{n}_{#1}}
\newcommand{\b}[1]{\mathsf{b}_{#1}}
\newcommand{\d}[1]{\mathsf{d}_{#1}}
\newcommand{\h}[1]{\mathsf{h}_{#1}}
\newcommand{\p}[1]{\mathsf{p}_{#1}}
\newcommand{\q}[1]{\mathsf{q}_{#1}}
\newcommand{\t}[0]{\mathsf{t}}
\newcommand{\r}[1]{\mathsf{r}_{#1}}
\newcommand{\s}[1]{\mathsf{s}_{#1}}
\newcommand{\w}[1]{\mathsf{w}_{#1}}
\newcommand{\c}[1]{\mathsf{c}_{#1}}
\newcommand{\j}[1]{\mathsf{j}_{#1}}
\newcommand{\jinv}[1]{\mathsf{j}^{-1}_{#1}}
\newcommand{\wn}[0]{\mathop{?}}
\newcommand{\codiag}[1]{\bigtriangledown_{#1}}

\newenvironment{changemargin}[2]{%
  \begin{list}{}{%
    \setlength{\topsep}{0pt}%
    \setlength{\leftmargin}{#1}%
    \setlength{\rightmargin}{#2}%
    \setlength{\listparindent}{\parindent}%
    \setlength{\itemindent}{\parindent}%
    \setlength{\parsep}{\parskip}%
  }%
  \item[]}{\end{list}}

\let\diagram\relax
\newenvironment{diagram}{
  \begin{center}
    \begin{math}
      \bfig
}{
      \efig
    \end{math}
  \end{center}
}

\newtheorem{theorem}{Theorem}[section]
\newtheorem{corollary}{Corollary}[theorem]
\newtheorem{lemma}[theorem]{Lemma}
\newtheorem{definition}[theorem]{Definition}

\begin{document}

\title{Notes on Fibrational Semantics of Simple, Polymorphic, and Dependent Type Theory}
\author{Harley Eades III}
\maketitle

\section{The Simple Fibration}
\label{sec:the_simple_fibration}

\begin{definition}
  \label{def:CT-structure}
  A \emph{CT-structure} is a pair $(\mathbb{B},T)$ where $\mathbb{B}$
  is a category with finite products, and $T \subseteq
  \mathsf{Obj}(\mathbb{B})$ is a collection of types.
\end{definition}

A CT-structure $(\mathbb{B},T)$ should be thought of as a category of
contexts $\mathbb{B}$ whose types draw their atomic elements from $T$.
Given contexts $\Gamma,\Delta \in \mathsf{Obj}(\mathbb{B})$, their
concatenation is defined as $(\Gamma,\Delta) = (\Gamma \times
\Delta)$.

\begin{definition}
  \label{def:simple-total-cat}
  ...
\end{definition}


\begin{definition}
  \label{def:simple-fibration}
  ...
\end{definition}

\begin{definition}
  \label{def:lambda1-category}
  The category $\mathcal{L}_1$ is defined as follows:
  \begin{center}
    \begin{tabular}{lll}
      \textbf{Objects:}   & Contexts $\Gamma = x_1:T_1,\ldots,x_n:T_n$.\\
      
      \textbf{Morphisms:} & Let $\Gamma$ and $\Delta =
      y_1:T_1,\ldots,y_n:T_n$ be contexts, then a morphism $\Gamma
      \mto \Delta$ is a $n$-tuple,\\ & $([t_1],\ldots,[t_n])$, such
      that $[t_i] = \{t \mid \text{$t$ differs from $t_i$ only by the
        names of its free variables}\}$ \\ & is the equivalence
      class of terms such that $\Gamma \vdash t_i : T_i$ holds for
      each $1 \leq i \leq n$.  \\
    \end{tabular}
  \end{center}
\end{definition}

\begin{lemma}[Classifying Category for STLC]
  \label{lemma:class-stlc}
  $\mathcal{L}_1$ is indeed a category.
\end{lemma}
\begin{proof}
  \textbf{(Identities)} Suppose $\Gamma = x_1:T_1,\ldots,x_i:T_i$ is a
  context.  Then $\mathsf{id} = (x_1,\ldots,x_i) : \Gamma \mto
  \Gamma$, because $\Gamma \vdash x_j : T_j$ for $1 \leq j \leq i$
  each hold by the variable rules.

  \ \\ \textbf{(Composition.)}  Suppose $\Gamma$, $\Delta =
  y_1:T_1,\ldots,y_i:T_i$ and $\Phi = z_1:T'_1,\ldots,z_j:T'_j$ are
  contexts, and $f = (t_1,\ldots,t_i) : \Gamma \mto \Delta$ and $g =
  (t'_1,\ldots,t'_j) : \Delta \mto \Phi$ are morphisms.  Then define
  their composition $f;g =
  ([t_1/y_1]\cdots[t_i/y_i]t'_1,\ldots,[t_1/y_1]\cdots[t_i/y_i]t'_j):
  \Gamma \mto \Phi$.

  \ \\ \textbf{(Composition respects identities.)}. Suppose $\Gamma =
  x_1:T_1,\ldots,x_i:T_i$ and $\Delta = y_1:T_1,\ldots,y_j:T_j$ are
  contexts, and $f = (t_1,\ldots,t_j) : \Gamma \mto \Delta$ is a
  morphism.  Then:
  \begin{center}
    \begin{math}
      \begin{array}{lll}
        id_{\Gamma};f
        & = & ([x_1/x_1]\cdots[x_i/x_i]t_1,\ldots,[x_1/x_1]\cdots[x_i/x_i]t_j)\\
        & = & (t_1,\ldots,t_j)\\
        & = & f\\
      \end{array}
    \end{math}
  \end{center}
  and
  \begin{center}
    \begin{math}
      \begin{array}{lll}
        f;id_{\Delta}
        & = & ([t_1/y_1]\cdots[t_j/y_j]y_1,\ldots,[t_1/y_1]\cdots[t_j/y_j]y_j)\\
        & = & (t_1,\ldots,t_j)\\
        & = & f\\
      \end{array}
    \end{math}
  \end{center}
\end{proof}

% section the_simple_fibration (end)


\appendix

\lstset{basicstyle=\footnotesize\ttfamily,breaklines=true,breakpages=true}
\def\fileps
  { ../src/Cat.agda
  , ../src/Cat/Categories/Cat.agda
  , ../src/Cat/Categories/Cube.agda
  , ../src/Cat/Categories/CwF.agda
  , ../src/Cat/Categories/Fam.agda
  , ../src/Cat/Categories/Free.agda
  , ../src/Cat/Categories/Fun.agda
  , ../src/Cat/Categories/Rel.agda
  , ../src/Cat/Categories/Sets.agda
  , ../src/Cat/Category.agda
  , ../src/Cat/Category/CartesianClosed.agda
  , ../src/Cat/Category/Exponential.agda
  , ../src/Cat/Category/Functor.agda
  , ../src/Cat/Category/Monad.agda
  , ../src/Cat/Category/Monad/Kleisli.agda
  , ../src/Cat/Category/Monad/Monoidal.agda
  , ../src/Cat/Category/Monad/Voevodsky.agda
  , ../src/Cat/Category/Monoid.agda
  , ../src/Cat/Category/NaturalTransformation.agda
  , ../src/Cat/Category/Product.agda
  , ../src/Cat/Category/Yoneda.agda
  , ../src/Cat/Equivalence.agda
  , ../src/Cat/Prelude.agda
  }

\foreach \filep in \fileps {
\chapter{\filep}
     %% \begin{figure}[htpb]
        \lstinputlisting{\filep}
     %%    \caption{Source code for \texttt{\filep}}
     %% \label{fig:\filep}
   %% \end{figure}
}
%% \lstset{framextopmargin=50pt}
%% \lstinputlisting{../../src/Cat.agda}


\end{document}

%%% Local Variables:
%%% mode: latex
%%% TeX-master: t
%%% End:
