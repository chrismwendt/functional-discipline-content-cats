\documentclass{article}
\usepackage[utf8]{inputenc}
\usepackage[T1]{fontenc}
\title{LambdaConf 2017}
\author{Krystal Maughan }
\date{May 27th 2017}

\begin{document}

\maketitle

\mathversion{bold}{Day Three}
\mathversion{normal}
\section{Lambda Calculus}
The Wizard Towel
\\
\\
Identity Function
$\lambda x . x$
\\
x is the function body
\\
$\lambda x$ is the head
\\
Takes an argument, binds the value to x and returns it
\\
In Haskell :  $\setminus$ x $\rightarrow x$
\\
\\
Typed vs untyped lambda calculus
\\
Alpha Equivalence $\lambda x.x$ $\equiv$ $\lambda y.y$
\\
($\lambda x.x$)2
\\
Simplification $beta$ reduction
\\
\\
1. Bind the argument to the name
\\
2. Strip off the function head and iterator
\\
3. Simplify until we can't anymore.
\\
$\lambda$ 2. 2
\\
Head: $\lambda$ 2
\\
Body: 2 
\\
Identity Simplification: 2
\\
($\lambda$ x . x)($\lambda$ y . y)
\\
\\
Lambda Calculus: Every function can only take one argument
\\
$\lambda$ x.($\lambda$ y. x y)
\\
\\
Converting a function with multiple args $\rightarrow$ currying
\\
\\
See John Carmac porting Haskell talk Wolf 3d
\\
\\
$(\lambda y . ( \lambda z . z) y)$
\\
\\
Types
\\
const :: a $\rightarrow$ b $\rightarrow$ a
\\
take in one value, take in anothe value, and return the first
\\
\\
:k type (kind of type)
\\
Lambda Calculus is a Turing Tarpit
\\
\\
Look up talk on : Lisp Interpreter in AWK
\\
(another turing tarpit $\rightarrow$ awk)
\\
\\
\section{Omega Combinator}
$(\lambda x. x x)(\lambda y . y y)$
\\
\\
Beta Reduction
\\
Alpha Conversion
\\
What we need is the Y combinator
\\
$Y$ = $\lambda f. (\lambda x . f (x x)) (\lambda x. f(x x))$
\\
\\
\section{What are Dependent Types}
Stephan Boyer
\\
\\
The Lambda Cube
\\
\section{Curry-Howard Isomorphism}
Coq
\\
\\
System F: Polymorphism
\\
Functions from types to terms
\\
$\forall X. T$
\\
\\
System $\lambda \omega$
\\
Kind syntax(K): 
\\
functions from * $\rightarrow$ *
\\
System LF: dependent types
\\
$\Pi x : T. K$
\\
\\
Curry-Howard Isomorphism
\\
Type Theory vs Logic
\\
eg Sum Type vs Disjunction
\\
\\
\section{Coq}
GADTs
\\
\\
conj : P $\rightarrow$ Q we have P $\land$ Q
\\
\\
Definition ($\neg$ A $\rightarrow$ False)
\\
0: nat
S: nat $\rightarrow$ nat.
Defined all natural numbers in Coq.
\\
\\
\section{Higher Order Abstractions}
Stephan Boyer
\\
\\
\section{Functor}
supports fmap
\\
Takes function a $\rightarrow$ b fa $\rightarrow$ fb
\\
\\
\section{Bifunctor}
Has two type parameters instead of one
\\
bimap :: (a $\rightarrow$ b) $\rightarrow$ (c $\rightarrow$ d) 
\\
Either is a bifunctor
\\
\\
\section{Contravariant}
Looks like fmap.
\\
take fb and returns fa
\\
Predicate is example of Contravariant
\\
contramap :: (a $\rightarrow$ b) $\rightarrow f b \rightarrow f a$
\\
\\
$a \rightarrow (a \rightarrow Int) \rightarrow Int)$ : Invariant
\\
data Phantom a = Phantom : Bivariant
\\
\\
\section{Profunctor}
Hungry for bs and contains cs
\\
Dimap
\\
Lens, Prism
\\
\\
\section{Haskell Singletons and You}
Justin Le (mstksg.github.io)
\\
Phantom Types
\\
closeDoor :: Door 'Opened $\rightarrow$ Door 'Closed
\\
closeDoor UnsafeMkDoor = UnsafeMkDoor
\\
can only open a close door. Compile type otherwise
\\
type can't depend on your inputs
\\
In Haskell, types only exist at runtime.
\\
They are erased at runtime.
\\
\\
\section{Singleton Pattern}
Only one value type : we can pattern match
\\
showSingDS :: SingDS s $\rightarrow$ String 
\\
   SOpened $\rightarrow$ "Opened"
\\
   SClosed $\rightarrow$ "Closed"
\\
   SLocked $\rightarrow$ "Locked"
\\
\\
Singletons are Poly-kinded typeclasses.
\\
\\
Further Reading: 
\\
https://blog.jle.im/entry/verified-instances-in-haskell.html
\\
\\
\section{The Origins of Free}
Adam Warski
\\
\\
Free Monads:
\\
\\
Separate Description from Interpretation
\\
Programmes as values
\\
\\
\section{Universal Algebra : study of general algebraic structures}
Syntax:
\\
Type names $\rightarrow$ set S
\\
Operation Names: family $\Omega$ of sets indexed by S* $\times$ S
\\
\\
Interpretation of signature 
\\
Function between appropriate sets
\\
$\Sigma$ algebra A:
\\
$succ_{A}$ =$\lambda . \lambda x + 1$
\\
F-Algebra
\\
\\
Term Algebra : built of pure syntax
\\
\\
Algebra: defined inductively
\\
base: all constants are terms
\\
Step: any functions we can apply on previous terms
\\
Base case: [0]
\\
$[[0], [0 + succ], etc]$
\\
\\
\section{Homomorphism}
A Function between algebras
\\
A function between type interpretations such 
\\
that operations are preserved
\\
\\
$\Sigma$-algebra I is initial when any other 
\\
$\Sigma$-algebra A there is exactly
\\
one homomorphism from I to A
\\
\\
$f : T_{\Sigma} \rightarrow A$
\\
$f(0_{T_{\Sigma}}) = 0_{A}$
\\
\\
\section{Initial Algebra}
Algebra: Only one way to interpret a term
\\
no junk: term algebra contains only what's necessary
\\
\\
Terms with Variables 
\\
For any set $X, T_{\Sigma}(X)$ is the term algebra with X
\\
added as constants (but called variables)
\\
\section{Free Algebra}
$\Sigma$-algebra I is free over X when for any other
\\
$\Sigma$-algebra A, any function
\\
$f : X \rightarrow A$
\\
extends uniquely to a homomorphism
\\
\\
Free:
\\
free to interpret in any way
\\
no constraints
\\
free of additional structure
\\
only contains what is necessary
\\
\\
Term Algebras with Equations
\\
some terms need to be combined
\\
Equivalence relation generated by $\phi$: $\equiv$ $_{\phi}$
\\
Homomorphism to other algebra
\\
free object is unique up to an isomorphism
\\
\section{Monads}
Monads return a pure value
\\
Compute what to do next based on previous result (flatMap)
\\
\\
\section{Free Monads}
Signature: pure and Flatmap
\\
Variables: operations (our Domain Specific Language)
\\
\\
\section{Free in Haskell}
data Free f r = Free(f (Free f r)) | Pure r 
\\
\\
\\
\section{Reference}
Foundations of Algebraic Specification and
\\
Formal Software Development
\\
Donald Sannella and Andrzej Tarlecki
\\
\\
\section{Refactoring Recursion}
Harold Carr
\\
\\
Factor recursion out of data with
\\
Foldable, Traversable, Fix
\\
Recursion over data
\\
cata, para(cata++), histo, zygo/mutu(para++)
\\
both: hylo
\\
corecursion: hylo
\\
\\
Fold left 0 is top element
\\
Fold right 0 is bottom of tree
\\
\\
\section{Recursion Schemes}
Sub-Categories
\section{Catamorphism}
catamorphism : means downwards : aka fold
\\
\section{Anamorphism}
Corecursion, meaning upwards : aka unfold
\\
:: b, $\rightarrow$ (a, b)$\rightarrow$  which it operates on b
$\rightarrow$ [a]
\\
\\
Anamorphism : building something up from a seed analogy
\\
\section{Hylomorphism}
Anamorphism, then catamorphism
\\
Composition of catamorphism and anamorphism
\\
hyloL :: (a $\rightarrow c \rightarrow c$)
\\
corecursive codata production
\\
followed by recursive data consumption
\\
run co-algebra first build up list from n to 0
\\
go across list and use catamorphism
\\
eg factorial goes from n to zero, then applies (*)
\\
\section{Fusion/ Deforestation}
Talk on that..later
\\
\section{Paramorphism}
Beside or parallel with
\\
Extension of Catamorphism
\\
paraL :: (a $\rightarrow [a] \rightarrow b \rightarrow b$)
\\
eg Tails library
\\
Sliding windows 
\\
$(slide 2 [1..5])$
\\
$[[1, 2, 3], [2, 3, 4], [3, 4, 5]]$
\\
\\
\section{Apomorphism}
Apo meaning apart
\\
dual of parammorphism
\\
extension of anamorphism
\\
enables short-circuiting traversal
\\
\\
apoL :: ([b] $\rightarrow$ Maybe(a, Either [b] [a]))
\\
\\
\section{Zygomorphism}
Generalization of paramorphism
\\
zygoL :: ($a \rightarrow b \rightarrow b) \rightarrow (a \rightarrow b \rightarrow c \rightarrow c)$
\\
$ b \rightarrow c$
\\
$[a]$
\\
c
\\
eg. Every third element I need to do...
\\
\\
L means List in slides
\\
\\
pmL :: $[Int] \rightarrow [Int]$
\\
pmL = zygoL ($\backslash_ b \rightarrow$ not b)
\\
\section{Histomorphism}
Gives access to previously computed values
\\
(see photo data History a b)
\\
histoL :: ($a \rightarrow History$ a b $\rightarrow b$)
\\
$\rightarrow b \rightarrow [a] \rightarrow b$
\\
\\
works bottom up on structure
\\
Histomorphism useful for dynamic programming
\\
\\
\section{Futurmorphism}
Corecursive dual of histomorphism
\\
(see photo)
\\
\\
\section{Refactoring Recursion out of data}
deriving Foldable
\\
\section{References}
Tim Williams' Recursion Schemes presentation
\\
http://www.timphilipwilliams.com/slides.html
\\
Functional Programming with Banana, Lenses, Envelopes and Barbed Wire
\\
\\
\section{Intuitionistic Logic III: Subschemas and Topologies}
Vlad Patryshev
\\
N is a category of natural numbers
\\
$A^{B}$ category of functors from B to A aka diagrams
\\
$Set^{2}$ category of 2 diagrams in sets - two sets, one function
\\
$S_{0} \rightarrow S_{1} \rightarrow S_{2}$
\\
Temporal Logic for Applications - Leslie Lamport
\\
\\
Ternary Logic: Two-State Sets
\\
"before" and "after"
\\
\\
Subobject Classifier :
\\
(see Slides)
\\
\\
\section{Grothendieck Topology}
J is the closure of truth in $\Omega$
\\
\\
At some point will be true : diamond classification
\\
\\
Groundhog Week : is a groupoud : Boolean Topology
\\
\\
\section{References}
P.T.Johnstone "Topos Theory"
\\
Ceciia Fiori "A First Course in Topos Quantum Theory"
\\
Introduction to Intuitionistic Type Theory
\\
Intuitionistic Logic of Database Schema
\\
\\
\section{I Command you to be Free}
Matt Parsons
\\
\\


\end{document}
