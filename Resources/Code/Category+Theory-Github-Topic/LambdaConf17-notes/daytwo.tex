\documentclass{article}
\usepackage[utf8]{inputenc}

\title{Lambda Conf Notes}
\author{Krystal Maughan }
\date{May 26th 2017}

\begin{document}

\maketitle

\mathversion{bold}{$ LambdaConf Day 2 $}
\\
\\
\mathversion{normal}
\\
\section{Introduction to Coq} 
Gabriel Claramunt
\\
\\
Programming Logic
\\
Type Theory
\\
Proof Assistants
\\
Implementation of the Calculus of Contructions
\\
Thierry Coquand
\\
based on Martin-Lof's Intuitionistic Type Theory
\\
\\
$\frac{hypothesis}{goal}$
\\
$\frac{\gamma}{\alpha}$
\\
\\
Assumption: Goal is in hypothesis
\\
Propositional Logic
\\
split $\rightarrow$ $\frac{\gamma}{\alpha \land \beta}$
\\
\\
Apply
\\
$\frac{H: \alpha \rightarrow \beta}{\beta}$
\\
\\
Cut $\beta$
\\
$\frac{\gamma}{\alpha}$
\\
\\
Unfold
\\
$\frac{\gamma}{\neg \alpha}$
\\
\\
Classical Logic
\\
Constructive tautologies (auto)
\\
\\
\section{Proofs}
A $\rightarrow$ A.
\\
intro Ha.
\\
assumption
\\
\\
$A \rightarrow B \rightarrow A$
\\
intro Ha.
\\
intro Hb.
\\
assumption.
\\
\\
$(A \rightarrow (B \rightarrow C)) \rightarrow (A \rightarrow B) \rightarrow (A \rightarrow C)$.
\\
intros will define three hypothesis values: 
\\
A, B and C.
\\
\\
intros.
\\
apply H.
\\
assumption.
\\
apply H0.
\\
assumption.
\\
\\
$A \rightarrow \neg \neg A$
\\
intros.
\\
unfold not.
\\
intro.
\\
apply H0.
\\
assumption.
\\
\\
\section{Higher Order Predicate Calculus}
$\forall x: U, a$
\\
\section{Curry-Howard Correspondence}
Computer Programs and Mathematics Proofs
\\
Typed Lambda Calculus
\\
Higher Order
\\
Parametric Functions
\\
Dependent Types
\\
Inductive Types
\\
Co-inductive Types
\\
\\
Based on proof, generates code
\\
\\
\section{Mercury: Haskell and Prolog}
Mercury-Alex Chambers
\\
:- (horn) is declaration
\\
main(I01, IO) : a rule
\\
solutions set takes a predicate output 
\\
non deterministic
\\
Advantages:
\\
relational computation
\\
parsing (Prolog is good at parsing)
\\
\\
:- parsemethod(method, list(char), list(char)).
\\
\\
\\
\\
\\
\\
\\
\\
\\
\\
\\
\\
\\
\\
\\
\\
\\
\\
\\
\\
\\
\\
\section{Mighty Morphin Data Types}
David Koontz
\\
\\
$a \rightarrow b$  = morphism
\\
eg. toString :: Int $\rightarrow$ String
\\
\\
\section{Endormorphism}
\\
$a \rightarrow a$
\\
$append :: List$ $a \rightarrow List$  $a$
\\
\\
m a $\rightarrow$ m b
\\
bind :: (a $\rightarrow$ m b) $\rightarrow$ m a $\rightarrow$ m b
\\
\\
\section{Homomorphism}
\\
f a $\rightarrow$ f b
\\
map :: (a $\rightarrow$ b) $\rightarrow$ f a $\rightarrow$ f b
\\
Maybe a $\rightarrow$ Maybe b 
\\
\\
\section{Isomorphism}
\\
Morphism that can go in the other direction
\\
a $\rightarrow$ b 
\\
b $\rightarrow$ a 
\\
\\
\section{Bijection} 
Can have both
\\
\section{Injection}
One to one (unique) but not every Y has to come
\\
from an X
\\
\section{Surjection}
All elements on lhs have an element
\\
that goes to rhs (can be mapped to more than one
\\
on rhs)
\\
\\
\section{Catamorphism}
a $\rightarrow$ b $\rightarrow$ b
\\
fold 
\\
List Int $\rightarrow$ Int
\\
\\
\section{Anamorphism}
Unfold
\\
String $\rightarrow$ List String
\\
\\
\section{Hylomorphism}
Anamorphism followed by a Catamorphism
\\
Unfold followed by a fold
\\
hylo = fold . unfold
\\
\section{Metamorphism}
Catamorphism followed by an Anamorphism
\\
\\
\\
\section{Algebraic Databases}
Ryan Wisnecky
\\
\\
Alebraic way of formalizing databases based on Category Theory
\\
Functor : Homomorphism of functions
\\
\section{Schemas and Instances}
Directed labelled multi-graphs 
\\
and set of equations between paths
\\
\section{Categorical Semantics of Schemas and Instances}
Each Schema denotes a category
\\
Objects are nodes in schema
\\
Each schema denotes a category [[S]]
\\
nodes(S) $\rightarrow$ nodes(T)
\\
edges(S) $\rightarrow$ paths(T)
\\
and S $\rightarrow$ is a functor [[S]] $\rightarrow$ [[T]]
\\
An S-instance of a functor [[S]] $\rightarrow$  Set
\\
\\
A morphism of S-instances I $\rightarrow$ J (right mapping) 
\\
is a natural transformation I $\rightarrow$ J
\\
\\
labelled nulls - Database Theory
\\
$null_{1}$ compared with $null_{2}$
\\
one less than or more than the other
\\
otherwise just null implies one value
\\
for null/missing data in database
\\






\end{document}