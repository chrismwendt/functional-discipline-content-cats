\documentclass{article}
\usepackage[margin=0.7in]{geometry}
\usepackage{amsmath,amsthm,amssymb,xcolor}
\usepackage{enumitem,tikz-cd}

\theoremstyle{definition}
\newtheorem*{excercise}{Excercise}

\begin{document}
\section{Chapter 1}
\section{Functors(pg26)}

\begin{excercise}[1.2.21] %{{{ 1.2.21		Excercise
	As $A \cong $ A', there's a bijection between them in $\mathcal{A} $ i.e.
	$\begin{tikzcd}
		A \arrow{r}{f} & \arrow{l}{f^{-1}}  A'
	\end{tikzcd}$
	with $f \in \mathcal{A}(A,A')$ and $ f^{-1} \in \mathcal{A}(A',A)$.
	Thus, $F(f) \in \mathcal{B}(F(A),F(A'))$ and $ F(f^{-1}) \in \mathcal{B}(F(A'),F(A))$.
	And as $F$ is a functor,
	\[ F(f) \circ F(f^{-1}) = F(f \circ f^{-1}) = F(1_A) = 1_{F(A)} \text{ similarly, }
	F(f^{-1}) \circ F(f) = 1_{F(A')} \]
	Hence, we have an isomorphism, $
	\begin{tikzcd}
		F(A) \arrow{r}{F(f)} & \arrow{l}{F(f^{-1})} A'.
	\end{tikzcd}$
\end{excercise}
\begin{excercise}[1.2.22] %{{{ 1.2.22		Excercise
	Let $a,a' \in A $ with $a \leq a'$, so that there's a morphism, $f:a\rightarrow a'$ in $\mathcal{A}$.\\
	Now, $F:\mathcal{A} \rightarrow \mathcal{B} $ gives $F(a) \xrightarrow{F(f)} F(a')$ i.e. $F(a)\leq F(a')$.
\end{excercise}
\begin{excercise}[1.2.23] %{{{ 1.2.23		Excercise
	\begin{enumerate}[label=(\alph*)]
		\item As $ob(G)=ob(G^{op})$,just need to ensure morphisms.
			\[\forall f \in \mathcal{A}(A,B) ,\; \; f^{-1} \in \mathcal{A}(B,A) \implies
			f \in \mathcal{A}^{op}(B,A) ,\; \; f^{-1} \in \mathcal{A}^{op}(A,B) \]
			Thus, for any two objects $A,B$ , morphisms between them in $\mathcal{A} $
			are also in $\mathcal{A} ^{op}$. More precisely, define functors
			\[ G \xrightarrow{F} G^{op} \text{ and } G^{op} \xrightarrow{F^{-1}} G \text{ as }
			F(f)\mapsto f^{-1} \text{ and } G(f^{-1}) \mapsto f \]
			Hence, $ F^{-1} \circ F : G \rightarrow G \text{ and } F \circ  F^{-1} : G^{op} \rightarrow G^{op}$.
			Giving an isomorphism, $G\cong G^{op}$  in \textbf{CAT}.
		\item Take the monoid, say $M$, consisting $2\times2$ matrices, $\Bigg\{
				I:=\begin{bmatrix}
					1 & 0 \\
					0 & 1
				\end{bmatrix},
				A:=\begin{bmatrix}
					1 & 0 \\
					0 & 0
			\end{bmatrix}\Bigg\}$ under matrix multiplication.
			So, $M$ has two morphisms, the identity morphism, $i$ and and $a$.
			Now, suppose $M \cong M^{op} $, then, $a$ must have an inverse i.e.
			$ k:= \in M : k.a=i $ i.e. a matrix $K: KA = I$ but this can't be as $A$ is a singular matrix.
	\end{enumerate}
\end{excercise}
\begin{excercise}[1.2.25] %{{{ 1.2.25		Excercise
		\begin{enumerate}[label=(\alph*)]
			\item Let $A \in \mathcal{A} $ and $F^A$ be defined as mentioned.
				For unitality, as $F^A(B)=F(A,B)$  \[
				B \in \mathcal{B} \implies F^A(1_B)=F(1_A,1_B)=1_{F_{(A,B)}}=1_{F^A(B)}\]
				And for composability, let $f\in \mathcal{B}(X,Y)$ and
				$ g \in \mathcal{B}(Y,Z) $,
				\[F^A(f)\circ F^A(g)=(1_A,f)\circ(1_A,g)=(1_A,f\circ g)=F^A(f\circ g)\]
				Hence, $F^A$ is a functor. To prove $F_B$ a functor, \\
				For unitality, as $F_B(A)=F(A,B)$  \[
				A\in\mathcal{A}\implies F_B(1_A)=F(1_A,1_B)=1_{F_{(A,B)}}=1_{F_B(A)}\]
				And for composability, let $f\in \mathcal{A}(X,Y)$ and
				$ g \in \mathcal{A}(Y,Z) $,
				\[F_B(f) \circ  F_B(g) = (f,1_B) \circ (g,1_B) = (f \circ g,1_B) =F_B(f \circ g) \]
			\item Let  $A \in \mathcal{A} $ and $B \in \mathcal{B} $, then,
				$ F^A (B) = F(A,B) = F_B(A) $.
				For $f \in \mathcal{A}(A,A')$ \text{ and } $g \in\mathcal{B}(B,B')$,
				\[ F^{A'}(g) \circ F_B(f) = F(1_{A'},g) \circ F(f,1_B)
					=F(1_{A'} \circ  f , g \circ 1_B)
					=F(f \circ 1_A,1_{B'} \circ g)
				=F_{B'}(f) \circ  F^A(g)\]
			\item As $\mathcal{C}$ is a categeory,
				Define functor $F:\mathcal{A}\times\mathcal{B}\rightarrow\mathcal{C}$
				as
				\[ \forall (A,B) \in \mathcal{A} \times \mathcal{B},\; F((A,B)):=F^A(B)=F_B(A)\]
				And, for any morphism in the product category,
				$(A,B)\xrightarrow{(f,g)} (A',B')$ ,
				$F((f,g)) := F^{A'}(g) \circ F_B(f)$
				Now, This functor exists, as
				\[F((1_A,1_B))=F^A(1_B) \circ F_B(1_A)=F(1_A,1_B) \circ F(1_A,1_B) = F(1_A,1_B)= 1_{(A,B)} \]
				(But this above equation is wrong, as don't yet have the existence of F)
				Need to somehow show that
				\[ F(1_A,1_B)=1_{(A,B)} \]
				and
				\[ F(g \circ f) = F(g) \circ F(f) \]
		\end{enumerate}
\end{excercise}
\begin{excercise}[1.2.26] %{{{ 1.2.26		Excercise
	\textcolor{blue}{need to read topology}
\end{excercise}
\begin{excercise}[1.2.27] %{{{ 1.2.27		Excercise
	Hom-set of each pair of objects must map injectively, but that condition need not hold for the objects themselves.
	So, consider a category, $\mathcal{A}$ with objects $A,A',B,B'$, and morphisms $f_1:A\rightarrow A'$
	$f_2:B\rightarrow B'$. And a category, $\mathcal{B} $ with objects $K,L$ and morphism $g:K\rightarrow L$. And define F as taking $f_1,f_2$ to $g$.
	\[ \begin{tikzcd}
		A \arrow[rr, "f"] \arrow[rd, dotted, maps to, bend right=49] & {} \arrow[r] \arrow[rd, "F", dashed] & A' \arrow[rd, dotted, maps to, bend left=49] &   \\
& K \arrow[rr, "g" description]        & {}                                          & L \\
		B \arrow[rr, "e"] \arrow[ru, dotted, maps to, bend left=49]  & {} \arrow[ru, "F", dashed]           & B' \arrow[ru, dotted,maps to , bend right=49]         &
	\end{tikzcd}\]
	This functor is faithful as every hom-set has at most one non-identity morphism. And the  functor  doesn't map any of those to identity.

\end{excercise}
\begin{excercise}[1.2.28] %{{{ 1.2.28		Excercise
	\begin{enumerate}[label=(\alph*)]
		\item \textcolor{blue}{To do 3}
		\item Identity functor from any category to itself will be full and faithful, as it'll map each morphism only to itself(injectiveness) and it'll so map every morphism (surjectiveness).
Also, $F$ from excercise 1.2.27 is both full and faithful. \\
The following digram describes a functor $F$ from $\mathcal{A} $ containing $A,B$ and $f$ to $\mathcal{B} $ containing $K,L$ and $g,h$ with $F(f) = g$. But, as $h$ isn't in $Im(F)$, $F$ isn't full. But it's faithful as it doesn't map $f$ to identity.
\[
\begin{tikzcd}
A \arrow[rr, "f"] \arrow[d, dotted, maps to]                          & {} \arrow[r] \arrow[d, "F", dashed] & B \arrow[d, dotted, maps to] \\
K \arrow[rr, "g" description] \arrow[rr, "h" description, bend right] & {}                                                      & L
\end{tikzcd}\]
For the same categories, take $G:\mathcal{B} \rightarrow \mathcal{A} $ as $K \mapsto A\; ; \; L \mapsto B $
and $g,h \mapsto f $. Gives G as a non-faithful but full functor. For a functor that's neither full nor faithful, take $H: S_n \rightarrow S_n$ defined as taking all homomorphisms to the identity map.

	\end{enumerate}
\end{excercise}


\begin{excercise}[1.2.29] %{{{ 1.2.29		Excercise
		\item \textcolor{blue}{To do 1}
\end{excercise}

\section{Natural Transformation (pg38)}

\begin{excercise}[1.3.26] %{{{ 1.3.26		Excercise
	\textcolor{blue}{Done but gotta write}\\
	If $ $ is a natural isomorphism, then

\end{excercise}

\begin{excercise}[1.3.27] %{{{ 1.3.27		Excercise
Need to find functors $F,G$  such that
$[\mathcal{A}^{op},\mathcal{B}^{op}] \overset{F}{\underset{G}{\rightleftharpoons}}[ \mathcal{A} ,\mathcal{B} ]^{op} $.

\end{excercise}

\begin{excercise}[1.3.28] %{{{ 1.3.28		Excercise
	What does canonical function even mean ??

		\begin{enumerate}[label=(\alph*)]
			\item
			\item
		\end{enumerate}

\end{excercise}

\begin{excercise}[1.3.29] %{{{ 1.3.29		Excercise	It'll be hard man, but thats what you're here for no ?


\end{excercise}

\begin{excercise}[1.3.30] %{{{ 1.3.30		Excercise

\end{excercise}

\begin{excercise}[1.3.31] %{{{ 1.3.31		Excercise

\end{excercise}

\begin{excercise}[1.3.32] %{{{ 1.3.32		Excercise

\end{excercise}

\begin{excercise}[1.3.33] %{{{ 1.3.33		Excercise

\end{excercise}

\begin{excercise}[1.3.34] %{{{ 1.3.34		Excercise

\end{excercise}




\pagebreak
TBD


\end{document}
