\chapter{Conclusion}
This thesis highlighted some issues with the standard inductive
definition of propositional equality used in Agda.  Functional
extensionality and univalence are examples of two propositions not
admissible in Intensional Type Theory (ITT).  This has a big impact on
what is provable and the reusability of proofs.  This issue is
overcome with an extension to Agda's type system called Cubical Agda.
With Cubical Agda both functional extensionality and univalence are
admissible.  Cubical Agda is more expressive, but there are certain
issues that arise that are not present in standard Agda.  For one
thing Agda enjoys Uniqueness of Identity Proofs (UIP) though a flag
exists to turn this off.  This feature is not present in Cubical Agda.
Rather than having unique identity proofs cubical Agda gives rise to a
hierarchy of types with increasing \nomen{homotopical
  structure}{homotopy levels}.  It turns out to be useful to build the
formalization with this hierarchy in mind as it can simplify proofs
considerably.  Another issue one must overcome in Cubical Agda is when
a type has a field whose type depends on a previous field.  In this
case paths between such types will be heterogeneous paths.  In
practice it turns out to be considerably more difficult to work with
heterogeneous paths than with homogeneous paths.  This thesis
demonstrated the application of some techniques to overcome these
difficulties, such as based path induction.

This thesis formalizes some of the core concepts from category theory
including: categories, functors, products, exponentials, Cartesian
closed categories, natural transformations, the yoneda embedding,
monads and more.  Category theory is an interesting case study for the
application of cubical Agda for two reasons in particular. One reason
is because category theory is the study of abstract algebra of
functions, meaning that functional extensionality is particularly
relevant.  Another reason is that in category theory it is commonplace
to identify isomorphic structures.  Univalence allows for making this
notion precise.  This thesis also demonstrated another technique that
is common in category theory; namely to define categories to prove
properties of other structures.  Specifically a category was defined
to demonstrate that any two product objects in a category are
isomorphic.  Furthermore the thesis showed two formulations of monads
and proved that they indeed are equivalent: Namely monads in the
monoidal- and Kleisli- form.  The monoidal formulation is more typical
to category theoretic formulations and the Kleisli formulation will be
more familiar to functional programmers.  It would have been very
difficult to make a similar proof with setoids and the proof would be
very difficult to read.  In the formulation we also saw how paths can
be used to extract functions.  A path between two types induce an
isomorphism between the two types.  This e.g.\ permits developers to
write a monad instance for a given type using the Kleisli formulation.
By transporting along the path between the monoidal- and Kleisli-
formulation one can reuse all the operations and results shown for
monoidal- monads in the context of Kleisli monads.
%%
%% problem with inductive type
%% overcome with cubical
%% the path type
%% homotopy levels
%% depdendent paths
%%
%% category theory
%% algebra of functions ~ funExt
%% identify isomorphic types ~ univalence
%% using categories to prove properties
%% computational properties
%% reusability, compositional
