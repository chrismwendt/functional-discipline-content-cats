%% Copyright (C) 1990-2009 LaTeX3 project
%%
%% It may be distributed and/or modified under the conditions of the
%% LaTeX Project Public License (LPPL), either version 1.3c of this
%% license or (at your option) any later version.  The latest version
%% of this license is in the file
%%
%%    http://www.latex-project.org/lppl.txt
%%
%% This file is part of the ``expl3 bundle'' (The Work in LPPL)
%% and all files in that bundle must be distributed together.
%%
%% The released version of this bundle is available from CTAN.
%%
%% -----------------------------------------------------------------------
%%
%% The development version of the bundle can be found at
%%
%%    http://www.latex-project.org/cgi-bin/cvsweb.cgi/
%%
%% for those people who are interested.
%%
%%%%%%%%%%%
%% NOTE: %%
%%%%%%%%%%%
%%
%%   Snapshots taken from the repository represent work in progress and may
%%   not work or may contain conflicting material!  We therefore ask
%%   people _not_ to put them into distributions, archives, etc. without
%%   prior consultation with the LaTeX Project Team.
%%
%% -----------------------------------------------------------------------

\documentclass{ltugboat}
\makeatletter
\def\acro#1{\scalebox{1.1}{\fontsize{8pt}{8pt}\selectfont #1}}
\def\pkg#1{\textsf{#1}}

\usepackage[T1]{fontenc}
\usepackage{array,calc,graphicx,listings,lmodern,microtype,multicol,shortvrb,url,xcolor}
\MakeShortVerb{|}

\lstset{basicstyle=\small\ttfamily,columns=flexible}

\title{Getting started with \LaTeX3}
\author{Will Robertson \&c.}
\netaddress{will.robertson@latex-project.org}
\address{}
\AtBeginDocument{\maketitle}
\AtEndDocument{\makesignature}

\begin{document}

\begin{abstract}

The \LaTeX3 project now has functional code for macro writing on \acro{CTAN}
and included in the major \TeX\ distributions. This article is a high-level
introduction to the code so that macros writers can begin to adopt this new
system.

\end{abstract}

\section{Introduction}
\label{sec:intro}

The \LaTeX3 project was initiated many years ago as a way to consider future
improvements of the then-current \LaTeX2.09 and later \LaTeXe\ systems. There
are three broad aspects to the project: to support the current version of
\LaTeX, to build a macro programming model to improve upon \LaTeXe, and to
develop new user-level interfaces for a \LaTeXe\ successor.

The \pkg{expl3} package is a suite of modules that define a new \LaTeX\
programming model. Over the years, many macro writers have been dissatisfied
with the tools (or lack thereof) provided by \LaTeXe, and rolled their own
layers above it. There are a number of packages on \acro{CTAN} that have their
own `spin-off' programming interface that are generally simple extensions to
the few features provided by \LaTeXe\ itself; the most modern and perhaps
popular of these is Philipp Lehman's \pkg{etoolbox} package.

Despite the lack of visible progress, \pkg{expl3} has been maturing now for
years and provides a flexible and robust platform for macro programming. It
provides hooks so that it can be used with \LaTeXe\ along with existing
documents and packages. This article shows how to get started with writing
code using the \pkg{expl3} codebase.

Note that this article does not discuss the more experimental components of
the \LaTeX3 project known as the `\pkg{xpackages}', which are more concerned
with the user interface of \LaTeX3 itself.

\section{The files}

The official release for the \pkg{expl3} package is located on \acro{CTAN}:
\url{http://tug.ctan.org/pkg/expl3}. Along with the code itself, two
documentation files are linked from there: \url{expl3.pdf} and
\url{source3.pdf}. The former is an introduction to philosophy of the
\pkg{expl3} package and full overview of the programming modules available;
the latter is the full documentation for the \pkg{expl3} package. Skimming
through \url{source3.pdf} gives a glimpse of the sheer scope of the
programming interface.

The complete \pkg{expl3} bundle provides the list of packages shown in
Table~\ref{tab:expl3-pkg}. While there is hardly scope to cover each of these
packages, the general theme of this collection will hopefully be made clear in
the course of this article.

\begin{table}
\caption{Packages provided by \pkg{expl3}; emphasised \emph{\pkg{l3doc}} is
the class used to document \pkg{expl3}-related packages.}
\label{tab:expl3-pkg}
\begin{multicols}{4}
\small\sffamily\centering
\begin{obeylines}
l3basics
l3box
l3calc
l3chk
l3clist
\emph{l3doc}
l3expan
l3int
l3io
l3keys
l3messages
l3names
l3num
l3precom
l3prg
l3prop
l3quark
l3seq
l3skip
l3tl
l3token
l3toks
l3xref
\end{obeylines}
\end{multicols}
\end{table}

\section{Starting off}

Most \LaTeX\ users will be familiar with the standard `name-spacing' technique
of hiding non-user macros with the |@| sign, which is set up when reading
package and class files to behave like a letter so that macros such as
|\@secondoftwo| can be defined without possibility of a document author
accidentally calling that macro.

The \pkg{expl3} bundle extends this concept to the characters |_| and |:|,
setting the scene for macros such as |\use_ii:nn| that look rather
unfamiliar to the typical \LaTeX\ user. (That particular macro is identical in
meaning to the preceding |\@secondoftwo|.)

Unlike \LaTeX\ coding practise, where |@| is used fairly haphazardly in
internal package names, the philosophy around the \pkg{expl3} bundle is
consistency and clarity. The |_| symbol is used to separate the macro name
into logical chunks, and the |:| symbol is used to delimit the \emph{name} of
the macro with the definition of its \emph{arguments}.

Here's how a package that uses the \pkg{expl3} bundle might begin:
\begin{lstlisting}
\RequirePackage{expl3}
\ProvidesExplPackage{test-expl3} {2008/08/22}
  {v0.1} {A test package for expl3}
\end{lstlisting}
Note the new command for declaring the \pkg{expl3}--based package; this is the
important command that sets up the package for the new syntax. There's an
equivalent command for declaring an \pkg{expl3}--based class, as well.

If the \pkg{expl3} bundle has been loaded but you're \emph{not} inside a class
or package that's set up for the new syntax, it is possible to locally
activate it for a block of code by wrapping it in
|\Expl|\-|Syntax|\-|On|/|\Expl|\-|Syntax|\-|Off|. These are
analogous to \LaTeXe's |\make|\-|at|\-|letter|/|\make|\-|at|\-|other| pair,
albeit with much more obvious names.\footnote{How many times have we all
answered the question `why \cs{makeatletter}?'!}

And now we're ready to start writing code! I guess it's time to introduce the
new programming syntax, then.

\section{The programming syntax}

Two of the biggest problems with programming in \LaTeXe\ are the fussiness of
how to deal with whitespace and the lack of structure for dealing with
advanced macro expansion (you know, all of that |\expandafter| stuff).

I bring up whitespace early because there's good news that's quick to
describe: you don't need to worry about it any more. Spaces in the input for
\pkg{expl3} code are ignored, which dramatically improves the readability and
maintenance of code like this:
\begin{lstlisting}
\cs_new:Nn \mymacro:n {
  \tl_if_blank:nTF {#1} {
    [empty]
  }{
    #1
  }
}
\end{lstlisting}
Here, we define |\mymacro|, a macro that expands to `[empty]' if given an empty
argument and expands to its argument otherwise.

There are two functions in this code that will look unfamiliar although their
meaning is clear from their context. Low level macros are defined in
\LaTeX\ with commands like |\def| and |\newcommand|, but in \pkg{expl3} we
have renamed things a little. In this case, |\cs_new:Nn| is what you will be
used to reading as |\newcommand| (with a somewhat completely different syntax);
we'll get into the reasons behind this change in section~\ref{sec:funcs}.

What about |\tl_if_blank:nTF|\,? Well, the meaning seems fairly obvious; if
its argument is blank (i.e., either empty or just whitespace) then it does one
thing; otherwise, another. But what about that |:nTF| suffix? These letters
act as \emph{argument specifiers}, coding the behaviour of the function into
the name of the function; the argument to test can be a list of tokens, which
is denoted by |n|, and the true and false branches to execute are denoted by
|T| and |F| respectively.

These `argument specifiers' require some further explanation, and full detail
is given in |expl3.pdf|. Some brief examples can help demonstrate the idea.
With a macro like |\tl_reverse:n|, we know simply from the macro name that
it takes a single argument; |\tl_reverse:n{abc}|, which
expands to |cba|.

Another example is |\tl_if_eq:nnTF| which expands to the |T| argument if
the first two arguments are the same and expands to the |F| argument
otherwise.

So |n| refers to a `normal' argument that is surrounded by braces;
other argument types include |N| which refers to a single token only (no
braces), and |x| which passes the argument to the macro only after fully
expanding it first. An example of this |x| argument is the analogue to the
familiar |\edef| command which now looks like |\cs_set:Npx|.

\section{The naming of things}

The \pkg{expl3} bundle has strict guidelines on the naming of variables and
functions. Since \TeX\ is such a primitive programming environment, we need
all the help we can get to ensure that the code we write doesn't hide implicit
errors in the absence of type-checking.

So when we are dealing with integers, we suffix their name with |_int|; with
comma-separated lists, |_clist|, even though internally it's just a specialised
list of tokens. Global variables start with |g_| and local variables start
with |l_|.

None of these formalisations is checked (usually) by the code itself; it's
simply a crutch to help us, the programmers. To steal a quote from Joel
Spolsky:\footnote{\url{http://www.joelonsoftware.com/articles/Wrong.html}}
`it increases collocation in code, which makes the code easier to
read, write, debug, and maintain, and, most importantly, it makes wrong code
look wrong'. He was writing about Hungarian notation, but the same concept
applies here. When you start writing higher-level abstractions on top of the
basic data types, such as |_clist| and |_prop|, it's the semantics of the
variables that are important, and this is what's shown in the variable name.

\section{Basic macro wrangling}

Having introduced why things looks weird, we can now show some examples of
these macros and things begin to make more sense. In this section and the next
we'll look at defining macros and using booleans and predicates for
conditional processing. An introduction to programming in \pkg{expl3}, in
brief.


\subsection{Defining functions}
\label{sec:funcs}

The \pkg{expl3} codebase provides a suite of commands for defining macros at a
very low level. Note that these are not intended to provide \emph{user-level}
interfaces to creating commands with optional arguments and other
bells'n'whistles. (The \pkg{xparse} package is designed for that, which we
won't be talking about further in this article.) In fact, since these commands
are centred around macros that take arguments, we're going to start calling
them `functions'.

Here are two equivalent ways of defining the function |\foo:nn|:
\begin{lstlisting}
\cs_set:Npn \foo:nn  #1#2 {(#1)/(#2)}
\cs_set:cpn {foo:nn} #1#2 {(#1)/(#2)}
\end{lstlisting}
These macros are respectively equivalent to \TeX's
\begin{quote}
|\long\def\foo:nn| , and\\
|\expandafter\long\expandafter\def|\\
|   \csname foo:nn\endcsname| .
\end{quote}
Note the |:nn| suffix to denote that |\foo| takes two arguments.

Related commands (with easy to guess names) are defined for defining commands
that are global, or aren't long, or are protected, or any combination of
the above. Again, remember that these commands are designed to be
understandable because they're used in low-level code; their verbosity is
intended. User macros can be as succinct as you wish.

\pkg{expl3}'s equivalent commands for \TeX's |\let| are a good demonstration
of the `function suffix' ideas. The following four functions each set the
macro |\foo| to have the same meaning as |\bar|:

\begin{lstlisting}
\cs_set_eq:NN \foo  \bar
\cs_set_eq:cN {foo} \bar
\cs_set_eq:Nc \foo  {bar}
\cs_set_eq:cc {foo} {bar}
\end{lstlisting}

The real strength of the \pkg{expl3} programming language is the separation
between functions that perform actual work from the mechanics of the
underlying macro expansion.

\section{True or false?}

Anyone who's experimented with complicated boolean nesting in \LaTeXe\ will be
familiar with the problem of the error `|missing \fi inserted|' and the like,
when the result of the conditional statement is expanded within another
|\if...\fi| statement. To avoid these problems, the \pkg{expl3} bundle takes a
different approach than the classical |\newif| approach. As we will see, this
results in a very rich framework for boolean logic that \LaTeX\ has typically
missed in the past.

\begin{lstlisting}
\bool_if:nTF{\c_false}{yes}{no}
\end{lstlisting}

\section{Lists of tokens and lists of lists of tokens}

In \TeX\ and \LaTeXe, values can be stored either in toks or macros.
The later is often preferable, but means that functions (macros which do
something) and storage (macros holding text) are intermixed. In
\textsf{expl3}, the data type "tl" (Token List variable) is used for
storing text in a macro. This means that the code immediately shows when
a storage area is being created.

\section{More examples of expansion}

It is quite common to want to keep all definitions local, so that
nesting of macros and localised effects can be accommodated.  This can
make temporarily altering definitions of other macros awkward:
\begin{lstlisting}
\def\MyMacro#1{%
  \begingroup
    \let\SomeOtherMacro\relax
    \edef\NewMacro{#1}%
    \expandafter\endgroup\expandafter
      \def\expandafter\NewMacro\expandafter
        {\NewMacro}}
\end{lstlisting}
This keeps the definition of |\NewMacro| local to any group that |\MyMacro|
is used in, but is hardly very clear.  The same in \pkg{expl3}:
\begin{lstlisting}
\cs_new:Npn \MyMacro #1 {
    \group_begin:
      \cs_set_eq:NN \SomeOtherMacro \scan_stop:
      \tl_set:Nx \NewMacro {#1}
  \exp_args:NNNo
    \group_end:
    \tl_set:Nn \NewMacro {\NewMacro}
}
\end{lstlisting}
requires only a single function (|\exp_args:NNNo|) to achieve the effect.

\section{Lua\TeX}

So what does it mean to be talking about a new macro programming interface
when Lua\TeX\ is just around the corner? The Con\TeX{}t project has been being
re-written in sync with Lua\TeX\ development, but the \LaTeX\ project is not
afforded that flexibility.

Consider that Con\TeX{}t has been evolving as a monolithic software project
for many years. Its underlying architecture is built on solid foundations, and
this makes it conceptually much more possible to switch out areas of
processing with Lua-based alternatives.

Despite the great advances over the years in scope and features of external
packages available, \LaTeX\ has no such structure. It has remained with a
piecemeal kernel for many years while the \pkg{expl3} programming syntax has
been developed, since there's no point fixing what will be thrown out. Once
complete, however, \pkg{expl3} gives \LaTeX\ a structured base that can be
re-written in Lua as needed. For example, the quicksort algorithm in
\pkg{l3prg}, which is currently written with \TeX\ code.

\end{document}
