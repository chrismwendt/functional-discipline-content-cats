% \iffalse meta-comment
%
% Copyright (C) 1993-2020
% The LaTeX3 Project and any individual authors listed elsewhere
% in this file.
%
% This file is part of the LaTeX base system.
% -------------------------------------------
%
% It may be distributed and/or modified under the
% conditions of the LaTeX Project Public License, either version 1.3c
% of this license or (at your option) any later version.
% The latest version of this license is in
%    http://www.latex-project.org/lppl.txt
% and version 1.3c or later is part of all distributions of LaTeX
% version 2008 or later.
%
% This file has the LPPL maintenance status "maintained".
%
% The list of all files belonging to the LaTeX base distribution is
% given in the file `manifest.txt'. See also `legal.txt' for additional
% information.
%
% The list of derived (unpacked) files belonging to the distribution
% and covered by LPPL is defined by the unpacking scripts (with
% extension .ins) which are part of the distribution.
%
% \fi
\documentclass[a4paper]{article}

% Commands and environments:

\newcommand{\eg}{e.g.~}
\newcommand{\ie}{i.e.~}


\newenvironment{citations}{%
   \list{}{%
      \renewcommand{\makelabel}[1]{\normalfont\itshape ##1}%
   }%
}{%
   \endlist
}

% Some logos:

\newcommand{\NFSS}{NFSS}

\newcommand{\LaTeXNews}{\LaTeX~News}

\newcommand{\AW}{Addison Wesley}

\newcommand{\SGML}{{\sc SGML}}
\newcommand{\DSSSL}{{\sc DSSSL}}
\newcommand{\HTML}{{\sc HTML}}
\newcommand{\XML}{{\sc XML}}
\newcommand{\PDF}{{\sc PDF}}

\newcommand{\PS}{{\sc Post\-Script}}

\newcommand{\AmSTeX}{$\mathcal A$\lower.4ex\hbox{$\!\mathcal
                                            M\!$}$\mathcal S$-\TeX}
\newcommand{\AmSLaTeX}{$\mathcal A$\lower.4ex\hbox{$\!\mathcal
                                              M\!$}$\mathcal S$-\LaTeX}
\newcommand{\MF}{{\sc Meta\-font}}

% sections

\setcounter{secnumdepth}{-2}

% Front matter:

\title{\Large The \LaTeX3 Project}

\author{\copyright 1995--1999 \
Frank Mittelbach and
Chris Rowley}

\date{12 January 1999}

\begin{document}

\thispagestyle{empty}

\maketitle

\begin{abstract}
  This article describes the motivation, achievements and future of
  the \LaTeX3 Project, which was established to produce a new version
  of \LaTeX{}, the widely-used and highly-acclaimed document
  preparation system.  It also describes how you can help us to
  achieve our aims.\\[2pt]
  \textbf{For Archive maintainers, Authors, Publishers and
          Distributors:}\\
  The project team request that, whenever possible, you include this
  article in any of the following:
  \begin{itemize}
  \item Books about \TeX{} and \LaTeX{}.
  \item Instructions for authors on using \LaTeX{}.
  \item The printed documentation of CD-ROM collections that contain
    \LaTeX.
  \item On-line collections that include a significant proportion of
    documents encoded in \LaTeX.
\end{itemize}
\end{abstract}

\section{Outline}

The purposes of the \LaTeX3 system can be summarized thus: it will
greatly increase the range of documents which can be processed; and it
will provide a flexible interface for typographic designers to easily
specify the formatting of a class of documents.

The \LaTeX3 Project Team is a small group of volunteers whose aim is
to produce this major new document processing system based on the
principles pioneered by Leslie Lamport in the current \LaTeX.

The major visible work of the team before 1997 was the development of
the current \emph{standard} version of \LaTeX{}.  This was first
released in 1994 and has since then been actively maintained and
enhanced by extensions to that core system.  They will continue to
develop and maintain this system, releasing updated versions every six
months and recording these activities in the \LaTeX{} bugs database
(see below).

Although \LaTeX{} may be distributed freely, the production and
maintenance of the system does require expenditure of reasonably large
sums of money.  The \LaTeX3 Project Fund has therefore been set up to
channel money into this work.  We know that some users are
aware of this fund as they have already contributed to it---many
thanks to all of them!  If you want to know more about how you can
help the project, see Page \pageref{fund}---and thanks in advance for
your generosity in the future.


\section{Background}

With \TeX{}, Knuth designed a formatting system that is able to
produce a large range of documents typeset to extremely high quality
standards. For various reasons (\eg quality, portability, stability
and availability) \TeX{} spread very rapidly and can nowadays be best
described as a world-wide de facto standard for high quality
typesetting. Its use is particularly common in specialized areas, such
as technical documents of various kinds, and for multi-lingual
requirements.

The \TeX{} system is fully programmable. This allows the development
of high-level user interfaces whose input is processed by \TeX{}'s
interpreter to produce low-level typesetting instructions; these are
input to \TeX{}'s typesetting engine which outputs the format of each
page in a device-independent page-description language.  The \LaTeX{}
system is such an interface; it was designed to support the needs of
long documents such as textbooks and manuals. It separates content and
form as much as possible by providing the user with a generic (\ie
logical rather than visual) mark-up interface; this is combined with
style sheets which specify the formatting.

Recent years have shown that the concepts and approach of \LaTeX{} are
now widely accepted. Indeed, \LaTeX{} has become the standard method
of communicating and publishing documents in many academic
disciplines.  This has led to many publishers accepting \LaTeX{}
source for articles and books; and the American Mathematical Society
now provides a \LaTeX{} package making the features of \AmSTeX{}
available to all users of \LaTeX{}.  Its use has also spread into many
other commercial and industrial environments, where the technical
qualities of \TeX{} together with the concepts of \LaTeX{} are
considered a powerful combination of great importance to such areas as
corporate documentation and publishing.  This has also extended to
on-line publishing using, for example, \PDF{} output incorporating
hypertext and other active areas.

With the spreading use of \SGML{}-compliant systems (\eg Web-based
publishing using \HTML{} or \XML{}) \TeX{} again is a common choice as
the formatting engine for high quality typeset output: a widely used
such system is {\em The Publisher\/} from ArborText, whilst a more
recent development is the object-oriented document editor Grif.  The
latter is used for document processing in a wide range of industrial
applications; it has also been adopted by the Euromath consortium as
the basis of their mathematician's workbench, one of the most advanced
of the emerging project-oriented user environments.  Typeset output
from \SGML{}-coded documents in these systems is obtained by
translation into \LaTeX{}, which will therefore soon also be a natural
choice for the output of \DSSSL-compliant systems.

Because a typical \SGML{} Document Type Definition (DTD) uses concepts
similar to those of \LaTeX{}, the formatting is often implemented by
simply mapping document elements to \LaTeX{} constructs rather than
directly to `raw \TeX'.
This enables the
sophisticated analytical techniques built into the \LaTeX{}
software to be exploited; and it avoids the need to program in \TeX{}.

\section{Motivation}

This increase in the range of applications of \LaTeX{} has highlighted
certain limitations of the current system, both for authors of
documents and for designers of formatting styles.

In addition to the need to extend the variety of classes of document
which can be processed by \LaTeX{}, substantial enhancements are
necessary in, at least, the following areas:
 \begin{itemize}
 \item
   the command syntax (attributes, short references, etc);
 \item
   the layout specification interface (style design);
 \item
   the level of robustness (error recovery, omitted tags);
 \item
   the extendibility (package interface);
 \item
   the layout specification of tabular material;
 \item
   the specification and inclusion of graphical material;
 \item
   the positioning of floating material, and other aspects of page
   layout;
 \item
   the requirements of hypertext systems.
 \end{itemize}

 Further analysis of these deficiencies has shown that some of the
 problems are to be found in \LaTeX{}'s internal concepts and design.
 This project to produce a new version therefore involves thorough
 research into the challenges posed by new applications and by the use
 of \LaTeX{} as a formatter for a wide range of documents, \eg \SGML{}
 documents; on-line \PDF{} documents with hypertext links.

 This will result in a major re-implementation of large parts of the
 system.  Some of the results of such rethinking of the fundamentals
 are already available in Standard \LaTeX{}, notably in the following
 areas:

\begin{itemize}
\item Font declaration and selection;
\item Font and glyph handling within mathematical formulas;
\item Handling multiple font glyph encodings within a document;
\item Allowing multiple input character encodings within a document;
\item A uniform interface for graphics inclusion;
\item Support for coloured text;
\item Building and interfacing new classes and extension packages.
\end{itemize}


\section{Description}

The strengths of the present version of \LaTeX{} are
as follows:
 \begin{itemize}
 \item excellent standard of typesetting for text, technical
formulas\\
and tabular material;
 \item  separation of generic mark-up from visual formatting;
 \item  ease of use for authors;
 \item  portability of documents over a wide range of platforms;
 \item  adaptability to many languages;
 \item  widespread and free availability;
 \item  reliable support and maintenance by the \LaTeX3 project team.
 \end{itemize}
 These will be preserved and in many cases greatly enhanced by the new
 version which is being developed to fulfill the following requirements.
 \begin{itemize}
 \item
 It will provide a syntax that allows highly automated translation
 from popular \SGML{} DTDs into \LaTeX{} document classes (these
 will be provided as standard with the new version).

 The syntax of the new \LaTeX{} user-interface will, for example,
 support the \SGML{} concepts of `entity', `attribute' and `short
 reference' in such a way that these can be directly linked to the
 corresponding  \SGML{} features.

 \item
 It will support hypertext links and other features required for
 on-line structured documents using, for example, \HTML{} and \XML{}.

\item
 It will provide a straightforward style-designer interface to support
 both the specification of a wide variety of typographic requirements
 and the linking of entities in the generic mark-up of a document to
 the desired formatting.  These two parts of the design process will
 be clearly separated so that it is possible to specify different
 layouts for the same DTD.

 The language and syntax of this interface will be as natural as
 possible for a typographic designer.  As a result, this language
 could easily be interfaced to a visually-oriented, menu-driven
 specification system.

 This interface will also support \DSSSL{} specifications and
 style-sheet concepts such as those used with \HTML{} and \XML{}.

 \item
 It will provide an enhanced user-interface that allows expression of
 the typesetting requirements from a large range of subject areas. Some
 examples are listed here.

 \begin{itemize}
 \item The requirements of technical documentation (\eg offset layout,\\
   change bars, etc).
 \item The requirements of academic publishing in the humanities\\
   (critical text editions, etc).
 \item The requirements of structural formulas in chemistry.
 \item Advanced use of the mathematical-typesetting features of \TeX{}.
 \item The integration of graphical features, such as shading,
   within text.
 \item the integration of hypertext and other links in on-line
   documents using systems such as \HTML{}, \XML{} and \PDF{}.

 \end{itemize}

 Special care will be taken to ensure that this interface is
 extensible: this will be achieved by use of modular designs.


  \item
  It will provide a more robust author-interface. For example,
  artificial restrictions on the nesting of commands will be removed.
  Error handling will be improved by adding
  a more effective, interactive help system.

  \item
  It will provide access to arbitrary fonts from any family (such as
  the \PS{} and TrueType fonts) including a wide range of fonts for
  multi-lingual documents and the specialist glyphs required by
  documents in various technical and academic areas.

  \item
  The new interfaces will be documented in detail and the
  system will provide extensive catalogues of examples, carefully
  designed to make the learning time for new users (both designers and
  authors) as short as possible.

 \item
  The code itself will be thoroughly documented and it will be
  designed on modular principles.  Thus the system will be easy to
  maintain and to enhance.
\end{itemize}

The resulting new \LaTeX{} will, like the present version, be usable
with any standard \TeX{} system (or whatever replaces it) and so will
be freely available on a wide range of platforms.


\section{\LaTeX{} documentation}

\begingroup
\setlength{\parindent}{0pt}

A complete description of Standard \LaTeX{} can be found in:
\begin{citations}
\item[\LaTeX: A Document Preparation System]
   Leslie Lamport,\\ \AW, 2nd ed, 1994.
\item[The \LaTeX{} Companion]
   Mittelbach, Goossens with Braams, Carlisle and Rowley,\\ \AW, 2nd ed, 2004.
\end{citations}

A recent addition to the publications closely associated with the
project is:
\begin{citations}
\item[The \LaTeX{} Graphics Companion]
   Goossens, Mittelbach and Rahtz,\\ \AW, 1997.
\end{citations}

This \LaTeX{} distribution comes with documentation on several aspects
of the system.  The newer features of the system are described in
the following documents:
\begin{citations}
\item[\LaTeXe{} for authors]
   describes the new features of \LaTeX{} documents,
   in the file \texttt{usrguide.tex};
\item[\LaTeXe{} for class and package writers]
   describes how to produce \LaTeX{} classes and packages,
   in the file \texttt{clsguide.tex};
\item[\LaTeXe{} font selection]
   describes the new features of \LaTeX{} fonts for
   class and package writers,
   in the file \texttt{fntguide.tex}.
\end{citations}

For further contacts and sources of information on \TeX{} and
\LaTeX{}, see the addresses on Page~\pageref{contacts}.

\endgroup
\pagebreak


\section{The \LaTeX3 Project Fund}
\label{fund}

Although \LaTeX{} may be distributed freely, the production and
maintenance of the system does require expenditure of reasonably large
sums of money.  There are many necessities that need substantial
financing: examples are new or enhanced computing equipment and travel
to team meetings (the volunteers come from many different countries,
so getting together occasionally is a non-trivial exercise).

This is why we are appealing to you for contributions to the fund.
Any sum will be much appreciated; the amount need not be large as
small contributions add up to very useful amounts.  Contributions of
suitable equipment and software will also be of great value.  This
appeal is both to you as an individual author and to you as a member
of a group or as an employee: please encourage your department or your
employer to contribute towards sustaining our work.

We should like to see funded projects that make considerable use of
\LaTeX{} (\eg conferences and research teams who use it to publish
their work, and electronic research archives using it) include
contributions to this fund in their budgets.  %% e-print systems

We are also asking commercial organisations to assess the benefits
they gain from using, or distributing, a well-supported \LaTeX{} and
to make appropriate contributions to the fund in order that we can
continue to maintain and improve the product.  If you work for, or do
business with, such an organisation, please bring to the attention of
the relevant people the existence and needs of the project.

In particular, we ask that all the large number of organisations and
businesses that distribute \LaTeX{}, within other software or as part
of a CD-ROM collection, should consider pricing all products containing
\LaTeX{} at a level that enables them to make regular donations to the
fund from the profit on these items.  We also ask all authors and
publishers of books about \LaTeX{} to consider donating part of the
royalties to the fund.

Contributions should be sent to one of the following addresses:
\begin{quote}\small\label{addrs}
   \TeX{} Users Group, P.O. Box 2311, Portland, OR~97208-2311 USA\\
   Fax:~+1~503~223~3960 \ Email: \texttt{tug@tug.org}

 \noindent
   UK TUG, 1 Eymore Close, Selly Oak, Birmingham B29~4LB UK\\
   Fax: +44 121 476 2159 \ Email: \texttt{uktug-enquiries@tex.ac.uk}
\end{quote}

Cheques should be payable to the user group (TUG or UKTUG) and be
clearly marked as contributions to the \LaTeX3 fund.
Many thanks to all of you who have contributed in the past and thanks
in advance for your generosity in the future.

\section{Contacts and information}
\label{contacts}

In addition to the sources mentioned above, \LaTeX{} has its home page
on the World Wide Web at:
\begin{verbatim}
   http://www.latex-project.org/
\end{verbatim}
This page describes \LaTeX{} and the \LaTeX3 project, and contains
pointers to other \LaTeX{} resources, such as the user guides, the
\TeX{} Frequently Asked Questions, and the \LaTeX{} bugs database.

More general information, including contacts for local User Groups,
can be accessed via:
\begin{verbatim}
   http://www.tug.org/
\end{verbatim}

The electronic home of anything \TeX-related is the Comprehensive
\TeX{} Archive Network (CTAN).  This is a network of cooperating ftp
sites, with over a gigabyte of \TeX{} material:
\begin{verbatim}
   ftp://cam.ctan.org/tex-archive/
   ftp://dante.ctan.org/tex-archive/
   ftp://tug.ctan.org/tex-archive/
\end{verbatim}

For more information, see the \LaTeX{} home page.


\end{document}
